Die vorliegende Doktorarbeit besch\"aftigt sich mit der konzeptionellen Entwicklung, der Realisierung des Timing Systems und der systematischen Untersuchung des Bunch-to-Bucket (B2B) Transfer Systems f\"ur die geplante FAIR-Beschleunigeranlage und deren Synchrotrons.

FAIR, ``Facility for Antiproton and Ion Research``, ist eine im Bau befindliche, internationale Teilchenbeschleunigeranlage, die unter der Leitung von GSI Helmholtzzentrum f\"ur Schwerionenforschung GmbH errichtet wird. Sie hat zum Ziel, hochenergetischen Teilchenstrahlen, bestehend aus Proton bis Uranium, aus Antiproton oder auch seltenen Isotopen zu erzeugen. Der FAIR-Beschleunigerkomplex besteht aus vielen Synchrotrons mit unterschiedlichen Funktionalit\"aten und Aufgaben. Alle FAIR Synchrotrons
sind \"uber Transferkan\"ale miteinander verbunden, \"uber die ein Teilchenaustauch mit nahezu Lichtgeschwindigkeit erm\"oglicht wird. Das FAIR B2B Transfer System spielt daher ein zentrale Rolle, beim komplexen Transfer von Teilchenpaketen in umlaufende Buckets, bei allen zuk\"unftigen FAIR-Synchrotrons. Das System konzentriert sich zun\"achst auf den Teilchentransfer vom SIS18 zum SIS100. Dieser wird vorab am Beispiel des Transfers zwischen SIS18 zum ESR und vom ESR zum CRYRING getestet. Das System wird auf Basis der f\"ur FAIR
vorgesehen, technischen Infrastruktur entwickelt. Dazu z\"ahlen das FAIR-Low-Level Radio-Frequency (LLRF) System und das Kontrollsystem f\"ur FAIR. Das FAIR B2B Transfer System hat eine Schnittstelle zum FAIR-Maschinenschutzsystem (Machine Protection System), um die Synchrotrons vor Schaden und fatalen Fehlern zu sch\"utzen. Au\ss erdem wird der Status des Strahls und der Zeitpunkt der Strahlinjektion vom FAIR B2B Transfer System an die Ger\"ate der Strahldiagnose gemeldet.

Das FAIR B2B Transfer System nutzt einen zweistufigen Synchronisationsprozess, um den exakten Kickzeitpunkt zu bestimmen. In der ersten Stufe, der Grobsynchronisation gibt ein Synchronisationsfenster ein Zeitintervall vor, indem der Mittenversatz zwischen Teilchenpaketen und Buckets innerhalb der geforderten Toleranzgrenze ist. Innerhalb dieses Synchronisationsfensters m\"ussen nun die
Kicker zum richtigen Zeitpunkt gez\"undet werden, um die Teilchenpakte in die leeren Buckets zu schie\ss en. Das \"ubernimmt die Feinsynchronisation. Hierzu wird ein HF Signal benutzt, das die genaue Position der Teilchenpakete repr\"asentiert und die
Kicker exakt zum richtigen Zeitpunkt z\"undet. F\"ur die Grobsynchronisation wird die Phasendifferenz zwischen den Hochfrequenzsignalen (HF) des Quell- und Ziel-Synchrotrons gemessen. Das wird erreicht, indem man die HF-Signale beider Synchrotrons gegen ein campusweit verteiltes, im Pikosekundenbereich genaues Referenzsignal vermisst. Ist das Zahlenverh\"altnis der Umf\"ange beider Synchrotrons ein Integer, so bleibt die Phasendifferenz der beiden HF-Signal w\"ahrend des Transfers konstant. Um die richtige Phasendifferenz zu erreichen, ist dann lediglich eine azimutale Positionierung der Teilchenpakete im Quellsynchrotron oder der Buckets im Zielsynchrotron erforderlich. Das nennen wir die ``phase shift method``. Nach der exakten Positionierung, bleibt die Phasenverschiebung somit konstant und erm\"oglicht theoretisch ein unendlich langes Synchronisationsfenster. Wenn das Zahlenverh\"altnis kein Integer ergibt, ver\"andert sich die Phasendifferenz periodisch. Innerhalb einer Periode gibt es dann nur einen Zeitpunkt, zu dem die Zielphase erreicht wird. Davor und danach kommt es zu einem Mittenversatz zwischen Teilchenpaket und Bucket. Das nennen wir die ``frequency beating method``. Die Feinsynchronisation wird \"uber ein ``bucket indication signal`` erreicht. Das erste
Bucket wird \"uber das HF-Signal gekennzeichnet und die folgenden Buckets werden \"uber eine feste Verz\"ogerungszeit ausgez\"ahlt.
Das FAIR B2B Transfer System sieht einen ``B2B transfer master`` mit folgender Funktionalit\"at vor.
\begin{itemize}

\item Die Datenakquise (z.B. die HF-Phasen).
\item Die Datenberechnung (z.B. der Beginn des Synchronisationsfensters, die Ermittlung der erforderlichen Zielphasendifferenz zwischen zwei HF Systemen, die Berechnung der Phasenkorrektur f\"ur das BucketIndikationssignals,
etc.).
\item Die Verteilung von Daten (z.B. Start des Synchronisationsfensters).
\item Die \"Uberpr\"ufung des B2B Transfer Status.
\end{itemize}

Die Dissertation stellt vor allem die Grundidee, die grundlegende Vorgehensweise und das Realisierungskonzept des FAIR B2B Transfer System vor. Danach wird eine systematische Untersuchung des Systems durchgeführt. Da sich das System zun\"achst auf den Transfer vom SIS18 zum SIS100 konzentriert, werden die strahldynamischen Auswirkungen im Fall der ``phase shift method`` und der ``frequency beating method`` untersucht. Zus\"atzlich werden verschiedenen TriggerStrategien f\"ur den SIS18 Extraktion- und SIS100 Injektion Kickern analysiert. Diese Arbeit untersucht auch die Timing-Anforderungen an das System. Es wird die Genauigkeitsanforderung an den Startzeitpunkt des Synchronisationsfensters untersucht und das Netzwerk für das FAIR B2B Transfer System charakterisiert. Zum
Schluss wird ein Testaufbau f\"ur das System vorgestellt, der haupts\"achlich das Timing des FAIR B2B Transfer Systems \"uberpr\"uft. Die Messergebnisse werden ausgewertet und vorgestellt.

%Die vorliegende Doktorarbeit leistet einen Beitrag zur konzeptionellen Entwicklung, zur systematischen Untersuchung, zur timing system Realisierung vom FAIR Bunch-to-Bucket (B2B) Transfer System und zur Anwendung des Systemes auf FAIR Beschleuniger.
%
%FAIR, Facility für Antiprotonen und Ionenforschung, ist eine neue internationale Teilchenbeschleunigeranlage und im Bau an der GSI Helmholtzzentrum f\"ur Schwerionenforschung GmbH. Es zielt auf die hochenergetischen Strahlen von Antiproton bis Uran mit hohen Intensit\"aten zu herstellen. Der FAIR Beschleunigerkomplex besteht aus vielen Synchrotrons mit unterschiedlicher Funktionalit\"at. Deshalb spielt das FAIR B2B Transfer system eine wichtige Rolle f\"ur verschiedene komplexe bunch-to-bucket Transfer von FAIR Beschleuniger in die Zukunft. Das System konzentriert sich vor allem auf die \"Ubertragung vom SIS18 zum SIS100, aber es wird zum einen für die \"Ubertragung vom SIS18 zum ESR und vom ESR zum CRYRING getestet werden. Das System wird aufgrund der FAIR technische Basis entwickelt, das Low Leveral Radio Frequency und das Kontrollsystem von FAIR. Es koordiniert mit der Maschinenschutzsystem, um das SIS100/SIS300 vor Beschädigung oder inakzeptable Versagen zu schützt. Au\ss erdem untersucht es den Status von Strahl\"Ubertragung und zeigt es die tatsächliche Zeit des Stahleinspritzunges f\"ur die Beam Instrumentation.
% 
%Die FAIR B2B Transfer System besteht aus zwei Synchronisationsprozesse zusammengesetzt, die erste ist eine grobe Synchronisation und die zweiter ist eine feine Synchronisation. Die grobe Synchronisation gibt einen groben Zeitrahmen, innerhalb des Trauben mit einem  bunch-to-bucket Zentrum Diskrepanz kleiner als eine obere Schranke in den Eimer übertragt werden. Dieser Zeitrahmen ist die `` Synchronisationsfenster``. Innerhalb des Synchronisationsfensteres m\"ussen die Extraktion und Injektion Kicker Magnete an die richtigen Zeitpunkt ausl\"osen werden, um Trauben in die richtige leeren Eimer zu übertragen. Der Auslösungprozess des Kicker an die richtigen Zeitpunkt ist die ``feine Synchronization``.
%
%Die grobe Synchronisation beruht auf der Messung des Phasendifferenzes zwischen zwei Rf Systeme von zwei Synchrotrons, die mittels eines Campus weit verteilte Referenzsignal mit Pikosekunden-Pr\"aision erhaltet wird. Wenn das Umfangsverhältnis zwischen zwei Synchrotrons eine ganze Zahl ist, ist die Phasendifferenz zwischen zwei rf Systeme konstant. Die azimutale Positionierung von Trauben in der Quelle Synchrotron oder Bucket im Ziel Synchrotron m\"ussen einstellt werden, um die richtige Phasendifferenz zu erreichen. Dies wird als ``Phasenverschiebungsverfahren`` genannt. Nach der Phasenverschiebung ist der Phasendifferenz zwischen zwei RF-Systemen  	korrekt und das Synchronisationsfenster ist theoretisch unendlich. Wenn das Umfangsverhältnis zwischen zwei Synchrotrons nicht eine ganze Zahl ist, variiert der Phasendifferenz zwischen zwei RF-Systeme periodisch. Innerhalb einer Periode gibt es einen Zeitpunkt, wenn die Phasendifferenz das Ziel ist. Vor und nach diesem Zeitpunkt besteht die Zentrum Diskrepanz zwischen Trauben und Eimern. Dies wird als ``Frequenzüberlagerungsverfahren`` genannt. Die feine Synchronisation f\"uhrt auf einem Eimer Anzeigesignal für den ersten Eimer plus eine feste Verzögerung aus. Das FAIR B2B Transfersystem hat eine ``B2B Transfer Master``. Es ist verantwortlich f\"ur die folgenden Funktionen.
%\begin{itemize}
%
%\item Die Datensammlung (z.B. die HF-Phase).
%\item Die Datenrechnung (z.B. der Beginn des Synchronisationsfensters, die erforderliche Phasenverschiebung f\"ur die Zielphasendifferenz zwischen zwei RF-Systeme, die Phasenkorrektur f\"ur die Eimer Anzeigesignal und etc.).
%\item Die Datenumverteilung (z.B. der Start des Synchronisationsfensters).
%\item Die Statusuntersuchung der Strahl\"Ubertragung.
%\end{itemize}
%
%Die Dissertation stellt vor allem die Grundidee, die grundlegende Vorgehensweise und die Realisierung des FAIR B2B Transfer System. Zweitens wird die systematische Untersuchung des Systems durchgeführt. Das System konzentriert vor allem auf die \"Ubertragung vom SIS18 zum SIS100. Deshalb wird der Strahldynamik des B2B Übertragung vom SIS18 zum SIS100 f\"ur die Phasenverschiebung mit den Frequenzüberlagerungsverfahren und Phasenverschiebungsverfahren Methoden simulieren. Danach werden die SIS18 Extraktion und SIS100 Injektion Kickern mit verschiedene Auslösestrategien analysiert. Diese Dissertation erkl\"art auch die zeitlichen Zwänge die Genauigkeit der Beginn des Synchronisationsfensters und Charakterisierung des Netzes f\"ur die FAIR B2B Transfer System. Schließlich wird ein Testaufbau im Wesentlichen auf die Timing Aspekte konzentriert vorgestellt und das Testergebnis wird ausgewertet.