Die \textit{Facility for Antiproton and Ion Research} (FAIR) ist eine im Bau befindliche, internationale Teilchenbeschleunigeranlage, die unter der Leitung von der GSI Helmholtzzentrum f\"ur Schwerionenforschung GmbH errichtet wird. Sie hat zum Ziel, hochenergetische
Ionenstrahlen zu erzeugen. FAIR wird in der Lage sein Strahlen h\"ochster Intensit\"at für Elemente vom leichten Wasserstoff bis zum schweren Uran zur Verf\"ugung zu stellen und darüber hinaus in der Lage sein Antiprotonen und exotische Nuklide zu erzeugen. Die existierende Beschleunigeranlage der GSI wird der Injektor der FAIR"=Beschleuniger sein und umfasst den Linerabeschleuniger UNILAC und das Schwerionen"=Synchrotron SIS18. Desweiteren umfasst die GSI Beschleunigeranlage zwei Speicherringe, den Experimentierspeichering (ESR) und den CRYRING. Der FAIR"=Beschleunigerkomplex in seiner Startversion, besteht aus drei Ringbeschleunigern mit unterschiedlichen Funktionalit\"aten und Aufgaben. Der Prim\"arstrahltreiber ist das supraleitende Schwerionen"=Synchrotron SIS100, das die Beschleunigung von stabilen Elementen mit h\"ochster Intensit\"at erlaubt. Der Prim\"arestrahl wird dazu genutzt Sekund\"arteichen zu produzieren, die dem \textit{Collector Ring} (CR) \"uber Hochenergieseparatoren zugef\"uhrt werden. F\"ur die Pr\"aparation der Sekund\"arstrahlen und die Experimente dienen der CR und der \textit{High Energy Storage Ring} (HESR). Der CR dient zur Sekund\"arstrahl"=Akkumulation und verbessert die Stahlqualit\"at durch \textit{stochastic cooling}. Die Ringbeschleuniger der GSI und von FAIR haben sehr unterschiedliche Verh\"altnisse ihrer Umf\"ange. Zum Beispiel ist das Umfangsverh\"altnis zwischen dem SIS18 und dem SIS100 ganzzahlig, zwischen dem SIS18 und dem ESR ann\"ahernd ganzzahlig und zwischen dem CR und dem HESR ist es weit weg von einem ganzzahligem Verh\"altnis. Alle FAIR"=Ringbeschleuniger sind \"uber Strahltransportlinien, die Targets zur Produktion von Sekund\"arteilchen und den zugeh\"origen Hochenergieseparatoren verbunden. F\"ur FAIR ist es nicht nur erforderlich Prim\"arstrahlen vom einen zum anderen Ring zu transferieren, sondern auch Sekund\"arstrahlen, wie Antiprotonen oder exotische Nuklide die im Antiprotonen"=Target, im Fragmentseparator oder im Suprafragmentseparator erzeugt werden, wieder in Speicherringe einzufangen. Zudem m\"ussen \textit{bunch} von einem Ring in \textit{bucket} eines anderen Rings, innerhalb einer bestimmten Zeit (z.B. unter \SI{10}{\ms} in fast allen FAIR"=Anwendungsf\"allen) und mit einem akzeptablen \textit{Bunch"=to"=Bucket}"=Injektions"=Mittenversatz (z.B. unter $\pm1^\circ$ in den meisten FAIR"=Anwendungsf\"allen) transferiert werden. Daher ist ein flexibles \textit{FAIR Bunch"=to"=Bucket (B2B) transfer system} erforderlich, um die verschiedenen und komplexen B2B"=Transfers zwischen den zuk\"unftigen FAIR"=Ringen realisieren zu k\"onnen. Im Fokus dieser Arbeit ist nat\"urlich der Teilchentransfer vom SIS18 zum SIS100, welcher am Beispiel des Transfers zwischen SIS18 zum ESR und vom ESR zum CRYRING an der GSI getestet werden kann. Das System wird auf Basis der f\"ur FAIR vorgesehen, technischen Infrastruktur entwickelt. Dazu z\"ahlen das FAIR"=Low"=Level"=Radio"=Frequency (LLRF)"=System und das Kontrollsystem f\"ur FAIR. Das \textit{FAIR B2B transfer system} hat eine Schnittstelle zum FAIR"=Maschinenschutzsystem (\textit{machine protection system}), welches das SIS100 und die nachgeschalteten Beschleuniger und Experimente vor Schaden durch Fehlerfunktionen und dadurch bedingten Prim\"arstrahlverlust bei hohen Intensit\"aten bewahrt. Au\ss{}erdem wird der Status des Strahls und der Zeitpunkt der Strahlinjektion vom \textit{FAIR B2B transfer system} an die Ger\"ate der Strahldiagnose gemeldet.

Diese Doktorarbeit stellt vor allem die Grundidee, das grundlegende Verfahren und die konzeptionelle Realisierung des \textit{FAIR B2B transfer system} vor. Darüber hinaus werden die Anforderungen an das Timing analytisch ermittelt und vorgestellt. Das \textit{FAIR B2B transfer system} nutzt einen zweistufigen Synchronisationsprozess, um den exakten Kickzeitpunkt zu bestimmen. In der ersten Stufe, der „Grobsynchronisation“ gibt ein Synchronisationsfenster ein Zeitintervall vor, indem der B2B"=Injektions"=Mittenversatz zwischen \textit{bunch} und \textit{bucket} innerhalb der geforderten Toleranzgrenze bleibt. Innerhalb dieses Synchronisationsfensters m\"ussen die Kicker zum richtigen Zeitpunkt ausgel\"ost werden, um die \textit{bunch} in die richtigen, leeren \textit{bucket} der Zielmaschine zu schie\ss{}en. Das \"ubernimmt die sogenannte „Feinsynchronisation“.
Die „Feinsynchronisation“ ist das \textit{bucket indication signal}, welches \"uber ein festes Delay verz\"ogert wird. Das \textit{bucket indication signal} wird von den Hochfrequenzsignalen (HF"=Signal) der Umlauffrequenzen abgeleitet und kennzeichnet immer das erste \textit{bucket}. Ein weiteres Delay wird dazu benutzt, um die folgenden, zu bef\"ullenden \textit{bucket} zu kennzeichnen. 

F\"ur die Grobsynchronisation wird die Phasendifferenz zwischen den HF"=Signalen der Quell"= und Zielmaschine gemessen. Die Phasendifferenz erhält man, indem man die Phasenabweichung der HF"=Signalen beider Ringe gegen ein campusweit verteiltes \textit{synchronization reference signal} vermisst. Wenn das Umfangsverh\"altnis  beider Ringe ganzzahlig ist, bleibt die Phasendifferenz der beiden HF"=Signale w\"ahrend des Transfers konstant. Um die richtige Phasendifferenz zu erreichen, muss die Phase eines (oder beider) HF"=Systeme mithilfe einer Frequenzmodulation verschoben werden. Das nennen wir die \textit{phase shift method}. Nach der exakten Ausrichtung der Phasen, bleibt die gew\"unschte Phasendifferenz konstant und erm\"oglicht, theoretisch ein unendlich langes Synchronisationsfenster. Wenn das Umfangsverh\"altnis beider Ringe nicht ganzzahlig ist, ver\"andert sich die Phasendifferenz periodisch. Innerhalb einer Periode gibt es dann nur einen Zeitpunkt, zu dem die Zielphase erreicht wird. Davor und danach kommt es zu einem B2B"=Injektions"=Mittenversatz zwischen \textit{bunch} und \textit{bucket}. Das nennen wir die \textit{frequency beating method}. Diese ist auch anwendbar, wenn das Umfangsverh\"altnis beider Ringe ganzzahlig ist. In diesem Fall wird die HF"=Frequenz eines (oder beider) HF"=Systeme am Ende der Beschleunigungsrampe leicht verstimmt, sodass sich eine Schwebungsfrequenz zwischen den HF"=Systemen der Quell"= und Zielmaschine ergibt. F\"ur FAIR wird die \textit{frequency beating method} pr\"aferiert, weil diese Methode f\"ur s\"amtliche Transferszenarien bei FAIR anwendbar ist. Au\ss{}erdem kann mit dieser Methode Zeit gespart werden, weil die Frequenzverstimmung w\"ahrend der Beschleunigungsrampe durchgef\"uhrt wird. Zur Erinnerung: Bei der \textit{phase shift method} muss die Frequenzmodulation auf \textit{rf flattop} langsam genug ausgef\"uhrt werden, damit der Strahl stabil bleibt. Das kostet viel Zeit.  Dennoch gibt es auch Vorteile, die f\"ur die \textit{phase shift method} sprechen. Das Synchronisationsfenster ist theoretisch unendlich lang und der B2B"=Injektions"=Mittenversatz ist nahezu „Null“. Au\ss{}erdem ist die Dauer der Frequenzmodulation vorab bekannt und der Transfer-Zeitpunkt ist exakt bestimmbar. Vorteilhaft ist auch, dass die gew\"unschte Phase des HF"=Systems  sprungartig eingestellt werden kann, wenn sich kein \textit{bunch} im Ring befindet.

Im Rahmen dieser Doktorarbeit wird eine systematische Untersuchung der strahldynamischen Aspekte, der zeitlichen Anforderungen an den B2B"=Transfer"=Prozess und der Trigger"=Szenarien f\"ur die Kicker"=Ausl\"osung durchgef\"uhrt. Die Timing"=Betrachtungen ber\"ucksichtigen die erforderliche Genauigkeit f\"ur den Beginn des Synchronisationsfensters, die Charakterisierung des \textit{White Rabbit} Netzwerks, das Flussdiagramm mit Timing"=Bedingungen f\"ur das \textit{FAIR B2B transfer system}. Danach wird ein Messaufbau zur Bestimmung des Timings vorgestellt, der dazu verwendet wird die Firmware, die auf einer \textit{soft"=CPU} ausgef\"uhrt wird, hinsichtlich der Einhaltung der Timing"= und Funktionsanforderungen zu \"uberpr\"ufen. Zum Schluss werden alle FAIR"=Anwendungsf\"alle, bei denen die \textit{frequency beating method} verwendet wird, er\"ortert.

Diese Doktorarbeit spielt eine wichtige Rolle bei der Realisierung der ver\"offentlichten Version des \textit{FAIR B2B transfer system} und der weiteren praktischen Anwendung des Systems f\"ur alle FAIR"=Transferszenarien.

