Die vorliegende Doktorarbeit leistet einen Beitrag zur konzeptionellen Entwicklung, zur systematischen Untersuchung, zur timing system Realisierung vom FAIR Bunch-to-Bucket (B2B) Transfer System und zur Anwendung des Systemes auf FAIR Beschleuniger.

FAIR, Facility für Antiprotonen und Ionenforschung, ist eine neue internationale Teilchenbeschleunigeranlage und im Bau an der GSI Helmholtzzentrum f\"ur Schwerionenforschung GmbH. Es zielt auf die hochenergetischen Strahlen von Antiproton bis Uran mit hohen Intensit\"aten zu herstellen. Der FAIR Beschleunigerkomplex besteht aus vielen Synchrotrons mit unterschiedlicher Funktionalit\"at. Deshalb spielt das FAIR B2B Transfer system eine wichtige Rolle f\"ur verschiedene komplexe bunch-to-bucket Transfer von FAIR Beschleuniger in die Zukunft. Das System konzentriert sich vor allem auf die \"Ubertragung vom SIS18 zum SIS100, aber es wird zum einen für die \"Ubertragung vom SIS18 zum ESR und vom ESR zum CRYRING getestet werden. Das System wird aufgrund der FAIR technische Basis entwickelt, das Low Leveral Radio Frequency und das Kontrollsystem von FAIR. Es koordiniert mit der Maschinenschutzsystem, um das SIS100/SIS300 vor Beschädigung oder inakzeptable Versagen zu schützt. Au\ss erdem untersucht es den Status von Strahl\"Ubertragung und zeigt es die tatsächliche Zeit des Stahleinspritzunges f\"ur die Beam Instrumentation.
 
Die FAIR B2B Transfer System besteht aus zwei Synchronisationsprozesse zusammengesetzt, die erste ist eine grobe Synchronisation und die zweiter ist eine feine Synchronisation. Die grobe Synchronisation gibt einen groben Zeitrahmen, innerhalb des Trauben mit einem  bunch-to-bucket Zentrum Diskrepanz kleiner als eine obere Schranke in den Eimer übertragt werden. Dieser Zeitrahmen ist die `` Synchronisationsfenster``. Innerhalb des Synchronisationsfensteres m\"ussen die Extraktion und Injektion Kicker Magnete an die richtigen Zeitpunkt ausl\"osen werden, um Trauben in die richtige leeren Eimer zu übertragen. Der Auslösungprozess des Kicker an die richtigen Zeitpunkt ist die ``feine Synchronization``.

Die grobe Synchronisation beruht auf der Messung des Phasendifferenzes zwischen zwei Rf Systeme von zwei Synchrotrons, die mittels eines Campus weit verteilte Referenzsignal mit Pikosekunden-Pr\"aision erhaltet wird. Wenn das Umfangsverhältnis zwischen zwei Synchrotrons eine ganze Zahl ist, ist die Phasendifferenz zwischen zwei rf Systeme konstant. Die azimutale Positionierung von Trauben in der Quelle Synchrotron oder Bucket im Ziel Synchrotron m\"ussen einstellt werden, um die richtige Phasendifferenz zu erreichen. Dies wird als ``Phasenverschiebungsverfahren`` genannt. Nach der Phasenverschiebung ist der Phasendifferenz zwischen zwei RF-Systemen  	korrekt und das Synchronisationsfenster ist theoretisch unendlich. Wenn das Umfangsverhältnis zwischen zwei Synchrotrons nicht eine ganze Zahl ist, variiert der Phasendifferenz zwischen zwei RF-Systeme periodisch. Innerhalb einer Periode gibt es einen Zeitpunkt, wenn die Phasendifferenz das Ziel ist. Vor und nach diesem Zeitpunkt besteht die Zentrum Diskrepanz zwischen Trauben und Eimern. Dies wird als ``Frequenzüberlagerungsverfahren`` genannt. Die feine Synchronisation f\"uhrt auf einem Eimer Anzeigesignal für den ersten Eimer plus eine feste Verzögerung aus. Das FAIR B2B Transfersystem hat eine ``B2B Transfer Master``. Es ist verantwortlich f\"ur die folgenden Funktionen.
\begin{itemize}

\item Die Datensammlung (z.B. die HF-Phase).
\item Die Datenrechnung (z.B. der Beginn des Synchronisationsfensters, die erforderliche Phasenverschiebung f\"ur die Zielphasendifferenz zwischen zwei RF-Systeme, die Phasenkorrektur f\"ur die Eimer Anzeigesignal und etc.).
\item Die Datenumverteilung (z.B. der Start des Synchronisationsfensters).
\item Die Statusuntersuchung der Strahl\"Ubertragung.
\end{itemize}

Die Dissertation stellt vor allem die Grundidee, die grundlegende Vorgehensweise und die Realisierung des FAIR B2B Transfer System. Zweitens wird die systematische Untersuchung des Systems durchgeführt. Das System konzentriert vor allem auf die \"Ubertragung vom SIS18 zum SIS100. Deshalb wird der Strahldynamik des B2B Übertragung vom SIS18 zum SIS100 f\"ur die Phasenverschiebung mit den Frequenzüberlagerungsverfahren und Phasenverschiebungsverfahren Methoden simulieren. Danach werden die SIS18 Extraktion und SIS100 Injektion Kickern mit verschiedene Auslösestrategien analysiert. Diese Dissertation erkl\"art auch die zeitlichen Zwänge die Genauigkeit der Beginn des Synchronisationsfensters und Charakterisierung des Netzes f\"ur die FAIR B2B Transfer System. Schließlich wird ein Testaufbau im Wesentlichen auf die Timing Aspekte konzentriert vorgestellt und das Testergebnis wird ausgewertet.