Die Facility for Antiproton and Ion Research (FAIR), ist eine im Bau befindliche, internationale Teilchenbeschleunigeranlage, die unter der Leitung von der GSI Helmholtzzentrum f\"ur Schwerionenforschung GmbH errichtet wird. Sie hat zum Ziel, hochenergetische
Ionenstrahlen zu erzeugen. Es wird nicht nur m\"oglich sein, die Elemente des Periodensystems von Wasserstoff bis Uran mit hohen Intensit\"aten zu beschleunigen, sondern auch Antiproton und seltene Isotope. Die existierende Anlage umfasst den SIS18 und den ESR. Der FAIR-Beschleunigerkomplex in seiner endg\"ultigen Ausbaustufe, besteht aus vielen Ringbeschleuniger mit unterschiedlichen Funktionalit\"aten und Aufgaben. Beispielsweise wird FAIR aus folgenden Ringbeschleuniger bestehen: Dem Synchrotron SIS100 und Speicherringe wie der High Energy Storage Ring (HESR) und der Collector Ring (CR). Das Zahlenverh\"altnis der Ring-Umf\"ange ist teilweise willk\"urlich. Zum Beispiel ist das Umfangsverh\"altnis zwischen dem SIS18 und dem SIS100 ein ganzzahlig. Zwischen dem SIS18 und dem ESR ann\"ahernd ein ganzzahlig. Zwischen dem CR und dem HESR ist es weit weg von einem ganzzahligem Verh\"altnis. Alle FAIR Ringbeschleuniger sind \"uber Transferkan\"ale miteinander verbunden. F\"ur FAIR ist es nicht nur erforderlich Prim\"arstrahlen vom einen zum anderen Ring zu transferieren, auch Sekund\"arstrahlen, wie Antiprotonen oder seltene Isotope die im Antiprotonen-Target, im FRS oder im Super FRS erzeugt werden, m\"ussen wieder in Speicherringen eingefangen werden. Zudem m\"ussen Bunche von einem Ring in Buckets eines anderen Rings, innerhalb einer bestimmten Zeit (z.B. unter \SI{10}{\ms} in fast allen FAIR-Anwendungsf\"allen) und mit einem akzeptablen bunch-to-bucket Injektions-Mittenversatz (z.B. unter $\pm1^\circ$ in den meisten FAIR-Anwendungsf\"allen) transferiert werden. Daher ist ein flexibles FAIR Bunch-to-Bucket (B2B) Transfer-System erforderlich um die diversen, komplexen bunch-to-bucket Transfers zwischen den zuk\"unftigen FAIR-Ringen realisieren zu k\"onnen. Das System konzentriert sich zun\"achst auf den Teilchentransfer vom SIS18 zum SIS100. Dieser wird vorab, am Beispiel des Transfers zwischen SIS18 zum ESR und vom ESR zum CRYRING getestet. Das System wird auf Basis der f\"ur FAIR vorgesehen, technischen Infrastruktur entwickelt. Dazu z\"ahlen das FAIR-Low-Level Radio-Frequency (LLRF) System und das Kontrollsystem f\"ur FAIR. Das FAIR B2B-Transfer-System hat eine Schnittstelle zum FAIR-Maschinenschutzsystem (Machine Protection System), welchen das SIS100 und die nachgeschalteten Beschleuniger und Experimente vor Schaden bewahrt. Au\ss{}erdem wird der Status des Strahls und der Zeitpunkt der Strahlinjektion vom FAIR B2B-Transfer-System an die Ger\"ate der Strahldiagnose gemeldet.

Das FAIR B2B-Transfer-System nutzt einen zweistufigen Synchronisationsprozess, um den exakten Kickzeitpunkt zu bestimmen. In der ersten Stufe, der „Grobsynchronisation“ gibt ein Synchronisationsfenster ein Zeitintervall vor, indem der Mittenversatz zwischen Bunch und Bucket innerhalb der geforderten Toleranzgrenze ist. Innerhalb dieses Synchronisationsfensters m\"ussen die Kicker zum richtigen Zeitpunkt gez\"undet werden, um die Teilchenpakte in die leeren Buckets zu schie\ss{}en. Das \"ubernimmt die „Feinsynchronisation“.
Die „Feinsynchronisation“ ist das bucket-indication-signal, welches \"uber ein festes Delay verz\"ogert wird. Das bucket-indication-signal ist abgeleitet vom Hochfrequenzsignalen (HF)-Signals der Umlauffrequenz und kennzeichnet immer das erste Bucket. Ein festes Delay wird dazu benutzt, um die folgenden, zu bef\"ullenden Buckets zu kennzeichnen. 

F\"ur die Grobsynchronisation wird die Phasendifferenz zwischen den HF-Signalen des Quell- und Ziel-Synchrotrons gemessen. Das wird erreicht, indem man die Phasenabweichung der HF-Signale beider Ringe gegen ein campusweit verteiltes Synchronization-Reference-Signal vermisst. Ist das Zahlenverh\"altnis der Umf\"ange beider Ringe ein Integer, so bleibt die Phasendifferenz der beiden HF-Signal w\"ahrend des Transfers konstant. Um die richtige Phasendifferenz zu erreichen, muss die Phase eines (oder beider) HF-Systeme mithilfe einer Frequenzmodulation, nach vorne oder zur\"uck geschoben werden. Das nennen wir die „phase shift method“. Nach der exakten Positionierung, bleibt die Phasenverschiebung somit konstant und erm\"oglicht theoretisch ein unendlich langes Synchronisationsfenster. Wenn das Zahlenverh\"altnis der Ringumf\"ange kein Integer ergibt, ver\"andert sich die Phasendifferenz periodisch. Innerhalb einer Periode gibt es dann nur einen Zeitpunkt, zu dem die Zielphase erreicht wird. Davor und danach kommt es zu einem Mittenversatz zwischen Bunch und Bucket. Das nennen wir die „frequency beating method“. Diese ist auch anwendbar, wenn das Zahlenverh\"altnis der Ringumf\"ange ein Integer ergibt. In diesem Fall wird die HF-Frequenz eines (oder beider) HF-Systeme am Ende der Beschleunigungsrampe leicht verstimmt, sodass sich eine Schwebungsfrequenz ergibt. F\"ur FAIR wird die „frequency beating method“ pr\"aferiert, weil es f\"ur den Prim\"arstrahl-Transfer zwischen zwei Ringen keine Einschr\"ankungen beim Zahlenverh\"altnis der Ringumf\"ange gibt. Auch ist es m\"oglich Sekund\"arstrahlen zu transferieren. Au\ss{}erdem wird die Frequenzverstimmung die langsam genug ausgef\"uhrt werden muss, damit der Strahl folgen kann, nicht w\"ahrend des bunch-to-bucket-transfer-Prozesses durchgef\"uhrt. Dennoch gibt es auch Vorteile, die f\"ur die „phase shift method“ sprechen. Das Synchronisationsfenster ist relativ lang und der bunch-to-bucket injektions Mittenversatz ist nahezu „Null“. Au\ss{}erdem ist die Dauer der Frequenzverstimmung vorab bekannt und die Transferzeit ist vorhersehbar. Die gew\"unschte Phase des HF-Systems kann sprungartig eingestellt werden, wenn sich kein Bunch im Ring befindet.

Diese Doktorarbeit stellt vor allem die Grundidee, das grundlegende Verfahren und die konzeptionelle Realisierung des FAIR B2B-Transfer-System vor. In zweiter Linie wird eine systematische Untersuchung unter strahldynamischen Gesichtspunkten und unter
Ber\"ucksichtigung der zeitlichen Anforderungen an das Ausl\"osen der Kicker durchgef\"uhrt. Die Timing-Betrachtungen umfassen die erforderliche Genauigkeit f\"ur den Beginn des Synchronisationsfenster, die Charakterisierung des White-Rabbit-Netzwerks f\"ur den B2B-Transfer, die Firmware und die Timing-Bedingungen f\"ur das System. Danach wird ein Messaufbau zur Bestimmung des Timings vorgestellt, der dazu verwendet wird die Firmware, die auf einer Soft-CPU ausgef\"uhrt wird, hinsichtlich der Einhaltung der Timing- und Funktions-Anforderungen zu \"uberpr\"ufen. Zum Schluss werden alle FAIR-Anwendungsf\"alle, bei denen die „frequency beating method“ verwendet wird, er\"ortert.

Diese Doktorarbeit spielt eine wichtige Rolle bei der Realisierung der ver\"offentlichten Version des FAIR B2B-Transfer-System und der weiteren praktischen Anwendung des Systems f\"ur alle FAIR-Anwendungsf\"alle.

%Die vorliegende Doktorarbeit besch\"aftigt sich mit der konzeptionellen Entwicklung, der Realisierung des Timing Systems und der systematischen Untersuchung des Bunch-to-Bucket (B2B) Transfer Systems f\"ur die geplante FAIR-Beschleunigeranlage und deren Synchrotrons.
%
%FAIR, ``Facility for Antiproton and Ion Research``, ist eine im Bau befindliche, internationale Teilchenbeschleunigeranlage, die unter der Leitung von GSI Helmholtzzentrum f\"ur Schwerionenforschung GmbH errichtet wird. Sie hat zum Ziel, hochenergetische Teilchenstrahlen aus Ionen zu erzeugen. Es wird nicht nur m\"oglich sein, die Elemente des Periodensystems von Wasserstoff bis Uran zu beschleunigen, sondern auch sehr exotische Ionen, wie Antiproton oder auch seltene Isotope mit hohen Intensit\"aten. Der FAIR-Beschleunigerkomplex besteht aus vielen Synchrotrons mit unterschiedlichen Funktionalit\"aten und Aufgaben. Alle FAIR Synchrotrons
%sind \"uber Transferkan\"ale miteinander verbunden, \"uber die ein Teilchenaustauch mit nahezu Lichtgeschwindigkeit erm\"oglicht wird. Das FAIR B2B Transfer System spielt daher ein zentrale Rolle, beim komplexen Transfer von Teilchenpaketen in umlaufende Buckets, bei allen zuk\"unftigen FAIR-Synchrotrons. Das System konzentriert sich zun\"achst auf den Teilchentransfer vom SIS18 zum SIS100. Dieser wird vorab am Beispiel des Transfers zwischen SIS18 zum ESR und vom ESR zum CRYRING getestet. Das System wird auf Basis der f\"ur FAIR
%vorgesehen, technischen Infrastruktur entwickelt. Dazu z\"ahlen das FAIR-Low-Level Radio-Frequency (LLRF) System und das Kontrollsystem f\"ur FAIR. Das FAIR B2B Transfer System hat eine Schnittstelle zum FAIR-Maschinenschutzsystem (Machine Protection System), um die Synchrotrons vor erheblichen Strahlverlust und dadurch bedingt vor Schaden zu bewahren. Au\ss erdem wird der Status des Strahls und der Zeitpunkt der Strahlinjektion vom FAIR B2B Transfer System an die Ger\"ate der Strahldiagnose gemeldet.
%
%Die Dissertation stellt vor allem die Grundideeund das grundlegende Verfahren FAIR B2B Transfer System vor und macht einen Realisierungsvorschlag, der die Timing-Anforderungen ber\"ucksichtigt. Danach wird eine systematische Untersuchung des Systems durchgef\"uhrt. Da sich das System zun\"achst auf den Transfer vom SIS18 zum SIS100 konzentriert, werden die strahldynamischen Auswirkungen im Fall der ``phase shift method`` und der ``frequency beating method`` untersucht. Zus\"atzlich werden verschiedenen TriggerStrategien f\"ur den SIS18 Extraktion- und SIS100 Injektion Kickern analysiert. Diese Arbeit untersucht auch die Timing-Anforderungen an das System, die Genauigkeitsanforderung an den Startzeitpunkt des Synchronisationsfensters und es wird das Netzwerk f\"ur das FAIR B2B Transfer System charakterisiert. Zum Schluss wird ein Testaufbau f\"ur das System vorgestellt, der haupts\"achlich das Timing des FAIR B2B Transfer Systems \"uberpr\"uft. Die Messergebnisse werden ausgewertet und vorgestellt.
%
%Das FAIR B2B Transfer System nutzt einen zweistufigen Synchronisationsprozess, um den exakten Kickzeitpunkt zu bestimmen. In der ersten Stufe, der Grobsynchronisation gibt ein Synchronisationsfenster ein Zeitintervall vor, indem der Mittenversatz zwischen Teilchenpaketen und Buckets innerhalb der geforderten Toleranzgrenze ist. Innerhalb dieses Synchronisationsfensters m\"ussen nun die
%Kicker zum richtigen Zeitpunkt gez\"undet werden, um die Teilchenpakte in die leeren Buckets zu schie\ss en. Das \"ubernimmt die Feinsynchronisation. Hierzu wird ein HF Signal benutzt, um die genaue Position der Teilchenpakete zu bestimmen und die Kicker pr\"azise zum richtigen Zeitpunkt z\"unden. 
%
%F\"ur die Grobsynchronisation wird die Phasendifferenz zwischen den Hochfrequenzsignalen (HF) des Quell- und Ziel-Synchrotrons gemessen. Das wird erreicht, indem man die HF-Signale beider Synchrotrons gegen ein campusweit verteiltes, im Pikosekundenbereich genaues Referenzsignal vermisst. Ist das Zahlenverh\"altnis der Umf\"ange beider Synchrotrons ein Integer, so bleibt die Phasendifferenz der beiden HF-Signal w\"ahrend des Transfers konstant. Um die richtige Phasendifferenz zu erreichen, ist dann lediglich eine azimutale Positionierung der Teilchenpakete im Quellsynchrotron oder der Buckets im Zielsynchrotron erforderlich. Das nennen wir die ``phase shift method``. Nach der exakten Positionierung, bleibt die Phasenverschiebung somit konstant und erm\"oglicht theoretisch ein unendlich langes Synchronisationsfenster. Wenn das Zahlenverh\"altnis kein Integer ergibt, ver\"andert sich die Phasendifferenz periodisch. Innerhalb einer Periode gibt es dann nur einen Zeitpunkt, zu dem die Zielphase erreicht wird. Davor und danach kommt es zu einem Mittenversatz zwischen Teilchenpaket und Bucket. Das nennen wir die ``frequency beating method``. Die Feinsynchronisation wird \"uber ein ``bucket indication signal`` erreicht. Das erste
%Bucket wird \"uber das HF-Signal gekennzeichnet und die folgenden Buckets werden \"uber eine feste Verz\"ogerungszeit ausgez\"ahlt.
%Das FAIR B2B Transfer System sieht einen ``B2B transfer master`` mit folgender Funktionalit\"at vor.
%\begin{itemize}
%
%\item Die Datenakquise (z.B. die HF-Phasen).
%\item Die Datenberechnung (z.B. der Beginn des Synchronisationsfensters, die Ermittlung der erforderlichen Zielphasendifferenz zwischen zwei HF Systemen, die Berechnung der Phasenkorrektur f\"ur das BucketIndikationssignals,
%etc.).
%\item Die Verteilung von Daten (z.B. Start des Synchronisationsfensters).
%\item Die \"Uberpr\"ufung des B2B Transfer Status.
%\end{itemize}
%
%
%%Die vorliegende Doktorarbeit leistet einen Beitrag zur konzeptionellen Entwicklung, zur systematischen Untersuchung, zur timing system Realisierung vom FAIR Bunch-to-Bucket (B2B) Transfer System und zur Anwendung des Systemes auf FAIR Beschleuniger.
%%
%%FAIR, Facility f\"ur Antiprotonen und Ionenforschung, ist eine neue internationale Teilchenbeschleunigeranlage und im Bau an der GSI Helmholtzzentrum f\"ur Schwerionenforschung GmbH. Es zielt auf die hochenergetischen Strahlen von Antiproton bis Uran mit hohen Intensit\"aten zu herstellen. Der FAIR Beschleunigerkomplex besteht aus vielen Synchrotrons mit unterschiedlicher Funktionalit\"at. Deshalb spielt das FAIR B2B Transfer system eine wichtige Rolle f\"ur verschiedene komplexe bunch-to-bucket Transfer von FAIR Beschleuniger in die Zukunft. Das System konzentriert sich vor allem auf die \"Ubertragung vom SIS18 zum SIS100, aber es wird zum einen f\"ur die \"Ubertragung vom SIS18 zum ESR und vom ESR zum CRYRING getestet werden. Das System wird aufgrund der FAIR technische Basis entwickelt, das Low Leveral Radio Frequency und das Kontrollsystem von FAIR. Es koordiniert mit der Maschinenschutzsystem, um das SIS100/SIS300 vor Besch\"adigung oder inakzeptable Versagen zu sch\"utzt. Au\ss erdem untersucht es den Status von Strahl\"Ubertragung und zeigt es die tats\"achliche Zeit des Stahleinspritzunges f\"ur die Beam Instrumentation.
%% 
%%Die FAIR B2B Transfer System besteht aus zwei Synchronisationsprozesse zusammengesetzt, die erste ist eine grobe Synchronisation und die zweiter ist eine feine Synchronisation. Die grobe Synchronisation gibt einen groben Zeitrahmen, innerhalb des Trauben mit einem  bunch-to-bucket Zentrum Diskrepanz kleiner als eine obere Schranke in den Eimer \"ubertragt werden. Dieser Zeitrahmen ist die `` Synchronisationsfenster``. Innerhalb des Synchronisationsfensteres m\"ussen die Extraktion und Injektion Kicker Magnete an die richtigen Zeitpunkt ausl\"osen werden, um Trauben in die richtige leeren Eimer zu \"ubertragen. Der Ausl\"osungprozess des Kicker an die richtigen Zeitpunkt ist die ``feine Synchronization``.
%%
%%Die grobe Synchronisation beruht auf der Messung des Phasendifferenzes zwischen zwei Rf Systeme von zwei Synchrotrons, die mittels eines Campus weit verteilte Referenzsignal mit Pikosekunden-Pr\"aision erhaltet wird. Wenn das Umfangsverh\"altnis zwischen zwei Synchrotrons eine ganze Zahl ist, ist die Phasendifferenz zwischen zwei rf Systeme konstant. Die azimutale Positionierung von Trauben in der Quelle Synchrotron oder Bucket im Ziel Synchrotron m\"ussen einstellt werden, um die richtige Phasendifferenz zu erreichen. Dies wird als ``Phasenverschiebungsverfahren`` genannt. Nach der Phasenverschiebung ist der Phasendifferenz zwischen zwei RF-Systemen  	korrekt und das Synchronisationsfenster ist theoretisch unendlich. Wenn das Umfangsverh\"altnis zwischen zwei Synchrotrons nicht eine ganze Zahl ist, variiert der Phasendifferenz zwischen zwei RF-Systeme periodisch. Innerhalb einer Periode gibt es einen Zeitpunkt, wenn die Phasendifferenz das Ziel ist. Vor und nach diesem Zeitpunkt besteht die Zentrum Diskrepanz zwischen Trauben und Eimern. Dies wird als ``Frequenz\"uberlagerungsverfahren`` genannt. Die feine Synchronisation f\"uhrt auf einem Eimer Anzeigesignal f\"ur den ersten Eimer plus eine feste Verz\"ogerung aus. Das FAIR B2B Transfersystem hat eine ``B2B Transfer Master``. Es ist verantwortlich f\"ur die folgenden Funktionen.
%%\begin{itemize}
%%
%%\item Die Datensammlung (z.B. die HF-Phase).
%%\item Die Datenrechnung (z.B. der Beginn des Synchronisationsfensters, die erforderliche Phasenverschiebung f\"ur die Zielphasendifferenz zwischen zwei RF-Systeme, die Phasenkorrektur f\"ur die Eimer Anzeigesignal und etc.).
%%\item Die Datenumverteilung (z.B. der Start des Synchronisationsfensters).
%%\item Die Statusuntersuchung der Strahl\"Ubertragung.
%%\end{itemize}
%%
%%Die Dissertation stellt vor allem die Grundidee, die grundlegende Vorgehensweise und die Realisierung des FAIR B2B Transfer System. Zweitens wird die systematische Untersuchung des Systems durchgef\"uhrt. Das System konzentriert vor allem auf die \"Ubertragung vom SIS18 zum SIS100. Deshalb wird der Strahldynamik des B2B Übertragung vom SIS18 zum SIS100 f\"ur die Phasenverschiebung mit den Frequenz\"uberlagerungsverfahren und Phasenverschiebungsverfahren Methoden simulieren. Danach werden die SIS18 Extraktion und SIS100 Injektion Kickern mit verschiedene Ausl\"osestrategien analysiert. Diese Dissertation erkl\"art auch die zeitlichen Zw\"ange die Genauigkeit der Beginn des Synchronisationsfensters und Charakterisierung des Netzes f\"ur die FAIR B2B Transfer System. Schlie\ss{}lich wird ein Testaufbau im Wesentlichen auf die Timing Aspekte konzentriert vorgestellt und das Testergebnis wird ausgewertet.