First and foremost, I would like to thank my professor Prof. Dr. Oliver Kester. I have been so lucky to be his Ph.D. student. He gave me the chance to study at Goethe Universit"at, Frankfurt am Main and work in GSI for this interesting Ph.D. topic. It has been an honor to be his Ph.D. student. I appreciate all his contributions of time, ideas and funding to make my Ph.D. experience productive and stimulating. I was deeply influenced by his enthusiasm for his research and his selfless support for his students. I was thankful for his support, that I participated many international conferences, schools and workshops. This experience enriched my life and  
broadened my horizons.

I wish to express my sincere gratitude to Dr. David Ondreka and Dr. Dietrich Beck for their supervision, valuable guidance and helpful suggestions throughout my Ph.D. study. I have been greatly lucky to have so good supervisors, who cared much about my work and who answered my doubts patiently. They were like a lighthouse in the ocean, which guides me in the right direction. They are not only my  scientific supervisors, but also my mentors. They encouraged and motivated me during tough time of my Ph.D. Hence, I will keep a positive attitude and keep moving forward when I face with challenges, difficulties and temporary setbacks in the future.

I would like to acknowledge all colleagues in the timing group, \gls{CSCO} department, GSI, Mathias Kreider, Stefan Rauch, Marcus Zweig, Alexander Hahn and former employee Dr. Wesley Terpstra, who provided me much technical support. I would like to extend my appreciation to department leader Dr. Ralph B"ar, who gave much support for my Ph.D. topic. Thanks for their friendship and collaboration. I am especially grateful for the group member Cesar Prados, by whom I learned not only technical knowledge, but also how to work efficiently and how to become a good engineer. I would also like to thank Matthias Thieme, who provided me the devices for the test setup and Marko Stanislav Mandakovic for the discussion about the MPS.

I am also thankful for the good cooperation with Thibault Ferrand, who studies at Technische Universit"at Darmstadt and works in \gls{PBRF} department, GSI. Thanks for his valuable contribution of the development of the LLRF system for the B2B transfer system for FAIR. I would like to extend my sincerest thanks and appreciation to PBRF department leader Prof. Dr. Ing. Harald Klingbeil for his support. In addition, a special thanks is also extended to Dr. Dieter Lens and Stefan Sch"afer for their technical support. Thanks for Dr. Bernhard Zipfel to give me support about BuTiS.

I wish to express my sincere gratitude to \gls{SBES} department leader Dr. Markus Steck for his technical support of ESR and CRYRING, Dr. Blell Udo in \gls{PBHV} department for the technical support of kicker and Dr. Michael Block in \gls{SHE-P} department for the supply of two SRS function generators. 

I must express my gratitude to GSI, who provides the Facility for Antiproton and Ion Research project for my experiment. And also HGS-HiRe, who provided the scholarship which allowed me to undertake this research.


Lastly, I would like to thank my family for all their love and encouragement. My parents always support me to pursuing my dreams. Most of all, my loving husband Zigao Li is so appreciated, who always encourages me to realize my dreams. Thank you.

 \rightline{Jiaoni Bai}
 \rightline{Darmstadt, September 2016}



