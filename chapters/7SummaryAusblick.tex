For many large scale accelerator facilities, it is inevitable to transfer bunched beam from one ring accelerator to another to gain high energy, high intensity and high quality beam. Without the proper transfer, the beam will be subject to various disturbances and even beam loss. Hence, the proper bunch-to-bucket transfer between two accelerators is of great importance. 

FAIR, Facility for Antiproton and Ion Research, is a new international accelerator facility under construction at GSI Helmholtz center for Heavy Ion Research GmbH, aiming at providing high-energy beams of ions from antiproton to uranium with high intensities. The existing GSI accelerator includes the SIS18 and the ESR. The new FAIR accelerator complex with storage rings consists of the SIS100, the SIS300, the Collector Ring (CR), the Recycled Experimental Storage Ring (RESR), the New Experimental Storage Ring (NESR) and the Hign Energy Storage Ring (HESR). Although the existing GSI control system realizes the B2B transfer from the SIS18 to the
ESR and from the ESR back to the SIS18, it is not applicable for the new FAIR accelerator complex due to the FAIR existing infrastructures, (e.g. the FAIR control system and the LLRF system) and unique FAIR demands (e.g. MPS). Hence, the development of the FAIR B2B transfer system is imperative. The FAIR B2B transfer system focuses first of all on the transfer from the SIS18 to the SIS100, but it will be firstly tested for the transfer from the SIS18 to the ESR and further to the CRYRING.   

The FAIR B2B transfer system is introduced in this work at hand from the functional point of view. For the B2B transfer, there is a “B2B transfer master“, which is responsible for the data collection of two synchrotrons, the data calculation, the data redistribution and the B2B transfer status check. The data of the source and target synchrotron must be transferred to the “B2B transfer master“ via the deterministic WR network in the format of the timing frame. For FAIR use cases, the source synchrotron works as the ``B2B transfer master``. The phase difference between two rf systems of two synchrotrons is obtained with the help of a shared reference signal at two synchrotrons. The source synchrotron is responsible for the calculation of the start of the synchronization window, the phase correction and the requried phase shift (only for phase shift method). The extraction and injection kicker firing is based on the bucket indication signal marker within the synchronization window plus a specified delay. In addition, this work presents how all FAIR use cases apply the FAIR B2B transfer system and how precise the bunch-to-bucket transfer is achieved with the system. 

The FAIR B2B transfer system supports both the phase shift and frequency beating methods. It is more flexible. It supports several B2B transfers running at the same time, e.g. the B2B transfer from the SIS18 to the SIS100 and the B2B transfer from the ESR to the CRYRING. It is capable to transfer different species beam from one machine cycle to another without the operator’s configuration. It is capable to transfer the beam between two synchrotrons via a FRS, Pbar or Super FRS. It can achieve various complex bucket pattern. What is more, the FAIR B2B transfer system coordinates with the MPS system, which protects SIS100/SIS300 from unacceptable failure or situation. For most FAIR use cases, it achieves the B2B transfer with the bunch-to-bucket injection center mismatch less than $\pm1^\circ$ and within a upper bound B2B transfer time.

In addition, the beam dynamic of the $U^{28+}$ B2B transfer from SIS18 to SIS100 is simulated for two synchronization methods, the phase shift and frequency beating method. This work explains the timing constraints of the system, the calculation of the synchronization window and presents the usage of the WR network for the B2B transfer system. Further, the SIS18 extraction and SIS100 injection kickers are analyzed for the different triggering possibilities. 

This work presents a test setup for the system, achieving the phase collection of two synchrotrons locally, the phase transfer from the target synchrotron to the source synchrotron, the calculation of the synchronization window at the source synchrotron, the redistribution of the start of the synchronization window to the WR network, the production of the synchronization window at the source/target synchrotron. 

Although the B2B transfer system for FAIR is flexible and with high compatibility, there still exists several improvement. 

In order to reduce the synchronization time, the synchronization process could be started during the acceleration. The phase
difference between two Reference RF Signals of the source and target synchrotrons at the flattop could be predicted by comparison the phases of these two signals at any time during the acceleration. Once the phase difference at the flattop is predicted, the synchronization process can be carried out: 
\begin{itemize}
	\item Phase shift method

First, the radial loop must be turned off. At some time during the acceleration, the phases difference between the source and target synchrotrons are obtained with the help of the Synchronization Reference Signal, and the phase difference at the flattop is picked up from the look-up table. Then, a rf frequency modulation is superposed on the initial frequency pattern. The integration of the rf frequency modulation equals to the required phase difference. With this new frequency pattern, the phase difference at the flattop will be the required phase difference when the cavity rf frequency of the source and target synchrotrons reach the flattop. 
	\item Frequency beating method

The radial loop keeps on. At some time during the acceleration, the phases difference between the source and target synchrotrons are obtained. Then, a frequency detune is superposed on the initial frequency pattern. With this new frequency pattern, the synchronization window will be calculated. 
\end{itemize}