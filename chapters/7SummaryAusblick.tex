For many large scale accelerator facilities, it is inevitable to transfer bunched beam from one ring accelerator to another to gain higher energy or to accumulate beam for some research experiments. Without the proper transfer, the beam will be subject to various disturbances and even beam loss, e.g. dipole oscillation caused by the injection energy or phase error, quadrupole oscillation caused by the cavity voltage error. Hence, the proper bunch-to-bucket transfer between two acceletrators is of great importance. 

Facility for Antiproton and Ion Research (FAIR) aims at providing high-energy beam with high intensities. SIS100/300 of FAIR is under construction at GSI Helmholtz Centre for Heavy Ion Research GmbH at current stage. The B2B transfer has never been practiced between the existing machines, e.g. SIS18, ESR and CRYRING. The new developed Bunch-to-Bucket transfer system for FAIR in the dissertation is designed for all complex B2B transfer between FAIR accelerators. It is capable to transfer different species beam from one machine cycle to another. It is capable to parallel transfer beam through FAIR accelerators. It is also able to transfer the beam between two synchrotrons via FRS or Super FRS. It focuses first of all on the transfer from SIS18 to SIS100, but it will be firstly tested for the transfer from SIS18 to ESR and further to CRYRING.

The B2B transfer system for FAIR is introduced in the dissertation at hand from the functional point of view. The basic principles for B2B transfer are realized based on the existing FAIR technical basis (e.g. LLRF and FAIR control systems) and unique FAIR demands (e.g. Machine Protection System, MPS). The phase difference between two RF systems of two ring accelerators is obtained with the help of a shared reference signal at two ring accelerators. The source synchrotron works as the ``B2B transfer master`` for the rf phase collection, data (e.g. synchronization window, phase correction, phase shift and so on) calculation, synchronization window redistribution and B2B status check. In addition, the dissertation presents how FAIR accelerators apply the B2B transfer system and how precise the bunch-to-bucket transfer is achieved with the system. The rules for the application of the system is explained, which is determined by the relation between the circumference ratio/energy ratio and the cavity harmonic number of two synchrotron.

In addition, the beam dynamic of the $U^{28+}$ B2B transfer from SIS18 to SIS100 is simulated for two synchronization methods, the phase shift and frequency beating method. The disseration explains the timing constraints of the system, the calculation of the synchronization window and presents the usage of the WR network for the B2B transfer system. Further, the SIS18 extraction and SIS100 injection kickers are analyzed for the different triggering possibilities. 

The dissertation presents a test setup for the system, achieving the phase collection of two synchrotrons locally, phase transfer from the target to source synchrotron, synchronization window calculation at the source synchrotron, synchronization window redistribution to the WR network, synchronization window reproduced at the source/target synchrotron. 

Although the B2B transfer system for FAIR is flexible and with high compatibility, there still exists several improvement. 

In order to reduce the synchronization time, the synchronization process could be started during the acceleration. The phase
difference between two Reference RF Signals of the source and target synchrotrons at the flattop could be predicted by comparison the phases of these two signals at any time during the acceleration. Once the phase difference at the flattop is predicted, the synchronization process can be carried out. 
\begin{itemize}
	\item Phase shift method

First, the radial loop must be turned off. At some time during the acceleration, the phases difference between the source and target synchrotrons are obtained with the help of the Synchronization Reference Signal, and the phase difference at the flattop is picked up from the look-up table. Then, a rf frequency modulation is superposed on the initial frequency pattern. The integration of the rf frequency modulation equals to the required phase difference. With this new frequency pattern, the phase difference at the flattop will be the required phase difference when the cavity rf frequency of the source and target synchrotrons reach the flattop. 
	\item Frequency beating method

The radial loop keeps on. At some time during the acceleration, the phases difference between the source and target synchrotrons are obtained. Then, a frequency detune is superposed on the initial frequency pattern. With this new frequency pattern, the synchronization window will be calculated. 
\end{itemize}