The FAIR project is aiming at providing high-energy beams of ions from proton to uranium, antiproton and rare isotope with high intensities. The existing facility at GSI includes the SIS18 and the ESR. The new FAIR accelerator complex will consist of the synchrotron SIS100 and SIS300, the collector ring CR and the storage ring HESR. Bunches are required to be transferred into buckets among GSI and FAIR ring accelerators for different purposes. Without the proper transfer, the beam will be subject to various disturbances and even beam loss. Hence, the proper bunch-to-bucket transfer between two rings is of great importance for FAIR. Although an implementation of the B2B transfer from the SIS18 to the ESR exists, this solution is not applicable for the new FAIR accelerator complex because of its limitation. It is realized based on the GSI control system, which will be replaced by the FAIR control system in the future. Besides, it doesn't support the phase shift method. Hence, a new FAIR B2B transfer system needs to be developed based on the FAIR technical basis, the FAIR control system and the low level rf system.

%The conceptual realization of the FAIR B2B transfer system is introduced in the dissertation. For the B2B transfer, there is a “B2B transfer master“, which is responsible for the data collection of two ring accelerators, the data calculation, the data redistribution and the B2B transfer status check. Synchronization reference signals are synchronously distributed around the FAIR campus. In the dissertation, the source ring works as the “B2B transfer master“. The phase deviation between the rf system and the synchronization reference signal is measured and extrapolated at both rings. The extrapolated phase of the target ring must be transferred to the “B2B transfer master“ via the deterministic WR network in the format of the timing frame. The phase difference between the two rf systems of the rings is obtained by the subtraction of two extrapolated phases. The source ring is responsible for the calculation of the start of the synchronization window, the phase correction and the required phase shift (only for phase shift method). The synchronization window is used to select the first occurrence of the bucket indication signal marker. The extraction and injection kickers are triggered based on the selected bucket indication signal marker plus a specified delay.

% Additionally, all FAIR use cases are analyzed and the precision of the B2B transfer is calculated for all use cases. 

The conceptual realization of the FAIR B2B transfer system is introduced in the dissertation, which takes all important factors into account. The FAIR B2B transfer system supports both the phase shift and frequency beating methods. It is flexible to support the transfer between two rings with an arbitrary circumference ratio and several B2B transfers running at the same time, e.g. the B2B transfer from the SIS18 to the SIS100 and at the same time the B2B transfer from the ESR to the CRYRING. It is capable to transfer beam of different ion species from one machine cycle to another. It has the ability to transfer the beam between two rings via the FRS, the pbar target and the Super FRS. It can achieve various complex bucket pattern. In addition, the FAIR B2B transfer system coordinates with the MPS system, which protects accelerators from unacceptable failure or situation. For all primary beam transfers of FAIR use cases, it achieves the B2B transfer with the bunch-to-bucket injection center mismatch less than $\pm1^\circ$ and within the required B2B transfer time \SI{10}{\ms}, because the circumference ratio between two rings is an integer or close to an integer. However, the system doesn't work properly for the FAIR use cases that the secondary beams are generated by the pbar target, the FRS or the Super FRS, because the energy ratio between the primary and secondary beams is arbitrary. For the rare isotope beam transfer from the SIS100 to the CR via the Super FRS with the \SI{1.5}{Gev/u} primary beam energy and the \SI{740}{Mev/u} secondary beam energy, the bunch-to-bucket injection center mismatch is only $\pm2.1^\circ$ by coincidence. For the antiproton B2B transfer from the SIS100 to the CR via the pbar target and the rare isotope beam transfer form the SIS18 to the ESR via the FRS, the bunch-to-bucket injection center mismatch is as large as $\pm40^\circ$. For these two FAIR use cases, the FAIR B2B transfer system must work together with specific beam accumulation methods, e.g. the barrier bucket or the unstable point accumulation.
 

In addition, according to the beam dynamic analysis of the $U^{28+}$ B2B transfer from the SIS18 to the SIS100, the sinusoidal rf frequency modulation is better to keep the beam stability and to guarantee the adiabaticity for the phase shift method compared with the parabolic modulation with the same duration. The sinusoidal rf frequency modulation for the SIS18 \SI{200}{Mev/u} $U^{28+}$ needs \SI{7}{\ms} and the sinusoidal rf frequency modulation for the SIS18 \SI{4}{Gev} $H^{+}$ needs approximately \SI{20}{\ms} for the phase shift of $\pi$. The firmware of the FAIR B2B timing system running on the soft CPU, LM32, meets the requirement of the timing constraints and the requirement of the accuracy of the start of the synchronization window is approximately \SI{500}{\ns}. Besides, a specified VLAN on the WR network is used for the B2B transfer and the tolerable layers of WR switch for the B2B transfer depends not only on the upper bound transfer latency, but also the tolerable frame error rate of the B2B transfer system. If no forward error correction mechanism is used for the B2B network, the layers of WR switch is mainly decided by the tolerable frame error rate. If the FAIR B2B transfer system is tolerant of one lost frame per month. The maximum 18 layers of WR switch are used between the B2B related FECs and DM and maximum 4 layers of WR switch are used between the B2B related FECs. If specific forward error correction mechanisms are used for the B2B network, the layers of WR switch is mainly decided by the tolerable transfer latency. In this case, the tolerable layers of WR is 67 between the B2B related FECs and DM and the tolerable layer of WR switch is 13 between the B2B related FECs. Further, the SIS18 extraction kicker magnets in the $2^{nd}$ crate can be triggered a fixed delay after the trigger of the SIS18 extraction kicker magnets in the $1^{st}$ crate for ion beams over the whole range of stable isotopes, when the bunch gap is $25\%$ of the cavity rf period. The SIS100 injection kicker magnets can be fired instantaneously for all ion beams, when the bunch gap is $35\%$ of the cavity rf period.

%This work presents a test setup for the system, achieving the phase collection of two ring accelerators locally, the phase transfer from the target synchrotron to the source synchrotron, the calculation of the synchronization window at the source synchrotron, the redistribution of the start of the synchronization window to the WR network and the reproduction of the synchronization window at the source/target synchrotron. 

The dissertation at hand presents the important investigations for the FAIR B2B transfer system from the beam dynamics, timing and kicker trigger perspectives. However, there are still some investigations which are beyond the scope of my dissertation.
\begin{itemize}
	\item The synchronization between the magnetic horn after the pbar target and the antiproton beam to the \SI{}{\us} order of magnitude.

	\item  The synchronization between the bunch compressor and the beam extraction.

	\item The solution for the FAIR use cases of the antiproton B2B transfer from the SIS100 to the CR via the pbar target and the rare isotope beam transfer form the SIS18 to the ESR via the FRS. e.g. the barrier bucket, the unstable point beam accumulation.

\end{itemize}


The FAIR B2B transfer system presented in the dissertation is applicable for all FAIR use cases. However, there is still potential for improvement. For the phase shift method, the rf frequency modulation must be slow enough for the beam to follow. In order to transfer bunches into buckets as soon as possible, the phase shift can be started during the acceleration ramp. At a certain time point during the acceleration, the phases difference between the two rf systems of the source and target rings is obtained with the help of the synchronization reference signal. There is a look-up table, which gives the phase difference at the rf flattop according to the phase difference obtained at the certain time point. Hence, the phase difference at the rf flattop can be picked up from the look-up table. Then, a rf frequency modulation is superposed on the initial frequency pattern. The integration of the rf frequency modulation equals to the required phase difference. With this new frequency pattern, the phase difference will be the required phase difference when the cavity rf frequency of the source and target rings reach the flattop. 

