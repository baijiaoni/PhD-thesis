This dissertation contributes to the conceptual development, the systematic investigation, the timing system realization of the FAIR Bunch-to-Bucket (B2B) transfer system and its application to FAIR accelerators. 

FAIR, the Facility for Antiproton and Ion Research, is a new international particle accelerator facility under construction at GSI Helmholtz center for Heavy Ion Research GmbH. It is aiming at providing high-energy beams of ions from proton to uranium, antiproton and rare isotope with high intensities. The FAIR accelerator complex in its full version consists of many synchrotrons with different functionality. The synchrotrons are connected by transfer beamlines. Hence, the FAIR B2B transfer system plays an important role, achieving various complex bunch-to-bucket transfer between the FAIR accelerators in the future. It focuses first of all on the transfer from the SIS18 to the SIS100, but it will be firstly tested at GSI on the transfer from the SIS18 to the ESR and from the ESR to the CRYRING. The system is developed based on the existing technical basis, the low level rf system and the FAIR control system. It coordinates with the Machine Protection System, which protects synchrotrons from fatal beam losses and considerable damage. Besides, it indicates the beam status and the actual beam injection time for the Beam Instrumentation. 
 
This dissertation first of all presents the basic idea, the basic procedure and the timing aspect realization of the FAIR B2B transfer system. Secondly the systematic investigation of the system is done. Because the system focuses first of all on the transfer from the SIS18 to the SIS100, the beam dynamic of the B2B transfer from the SIS18 to the SIS100 is simulated for both the phase shift and the frequency beating methods. In addition, the SIS18 extraction and SIS100 injection kickers are analyzed for different triggering strategies. This dissertation also explains the time constraints of the system, the accuracy of the start of the synchronization window and characterization of the network for the FAIR B2B transfer system. Finally, a test setup focusing mainly on the timing aspects is introduced and the test result is presented. 

In order to find the correct trigger time for the kickers, the FAIR B2B transfer system is composed of two synchronization processes, a coarse synchronization and a fine synchronization. The coarse synchronization gives a coarse time frame, within which bunches are transferred into buckets with a bunch-to-bucket center mismatch smaller than a upper bound. This time frame is called the ``synchronization window``. With the synchronization window, the extraction
and injection kicker magnets must be fired at the correct time in order to transfer bunches into correct empty buckets. The process of the kicker firing at the correct time is the ``fine synchronization``. 

The coarse synchronization is based on the measurement of the radio frequency (rf) phase difference between the two rf systems of two ring accelerators, which is obtained by means of a campus-wide distributed reference signal with picosecond precision. When the circumference ratio between the source and target synchrotrons is an integer, the phase difference between the two rf systems is constant. In order to get the correct phase difference, an azimuthal position of bunches in the source synchrotron or buckets in the target synchrotron must be adjusted. This is called “phase shift method“. After the phase shift, the phase difference between the two rf systems is correct and the synchronization window is infinite theoretically. When the circumference ratio is not an integer, the phase difference between the two rf systems varies periodically. Within one period, there must be one time point when the phase difference is the target one. Before and after this time point, there exists the mismatch between bunches and buckets. This is called ”frequency beating method”. The fine synchronization is achieved based on a bucket indication signal for the first bucket plus a fixed delay. 

For the FAIR B2B transfer system, there exists a “B2B transfer master“. It is responsible for the following functions. 
\begin{itemize}

	\item 	The data collection (e.g. the rf phase). 
   \item 	The data calculation (e.g. the start of the synchronization window, the required phase shift for the goal phase difference between the two rf systems, the phase correction for the bucket indication signal and etc). 
   \item 	The data redistribution (e.g. the start of the synchronization window).
	\item The B2B transfer status check.
\end{itemize}


