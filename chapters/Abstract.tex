The Facility for Antiproton and Ion Research (FAIR) is a new international particle accelerator facility under construction at GSI Helmholtz center for Heavy Ion Research GmbH. It is aiming at providing high-energy beams of ions of all elements from hydrogen to uranium with high intensities, as well as beams of rare isotopes and beams of antiprotons. The FAIR accelerators will be supplied with ion beams by the GSI accelerator facility, which comprises the injectors for the FAIR accelerators. The injection chain consists of the linear accelerator UNILAC and the heavy ion synchrotron SIS18. In addition, the GSI accelerator facility comprises the experimental storage ring (ESR) and the CRYRING, which complement the planned accelerators of FAIR. The FAIR facility in its start version will consist of three circular accelerators, which add to the three rings at GSI. The driver accelerator of FAIR is the fast ramping, superconducting heavy ion synchrotron SIS100, that allows the acceleration of the most intense beams of stable elements. The primary beams are used to produce secondary beams, which are delivered to the collector ring (CR) via high energy separators. The CR will accumulate the secondary beams and improve their quality by stochastic cooling. The storage ring HESR will host a large fraction of the experiment platforms with a variety of different experiments. These circular accelerators of GSI and FAIR have different ratios in their circumference. For example, the circumference ratio between the SIS100 and the SIS18 is an integer and between the SIS18 and the ESR is close to an integer and between the CR and the HESR is far away from an integer. The ring accelerators are connected via a complicated system of beam transfer lines, targets for the secondary particle production and the high energy separators mentioned above. For FAIR, not only the primary beams are required to be transferred from one ring to another, but also the secondary beams, e.g. the antiproton or rare isotope beams produced by the pbar target, the Fragment Separator (FRS) or the Super FRS. An important topic for this system of accelerators is the proper transfer of beam between the different machines. Bunches of one ring must be transferred into buckets of another ring within an upper bound time constraint (e.g. \SI{10}{\ms} for most FAIR use cases) and with an acceptable bunch-to-bucket injection center mismatch (e.g. $\pm1^\circ$ for most FAIR use cases). Hence, a flexible FAIR Bunch-to-Bucket (B2B) transfer system is required to realize the different complex bunch-to-bucket transfers between the FAIR rings in the future. In the focus of the system development and of this thesis is the transfer from the SIS18 to the SIS100, which can be tested at GSI on the transfer from the SIS18 to the ESR and from the ESR to the CRYRING. The system is based on the existing technical basis at GSI, the low-level radio frequency (LLRF) system and the FAIR control system. It coordinates with the Machine Protection System, which protects SIS100 and subsequent accelerators and experiments from damage caused by high intensity primary beams in case of malfunctioning. Besides, it indicates the beam status and the actual beam injection time for the beam instrumentation and diagnostics. 

In this thesis the basic idea, the basic procedure and the conceptual realization of the bunch-to-bucket transfer system are described in detail and the mathematical evaluation of the required timing parameters is presented. In order to trigger the extraction and injection kickers correctly, the FAIR B2B transfer system is composed of two synchronization processes, a coarse synchronization and a fine synchronization. The coarse synchronization gives a coarse time frame, within which bunches are transferred into buckets with a bunch-to-bucket center mismatch smaller than a well defined bound. This time frame is called the ``synchronization window``. With the synchronization window, the extraction and injection kickers are triggered at the correct time in order to transfer bunches into correct empty buckets. The process of the kicker trigger at the correct time is the ``fine synchronization``. The fine synchronization is achieved based on a bucket indication signal plus a fixed delay. The bucket indication signal is derived from the rf revolution frequency signal and always indicates the first bucket. A fixed delay is used to indicate the correct buckets to be filled.

The coarse synchronization is based on the phase difference between the two rf systems of two rings, which is obtained by the phase deviation measurement between the rf system and a campus-wide distributed synchronization reference signal at both rings. When the circumference ratio between two rings is an integer, the phase difference between the two rf systems is constant. In order to get the correct phase difference, the phase of either (or both) rf systems must be shifted by means of an rf frequency modulation. This is called ``phase shift method``. After the rf frequency modulation, the phase difference between the two rf systems is correct and the synchronization window is infinitely long theoretically. When the circumference ratio between two rings is not an integer, the phase difference between the two rf systems varies periodically. The synchronization window brings a symmetric time frame with respect to the time, when the phase difference between two rf systems is closest to the required phase difference. This is called ``frequency beating method``. The frequency beating method is also applicable when the circumference ratio between two rings is an integer. In this case, the rf frequency of either (or both) rf systems is detuned at the end of the acceleration ramp, so that two rf systems are beating. For the FAIR project, the frequency beating method is preferable, because it is applicable for all beam transfer scenarios. In addition, it reduces the synchronization time, because the rf frequency detune is executed during the rf acceleration ramp. For the phase shift method, the rf frequency modulation must be executed slowly enough at the rf flattop for beams to follow according to the calculated limits. The phase shift method therefore needs much more time to be executed. However, there are also some advantages of the phase shift method. The synchronization window is relatively long and the bunch-to-bucket injection center mismatch is approximately $0^\circ$. Besides, the duration of the rf frequency modulation is known in advance and the time point for the transfer is predictable. The phase of the rf system can jump to a desired value, when there is no bunch at the ring.  

%For the FAIR B2B transfer system, there is a “B2B transfer master“, which is responsible for the data collection (e.g. the phase of rf system), the data calculation (e.g. the start of the synchronization window, the required shift phase), the data redistribution (e.g. the start of the synchronization window) and the B2B transfer status check.

 
In this thesis, a systematic investigation has been done from the beam dynamics, timing requirement of the transfer and kicker trigger perspectives. The timing perspective includes the accuracy of the start of the synchronization window, the characterization of the White Rabbit network for the B2B transfer, the flow chart and the time constraints of the system. A test setup of the timing system required for the transfer using the frequency beating method has been developed and the firmware running on the soft CPU has been tested, which meets the functional requirement and the time constraints. Finally, all FAIR use cases with the frequency beating method have been discussed.

The dissertation plays a significant important role for the realization of the FAIR B2B transfer system and the further practical application of the system to all FAIR use cases.  



