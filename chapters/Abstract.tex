The Facility for Antiproton and Ion Research (FAIR) is a new international particle accelerator facility under construction at GSI Helmholtz center for Heavy Ion Research GmbH. It is aiming at providing high-energy beams of ions from proton to uranium with high intensities, as well as beams of rare isotopes and beams of antiprotons. The existing facility at GSI includes the SIS18 and the ESR. The FAIR accelerator complex in its full version consists of many ring accelerators with different functionality. FAIR has synchrotrons (e.g. the SIS100), storage rings (e.g. the HESR) and collect rings (e.g. the CR). These rings have an arbitrary circumference ratio. For example, the circumference ratio between the SIS100 and the SIS100 is an integer and between the SIS18 and the ESR is close to an integer and between the CR and the HESR is far away from an integer. The ring accelerators are connected by transfer beamlines. For FAIR, not only the primary beams are required to be transferred from one ring to another, but also the secondary beams. e.g. the antiproton and rare isotope beams produced by the pbar target, the FRS or the Super FRS. Besides, bunches of one ring must be transferred into buckets of another ring within an upper bound time constraint (e.g. \SI{10}{\ms} for most FAIR use cases) and with an acceptable bunch-to-bucket injection center mismatch (e.g. $\pm1^\circ$ for most FAIR use cases). Hence, a flexible and desired FAIR Bunch-to-Bucket (B2B) transfer system is required to realize various complex bunch-to-bucket transfer between the FAIR rings in the future. It focuses first of all on the transfer from the SIS18 to the SIS100, but it will be firstly tested at GSI on the transfer from the SIS18 to the ESR and from the ESR to the CRYRING. The system is developed based on the existing technical basis, the low level radio frequency (rf) system and the FAIR control system. It coordinates with the Machine Protection System, which protects SIS100 and subsequent accelerators or experiments from damage. Besides, it indicates the beam status and the actual beam injection time for the beam instrumentation and diagnostics. 

In order to trigger the extraction and injection kickers correctly, the FAIR B2B transfer system is composed of two synchronization processes, a coarse synchronization and a fine synchronization. The coarse synchronization gives a coarse time frame, within which bunches are transferred into buckets with a bunch-to-bucket center mismatch smaller than an upper bound. This time frame is called the ``synchronization window``. With the synchronization window, the extraction and injection kickers are triggered at the correct time in order to transfer bunches into correct empty buckets. The process of the kicker trigger at the correct time is the ``fine synchronization``. The fine synchronization is achieved based on a bucket indication signal plus a fixed delay. The bucket indication signal is derived from the rf revolution frequency signal and always indicates the first bucket. A fixed delay is used to indicate the correct buckets to be filled.

The coarse synchronization is based on the phase difference between the two rf systems of two rings, which is obtained by the phase deviation measurement between the rf system and a campus-wide distributed synchronization reference signal at both rings. When the circumference ratio between two rings is an integer, the phase difference between the two rf systems is constant. In order to get the correct phase difference, the phase of either (or both) rf systems must be shifted backward or forward by means of the rf frequency modulation. This is called “phase shift method“. After the rf frequency modulation, the phase difference between the two rf systems is correct and the synchronization window is infinitely long theoretically. When the circumference ratio between two rings is not an integer, the phase difference between the two rf systems varies periodically. The synchronization window brings a symmetric
time frame with respect to the time, when the phase difference between two rf systems is closest to the required phase difference. This is called ”frequency beating method”. The frequency beating method is also applicable when the circumference ratio between two rings is an integer. In this case, the rf frequency of either (or both) rf systems is detuned at the end of the acceleration ramp, so that two rf systems are beating. For the FAIR project, the frequency beating method is preferable, because it is applicable for the primary beam transfer between two rings with an arbitrary circumference ratio, as well as the secondary beam transfer. In addition, the rf frequency detune is not executed during the bunch-to-bucket transfer process, which is required to change slowly enough for the beam to follow. However, there are also some advantages of the phase shift method. The synchronization window is relatively long and the bunch-to-bucket injection center mismatch is approximately $0^\circ$. Besides, the duration of the rf frequency modulation is known in advance and the transfer time is predictable. The phase of the rf system can jump to a desired value, when there is no bunch at the ring.  

For the FAIR B2B transfer system, there is a “B2B transfer master“, which is responsible for the data collection (e.g. the phase of rf system), the data calculation (e.g. the start of the synchronization window, the required shift phase), the data redistribution (e.g. the start of the synchronization window) and the B2B transfer status check.

 
This dissertation first of all presents the basic idea, the basic procedure and the conceptual realization of the FAIR B2B transfer system. Secondly, the systematic investigation is done from the beam dynamics, timing and kicker trigger perspectives. The timing perspective includes the accuracy of the start of the synchronization window, the characterization of the White Rabbit network for the B2B transfer, the firmware and the time constraints of the system. Then a test setup of the timing aspects is presented, which proves that the firmware running on the soft CPU meets the functional requirement and the time constraints. Finally, all FAIR use cases with the frequency beating method are discussed.

The dissertation plays a significant important role for the realization of the FAIR B2B transfer system of the released version and the further practical application of the system to all FAIR use cases.  



