This dissertation contributes to the conceptual development, systematic investigation and timing system realization of the Bunch-to-Bucket (B2B) transfer system for FAIR, Facility for Antiproton and Ion Research at GSI Helmholtzzentrum f$\ddot{u}$r Schwerionenforschung GmbH. 

The B2B transfer system for FAIR plays an important role for FAIR project, which will achieve various complex bunch to bucket transfer for FIAR accelerators in the future. It focuses first of all on the transfer from SIS18 to SIS100, but it will be firstly tested for the transfer from SIS18 to ESR and further to CRYRING. The system is developed based on the FAIR existing infrastructures, LLRF and FAIR control systems. It coordinates with the Machine Protection System (MPS), which protects SIS100/SIS300 from unacceptable failure or situation and indicates beam for the Beam Instrumentation (BI). 
 
The B2B transfer system obtains the rf phase difference between two synchrotrons by means of a campus distributed reference signals with picosecond precision, which is synchronized with Bunchphase Timing System (BuTiS). The source synchrotron works as a ``B2B transfer master`` for the rf phase collection, data (e.g. synchronization window indicating the coarse time frame for the transfer, phase shift for the phase match between two rf systems, phase correction for the bucket label and so on) calculation, synchronization window redistribution and B2B transfer status check. The synchronization window is a coarse time frame for the transfer and the bucket label signal is used to indicate the fine transfer. All application cases of FARI project are analyzed with this system and all transfers achieve the bunch-to-bucket center mismatch smaller than the tolerance.

Because the system focuses first of all on the transfer from SIS18 to SIS100, the beam dynamic of the $U^{28+}$ B2B transfer from SIS18 to SIS100 is simulated for two synchronization methods, the phase shift and frequency beating method. In addition, the SIS18 extraction and SIS100 injection kickers are analyzed for different triggering strategies. The disseration also explains the timing constraints of the system, the calculation of the synchronization window and presents the usage of the WR network for the B2B transfer system. 

The test setup of the timing system of the B2B transfer system for FAIR is also presented in this dissertation. 
