This dissertation contributes to the conceptual development, systematic investigation and timing system realization of the Bunch-to-Bucket (B2B) transfer system for FAIR, Facility for Antiproton and Ion Research at GSI Helmholtzzentrum f"ur Schwerionenforschung GmbH. 

The B2B transfer system for FAIR plays an important role for the FAIR project, which will achieve various complex bunch to bucket transfer for FAIR accelerators in the future. It focuses first of all on the transfer from SIS18 to SIS100, but it will be firstly tested for the transfer from SIS18 to ESR and from ESR to CRYRING. The system is developed based on the FAIR existing infrastructures, Low Level Radio Frequency system (LLRF) and FAIR control systems. It coordinates with the Machine Protection System (MPS), which protects SIS100/SIS300 from fatal errors and considerable damage and indicates beam status for Beam Instrumentation (BI). 
 
The B2B transfer system obtains the radio frequency (rf)-phase difference between two synchrotrons by means of a campus wide distributed reference signal with picosecond precision, which is provided by the Bunchphase Timing System (BuTiS). The part of the B2B electronic locates in the the source synchrotron supply room and serves as a kind of ``B2B transfer master``. The most important tasks of B2B transfer master are:
\begin{itemize}

	\item 	The data collection (e.g. the rf phase collection). 

   \item 	The data processing (e.g. the calculation of the synchronization window, the phase shift for the phase match between two rf systems, the phase correction for the bucket label, the B2B transfer status check and etc.). 

   \item 	The data redistribution (e.g. the synchronization window).
\end{itemize}
The synchronization window is a coarse time frame for the transfer (coarse synchronization) and the bucket label signal is used to indicate a certain bucket to be injected within the window, which is called the ``fine synchronization``. This system is applied to all FAIR B2B transfer cases and all transfers have to achieve the bunch-to-bucket injection center mismatch within the tolerance limits.

Because the system focuses first of all on the transfer from SIS18 to SIS100, the beam dynamic of the B2B transfer from SIS18 to SIS100 is simulated for two synchronization methods, the phase shift and the frequency beating method. In addition, the SIS18 extraction and SIS100 injection kickers are analyzed for different triggering strategies. This dissertation also explains the timing constraints of the system, the calculation of the synchronization window and presents the usage of the WR network for the B2B transfer system. 

A test setup of the timing system of the B2B transfer system for FAIR is also presented in this dissertation. 
