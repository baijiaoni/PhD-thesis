!!!!!!!!!!!!!!!!please stop here. This chapter needs to be reworked, including the corresponding abbreviation.

This chapter concentrates on the realization and systematic investigation of the \gls{B2B} transfer system. In Sec. 6.1, both the phase shift and frequency beating synchronization methods are analyzed from the beam dynamic perspective. The WR network is investigated for the B2B transfer and the calculation of the synchronization window are presented in Sec. 6.2. The B2B transfer system for FAIR focuses first of all on the transfer from the SIS18 to the SIS100, so the trigger possibility of the SIS18 extraction and SIS100 injection kicker are systematically investigated in Sec. 6.3. Besides, the test setup from the timing aspect is introduced and the test result is analyzed in Sec. 6.4. 

\section{Beam Dynamic Analysis of two Synchronization Methods for $U^{28+}$ B2B Transfer from SIS18 to SIS100}
This section analyzes the phase shift and frequency beating methods from the beam-dynamics perspective for the synchronization of the SIS18 with the SIS100. 
%In this chapter, the circumference of SIS18 and SIS100 are denoted by $C^{SIS18}$ and $C^{SIS100}$, the revolution frequency by $f_{h=1}^{SIS18}$ and $f_{h=1}^{SIS100}$ and the rf frequency by $f_{h=2}^{SIS18}$ and $f_{h=10}^{SIS100}$. Since SIS18 and SIS100 harmonic number are 2 and 10, the relationship between the revolution and rf frequencies are $f_{h=2}^{SIS18}=2f_{h=1}^{SIS18}$ and $f_{h=10}^{SIS100}=10f_{h=1}^{SIS100}$. Since $C^{SIS100}$ is five times as long as $C^{SIS18}$, we could get the relation  $f_{h=1}^{SIS18}$=5$f_{h=1}^{SIS100}$ and $f_{h=10}^{SIS100}$=$f_{h=2}^{SIS18}$.
%%%%%%%%%%%%%%%%%%%%%%%%%%%%%%%%%%%%%%%%%%%%%%%%%%%%%%%%%%%%%%%%%%%%%%%%%%%%%%%%%%
\subsection{Frequency Modulation for Phase Shift}
The obtained phase shift $\Delta \phi$ is determined by the rf frequency modulation $\Delta f_{rf}$ and the duration of the frequency modulation $T$. 
\begin{equation}
\Delta \phi= 2\pi \int_{t_0}^{t_0+T} \Delta f_{rf}(t)dt \label{phase_integration}
\end{equation}
In order to make the rf frequency modulation effective, the beam feedback loops on the rf system are switched off before the B2B starts. Sec. ~\ref{sec:requirement_phase_shift} shows that there are several criterion for the rf frequency modulation for the longitudinal emittance to be preserved.
\begin{itemize}
\item[-]
There exists a maximum rf frequency offset. 
\item[-]
$\frac{d\Delta f_{\mathit{rf}}}{dt}$ must be continuous and small enough. 
\item[-]
$\frac{d^2\Delta f_{\mathit{rf}}}{dt^2}$ must be small enough. 
\end{itemize}

According to these criterion, some rf frequency modulations are obviously ruled out of consideration. e.g. a trapezoid modulation and a triangular modulation, whose first derivative are not continuous. The following three examples of rf frequency modulation are analysed, which comply with the above mentioned criterion. The case (1) is a sinusoidal modulation, the case (2) is a parabolic modulation, which consists of three parabolas and two lines between every two parabolas, and the case (3) is also a parabolic modulation, including of three parabolas. The phase shift is assumed to be achieved within \SI{7}{ms}. Three rf frequency modulation cases are shown in Fig.~\ref{4case}. All three cases give the same phase shift, $\Delta \phi=\pi$, which is proved by substituting each form of $\Delta f_{rf}(t)$ into eq.~\ref{phase_integration} and performing integration. 

%Case (1)
%\begin{eqnarray}\label{case1}
%\Delta f(t)=
%\begin{cases}
%50(t-t_1), &t_1< t\le t_1+2ms\cr
%100, &t_1+2ms < t \le t_1+5ms \cr
%-50(t-t_1) + 7\cdot 50, &t_1+5ms < t\le t_1+7ms
%\end{cases}
%\end{eqnarray}
%
%Case (2)
%\begin{eqnarray}\label{case2}
%\Delta f(t)=
%\begin{cases}
%\frac {500}{3.5 \cdot 3.5}(t-t_1), &t_1< t\le t_1+3.5ms\cr
%-\frac {500}{3.5 \cdot 3.5}(t-t_1) +7
%\cdot \frac {500}{3.5 \cdot 3.5}, &t_1+3.5ms < t \le t_1+7ms 
%\end{cases}
%\end{eqnarray}
%
%Case (3)
%\begin{eqnarray}\label{case3}
%\Delta f(t)=
%\frac {1000}{7 \cdot 2} (1-cos(\frac{2\pi}{7}\cdot (t-t_1)), &t_1 < t\le t_1+7ms
%\end{eqnarray}
%
%Case (4)
%\begin{eqnarray}\label{case4}
%\Delta f(t)=
%\begin{cases}
%30(t-t_1)^2, &t_1< t\le t_1+1ms\cr
%30+ 60((t-t_1)-1), &t_1+1ms< t\le t_1+2.5ms\cr
%30(5-((t-t_1)-3.5)^2), &t_1+2.5ms< t\le t_1+4.5ms\cr
%
%30+60(6-(t-t_1)), &t_1+4.5ms< t\le t_1+6ms\cr
%30(7-(t-t_1))^2, &t_1+6ms< t\le t_1+7ms
%\end{cases}
%\end{eqnarray}

%Case (1) 
%\begin{eqnarray}\Delta f_{rf}(t)=
%\begin{cases}
%50Hz/ms \cdot (t-t_0) &t_0+0<t\le t_0+2ms\cr  100Hz &t_0+2<t\le t_0+5ms \cr 100Hz-50Hz/ms \cdot (t-t_0) &t_0+5ms<t\le t_0+7ms\cr 
%\end{cases}
%\end{eqnarray}
%
%Case (2) 
%\begin{eqnarray}\Delta f_{rf}(t)=
%\begin{cases}
%\frac{10^3}{7\cdot 3.5}Hz/ms \cdot (t-t_0) &t_0+0<t\le t_0+3.5ms\cr  \frac{10^3}{7}Hz-{\frac{10^3}{7\cdot 3.5}Hz/ms}\cdot {(t-t_0-3.5ms)} &t_0+3.5ms<t\le t_0+7ms \cr 
%\end{cases}
%\end{eqnarray}
%
%Case (1) 
%\begin{eqnarray}\Delta f_{rf}(t)=
%\frac{10^3}{14}Hz \cdot (1-cos(\frac{2\pi}{7} rad/ms\cdot (t-t_0))) &t_0+0<t\le t_0+7ms\cr  
%\end{eqnarray}
%
%Case (2) 
%\begin{eqnarray}\Delta f_{rf}(t)= \frac{20}{21}\cdot
%\begin{cases}
%30Hz/ms^2 \cdot (t-t_0)^2 &t_0+0<t\le t_0+1ms\cr  
%30Hz + 60Hz/ms\cdot (t-t_0 -1ms) &t_0+1ms<t\le t_0+2.5ms\cr 
%30Hz/ms^2 \cdot [5ms^2-(t-t_0-3.5ms)^2] &t_0+2.5ms<t\le t_0+4.5ms\cr  
%30Hz + 60Hz/ms\cdot [6ms-(t-t_0)] &t_0+4.5ms<t\le t_0+6ms\cr  
%30Hz/ms^2 \cdot [7ms^2-(t-t_0)]^2 &t_0+6ms<t\le t_0+7ms\cr  
%\end{cases}
%\end{eqnarray}
\begin{figure}[H]
   \centering   
   \includegraphics*[width=160mm]{4case.png}
   \caption{Examples of rf frequency modulation.}
   \label{4case}
\end{figure}

Case (1) 
\begin{eqnarray}\Delta f_{rf}(t)=
\frac{1}{2T}  [1-cos(\frac{2\pi}{T}(t-t_0))] &t_0+0<t\le t_0+T\cr  
\end{eqnarray}

Case (2) 
\begin{eqnarray}\Delta f_{rf}(t)= 
\begin{cases}
\frac{9}{T^3}(t-t_0)^2 &t_0+0<t\le t_0+\frac{T}{6}\cr  
\frac{1}{4T} +\frac{3}{T^2}(t-t_0 -\frac{T}{6}) &t_0+\frac{T}{6}<t\le t_0+\frac{2T}{6}\cr 
\frac{1}{T}-\frac{9}{T^3}(t-t_0-\frac{T}{2})^2 &t_0+\frac{2T}{6}<t\le t_0+\frac{4T}{6}\cr  
\frac{3}{4T} -\frac{3}{T^2}(t-t_0 -\frac{4T}{6})  &t_0+\frac{4T}{6}<t\le t_0+\frac{5T}{6}\cr  
\frac{9}{T^3}(t-t_0-T)^2 &t_0+\frac{5T}{6}<t\le t_0+T\cr  
\end{cases}
\end{eqnarray}

Case (3) 
\begin{eqnarray}\Delta f_{rf}(t)= 
\begin{cases}
\frac{8}{T^3}(t-t_0)^2&t_0+0<t\le t_0+\frac{T}{4}\cr  
\frac{1}{T}-\frac{8}{T^3}[(t-t_0)-\frac{T}{2}]^2	&t_0+\frac{T}{4}<t\le t_0+\frac{3T}{4}\cr 
\frac{8}{T^3}[T-(t-t_0)]^2	&t_0+\frac{4T}{4}<t\le t_0+T\cr  

\end{cases}
\end{eqnarray}


Fig.~\ref{1st_derivation} and Fig.~\ref{2nd_derivation} show the first and second derivative of three rf frequency modulations.
%, which are smaller than the maximum time derivative of rf frequency during the acceleration ramp 64Hz/ms for the adiabaticity consideration. The acceleration ramp is an adiabatical process.
\begin{figure}[H]
   \centering   
   \includegraphics*[width=160mm]{1st_derivation.png}
   \caption{First derivation of three cases.}
   \label{1st_derivation}
\end{figure}
\begin{figure}[H]
   \centering   
   \includegraphics*[width=160mm]{2nd_derivation.png}
   \caption{Second derivation of three cases.}
   \label{2nd_derivation}
\end{figure}

Fig.~\ref{phase_shift_four_case} shows the corresponding phase shift modulation of three cases. 
\begin{figure}[H]
   \centering   
   \includegraphics*[width=160mm]{phase_shift_four_case.png}
   \caption{The phase shift modulation of three cases.}
   \label{phase_shift_four_case}
\end{figure}

\subsubsection{Longitudinal Dynamic Analysis for Frequency Modulation}
In this section, the average radial excursion, the relative momentum shift, the synchronous phase, the bucket size and the adiabaticity of three rf frequency modulations are analyzed. 
\begin{itemize}
%%%%%%%%%%%%%%%%%%%%%%%%%%%%%%%%%%%%%%%%%%%%%%%%%%%%%%%%%%%%%%%%%%%
\item Average radial excursion

The average radial excursion is calculated for the three cases by eq.~(\ref{eq:phaseR}). Fig.~\ref{radial} shows the calculation result ~\cite{bai12_first_2014}. 
\begin{figure}[H]
   \centering   
   \includegraphics*[width=160mm]{Radial.png}
   \caption{Average radial excursions of three cases.}
   \label{radial}
\end{figure}

\begin{table}[H]
\newcommand{\tabincell}[2]{\begin{tabular}{@{}#1@{}}#2\end{tabular}}
\caption{The maximum average radial excursion of three cases}
\label{radial excursion}
\begin{center}
    \begin{tabular}{ | c | c | c | c | c | c | }
    \hline
      &Case (1) & Case (2)&Case (3) \\ \hline
       \tabincell{c}{Max average\\radial excursion} &$4.18\times 10^{-6}$ &$4.18\times 10^{-6}$ &$4.18\times 10^{-6}$\\ \hline
			Time & \SI{3.5}{\ms} & \SI{3.5}{\ms} & \SI{3.5}{\ms}\\ \hline
    \end{tabular}
\end{center}
\end{table}
Tab. \ref{radial excursion} shows the maximum average radial excursion and the corresponding time of three cases. The maximum SIS18 tolerable radial excursion is $\pm 2.4\times10^{-4}$. For all three cases, the average radial excursion is within the acceptable range. Hence, all cases are applicable. 

%%%%%%%%%%%%%%%%%%%%%%%%%%%%%%%%%%%%%%%%%%%%%%%%%%%%%%%%%%%%%%%%%%%
\item Relative momentum shift

The relative momentum shift is calculated for three cases by eq.~\ref{eq:phaseP11}. Fig.~\ref{moment} shows the calculation result. 
\begin{figure}[H]
   \centering   
   \includegraphics*[width=160mm]{moment.png}
   \caption{Relative momentum shift of three cases.}
   \label{moment}
\end{figure}
\begin{table}[H]
\newcommand{\tabincell}[2]{\begin{tabular}{@{}#1@{}}#2\end{tabular}}
\caption{The maximum relative momentum shift of three cases}
\label{momentum excursion}
\begin{center}
    \begin{tabular}{ | c | c | c | c | c | c | }
    \hline
      &Case (1) & Case (2)&Case (3) \\ \hline
       \tabincell{c}{Max relative \\momentum shift} & $1.40\times 10^{-4}$ & $1.40\times 10^{-4}$ &$1.40\times 10^{-4}$\\ \hline
			Time 		& \SI{3.5}{\ms} & \SI{3.5}{\ms} & \SI{3.5}{\ms}\\ \hline
    \end{tabular}
\end{center}
\end{table}
Tab. \ref{momentum excursion} shows the maximum relative momentum shift and the corresponding time for three cases. The maximum SIS18 tolerable relative momentum shift is $\pm 0.008$. For all cases, the maximum relative momentum shift is within the acceptable range. Hence, all cases are applicable. 
%%%%%%%%%%%%%%%%%%%%%%%%%%%%%%%%%%%%%%%%%%%%%%%%%%%%%%%%%%%%%%%%%%%
\item Synchronous phase

The rf frequency modulations make the synchronous phase deviate from the nominal value $0^\circ$. Fig.~\ref{synch_phase} shows the changes in the synchronous phase $\phi_s$(t). It is calculated by substituting values into eq.~\ref{deriva_voltage}. For all three cases, the synchronous phase $\Delta \phi_s(t)$ during the modulations are continuous without any phase jumps and small enough. Hence, all cases are applicable.
\begin{figure}[H]
   \centering   
   \includegraphics*[width=160mm]{synch_phase.png}
   \caption{Changes in synchronous phase of three cases.}
   \label{synch_phase}
\end{figure}
%%%%%%%%%%%%%%%%%%%%%%%%%%%%%%%%%%%%%%%%%%%%%%%%%%%%%%%%%%%%%%%%%%%5
\item Bucket size

The bucket area factor \gls{symb:bucket_size} varies during rf frequency modulations. Before the modulations, the synchronous phase $\phi_s$=$0^\circ$ and  $\alpha_b(0^\circ) = 1$. By substituting the changes in synchronous phase into eq.~\ref{eq:buckt_area_factor11}, we get the ratio of bucket areas of a running bucket to the stationary bucket for three cases, see Fig.~\ref{bucket_size}.

\begin{figure}[H]
   \centering   
   \includegraphics*[width=160mm]{bucket_size.png}
   \caption{Ratio of bucket areas of a running bucket to the stationary bucket of three cases.}
   \label{bucket_size}
\end{figure}
Tab. ~\ref{bucket size} shows the minimum bucket area factor for three cases. For case (1) and (2), the running bucket area factor is larger than 86$\%$, which is larger than that of the case (3). Hence, case (1) and (2) are preferred compared with the case (3). 
\begin{table}[H]
\newcommand{\tabincell}[2]{\begin{tabular}{@{}#1@{}}#2\end{tabular}}
\caption{The minimum bucket area factor of three cases}
\label{bucket size}
\begin{center}
    \begin{tabular}{ | c | c | c | c | c | c | }
    \hline
      &Case (1) & Case (2)&Case (3) \\ \hline
       \tabincell{c}{Min bucket \\area factor} & 86.0$\%$ & 86.5$\%$ & 82.5$\%$\\ \hline
			Time 		& \SI{1.750}{\ms} and \SI{5.250}{\ms} &\tabincell{c}{\SI{1.167}{\ms}-\SI{2.333}{\ms}, \\ \SI{4.667}{\ms}-\SI{5.833}{\ms}}  & \SI{1.750}{\ms} and \SI{5.250}{\ms}\\ \hline
    \end{tabular}
\end{center}
\end{table}

%%%%%%%%%%%%%%%%%%%%%%%%%%%%%%%%%%%%%%%%%%%%%%%%%%%%%%%%%%%%%%%%%%%5
\item Adiabaticity

By substituting the values of $\phi_s(t)$, $\dot{\phi_s(t)}$ and the angular synchrotron frequency into eq.~\ref{eq:derivation}, we get the adiabaticity parameter $\varepsilon$ for three cases, see Fig.~\ref{adiabaticity2}. 

Tab. ~\ref{adiabaticity_param} shows the maximum adiabaticity parameter for three cases. For case (1), the maximum of $\varepsilon$ is 0.000030. For case (2), the maximum of $\varepsilon$ occurs at $1/6T$, $2/6T$, $4/6T$ and $5/6T$, when the change of the synchronous phase $\dot{\phi_s(t)}$ is big, shown in Fig.~\ref{synch_phase}. For case (3), the maximum of $\varepsilon$ occurs at $1/4T$ and $3/4T$, when the change of the synchronous phase $\dot{\phi_s(t)}$ is big. For all three cases, the adiabaticity parameter has the order of magnitude of $-5$, so all three cases are applicable. 

\begin{figure}[H]
   \centering   
   \includegraphics*[width=160mm]{adiabaticity2.png}
   \caption{Adiabaticity parameter of three cases.}
   \label{adiabaticity2}
\end{figure}
\end{itemize}

\begin{table}[H]
\newcommand{\tabincell}[2]{\begin{tabular}{@{}#1@{}}#2\end{tabular}}
\caption{The maximum adiabaticity of three cases}
\label{adiabaticity_param}
\begin{center}
    \begin{tabular}{ | c | c | c | c | c | c | }
    \hline
      &Case (1) & Case (2)&Case (3) \\ \hline
       \tabincell{c}{Maximum \\adiabaticity} & $5.3\times10^{-5}$ & $5.9\times10^{-5}$ & $6.3\times10^{-5}$\\ \hline
			Time 		& \tabincell{c}{\SI{0.875}{\ms}, \SI{2.625}{\ms}\\ \SI{4.250}{\ms} and \SI{6.125}{\ms} }&\tabincell{c}{\SI{1.167}{\ms}, \SI{2.333}{\ms}, \\ \SI{4.667}{\ms} and \SI{5.833}{\ms}} & \SI{1.750}{\ms} and \SI{5.250}{\ms}\\ \hline
    \end{tabular}
\end{center}
\end{table}
%%%%%%%%%%%%%%%%%%%%%%%%%%%%%%%%%%%%%%%%%%%%%%%%%%%%%%%%%%%%%%%%%
\subsubsection{Transverse Dynamics Analysis for Frequency mMdulation}
For the SIS18, the chromaticity $Q^`_x$ and $Q^`_y$ is $4.17$ and $3.4$. Substituting chromaticity and maximum momentum shift (see. Tab. \ref{momentum excursion}) into eq. ~\ref{eq:chromaticity_x}. The chromatic \gls{glos:tune} shift $\Delta Q_x$ and $\Delta Q_y$ during rf modulations for three cases can be calculated. Beacuse case (1), case (2) and case (3) have same maximum relative momentum shift, the chromatic tune shift is same for three rf frequency modulations.

%Case (1) 
%\begin{equation}
%\Delta Q_x = 4.17 \cdot 9.83 \cdot 10^{-5}=4.10 \cdot 10^{-4}
%\end{equation}
%\begin{equation}
%\Delta Q_y = 3.4 \cdot 9.83 \cdot 10^{-5}=3.34 \cdot 10^{-4} 
%\end{equation}
%
%Case (2)
%\begin{equation}
%\Delta Q_x = 4.17 \cdot 1.38 \cdot 10^{-4}=5.75 \cdot 10^{-4}
%\end{equation}
%\begin{equation}
%\Delta Q_y = 3.4 \cdot 1.38 \cdot 10^{-4}=4.69 \cdot 10^{-4} 
%\end{equation}


\begin{equation}
\Delta Q_x = 4.17\times 1.40\times 10^{-4}=5.84 \times 10^{-4}
\end{equation}
\begin{equation}
\Delta Q_y = 3.4\times 1.40\times 10^{-4}=4.76\times 10^{-4} 
\end{equation}

%Case (2) 
%\begin{equation}
%\Delta Q_x = 4.17 \cdot 1.48 \cdot 10^{-4}=6.17 \cdot 10^{-4}
%\end{equation}
%\begin{equation}
%\Delta Q_y = 3.4 \cdot 1.48 \cdot 10^{-4}=5.03 \cdot 10^{-4} 
%\end{equation}

The chromatic tune shift for three cases are significantly small, which could be neglected.
%%%%%%%%%%%%%%%%%%%%%%%%%%%%%%%%%%%%%%%%%%%%%%%%%%%%%%%%%%%%%%%%%
\subsection{Frequency Detune for Frequency Beating}
In the case of the frequency beating method, we guarantee the extraction and injection energy always match, which means that the momentum is not affected by the frequency detune, namely $\Delta p = 0$. Hence, the frequency detune has influence only on the longitudinal dynamics.

\subsubsection{Longitudinal Dynamics Analysis for Frequency Detune}
For the frequency beating method, the rf frequency detune is done directly after the SIS18 rf ramp. Due to the SIS18 $U^\mathit{U28+}$ lattice, the SIS18 $U^\mathit{U28+}$ accepted orbit excursion is~\cite{liebermann_fair_2013}
\begin{equation}
\frac{\Delta{R}}{R} = \pm 2.4 \times 10^{-4}
%\gls{symb:radius} = \pm 2.4 \cdot 10^{-4}
\end{equation}
From eq. ~\ref{eq:eq4}, the tolerate rf frequency change for $U^{28+}$ at the extraction energy \SI{200}{MeV/u} is
\begin{equation}
\frac{\Delta{f}_\mathit{rf}}{f_\mathit{rf}} = \pm 2.4 \times 10^{-4}
%\gls{symb:freq} = \pm 2.4 \cdot 10^{-4}
\end{equation}

%\begin{equation}
%\frac{\Delta{B}}{B}= \pm 8.1 \times 10^{-3}
%%\gls{symb:magnetic}=\frac{\Delta{f}}{f}{\gamma_t}^2 = \pm 8.1 \cdot 10^{-3}
%\end{equation}

where the maximum rf frequency detune approximates to \SI{370}{Hz} for the cavity rf frequency of \SI{1.572536}{MHz} of $U^{ 28+}$. Fig.~\ref{sis18_ramp} shows the rf frequency detune during the rf ramp. In the simulation, the rf frequency is detuned at \SI{0.2756}{s} with \SI{6.08}{Hz/us}, see blue rectangle in Fig.~\ref{sis18_ramp}. For the sake of simplicity, \SI{200}{Hz} is used as the frequency detune. The SIS18 needs approximate \SI{33}{us} to reach \SI{200}{Hz} with \SI{6.08}{Hz/us}.
\begin{figure}[!htb]
   \centering   
   \includegraphics*[width=160mm]{sis18_ramp.png}
   \caption{Frequency detune during the SIS18 $U^{28+}$ rf ramp.}
   \label{sis18_ramp}
\end{figure}

%\begin{figure}[!htb]
%   \centering   
%   \includegraphics*[width=160mm]{detune_ramp.jpg}
%   \caption{$U^{28+}$ rf detune during the rf ramp}
%   \label{detune_ramp}
%\end{figure}

%From eq.~\ref{eq:eq4} and eq.~\ref{eq:eq5}, we could get the corresponding radial excursion and the magnetic field change during the detune process. The maximum radial excursion is $-1.27 \times 10^{-4}$ and the maximum magnetic field change is $4.3 \times 10^{-3}$ at the end of the rf detune process.  
%%%%%%%%%%%%%%%%%%%%%%%%%%%%%%%%%%%%%%%%%%%%%%%%%%%%%%%%%%%%%%%%%%%%%%%%%%%%%%%%%%%%%%%%%%%%%%%%%%%%%%%%
\section{GMT systematic Investigation}
The B2B transfer system makes use of certain aspects of the GMT system to implement the data collection, merging and redistribution. The main task of the data merging is the calculation of the synchronization window, within which bunches could be injected into buckets with the bunch-to-bucket center mismatch smaller than the upper bound. The data collection and redistribution make use of the WR network, so the test and measurement of the WR network for the B2B transfer is important. 

\subsection{Calculation of Synchronization Window}
According to the phase difference between two rf systems of two synchrotrons, the fine time for the correct phase alignment between two synchronization frequencies for both the phase shift and frequency beating methods can be calculated. This time is called the ``\gls{glos:best_align}`` and denoted by $t_{best}$, see Fig. ~\ref{alignment}. The uncertainty is a non-negative parameter characterizing the dispersion of the values attributed to a measured quantity. Because of the \gls{glos:uncertainty} ~\cite{taylor_introduction_1982} of the phase extrapolation and rf frequency modulation, the fine alignment lies between \gls{symb:best_align} - $\delta t_{best}$ and $t_{best}$ + $\delta t_{best}$, where \gls{symb:probable_aligh} is the uncertainty of the alignment. [$t_{best}$ - $\delta t_{best}$, $t_{best}$ + $\delta t_{best}$] is called the ``\gls{glos:pro_align}``. In Sec. 6.2.1.1 and Sec. 6.2.1.2, the calculation of the best estimation of alignment and the probable range of alignment for the phase shift and frequency beating methods are explained. The probable range of alignment is within the synchronization window. For the correct selection of the same rising edge of the bucket indication signal at different SCUs, the start of the synchronization window must be properly calculated. In Sec. 6.2.1.3, the calculation of the start of the synchronization window is explained. 
\begin{figure}[!htb]
   \centering   
   \includegraphics*[width=160mm]{alignment.jpg}
   \caption{The illustration of the best estimate of alignment, the probable range of alignment and the synchronization window.}
   \label{alignment}
\end{figure}

%In fact, two SIS100 revolution periods is enough for the correct bucket selection, achieving much preciser injection. The beginning of the synchronization window denotes by $WIN_{start}$. The synchronization window is within the range [$WIN_{start}$ , $WIN_{start}$  + 2 $\cdot T_{rev}^{SIS100}$]. $T_{rev}^{SIS100}$is the revolution period of SIS100, which equals to 6.359 us for U$^{28+}$ at 200Mev/u.  

For both the phase shift and frequency beating methods, the calculation is based on the extrapolated phase of the rf signal locally. Here the $U^{28+}$ B2B transfer from the SIS18 to the SIS100 is taken as an example, two synchronization frequencies are $f_{\mathit{syn}}^{SIS18}=f_{\mathit{rf}}^{SIS18}$ and $f_{\mathit{syn}}^{SIS100}=f_{\mathit{rf}}^{SIS100}$ and two Reference RF Signals are $f_{\mathit{B2B}}^{SIS18}=1/5f_{\mathit{rev}}^{SIS18}$ and $f_{\mathit{B2B}}^{SIS100}=f_{\mathit{rev}}^{SIS100}$ (more details about $f_{\mathit{syn}}^{X}$ and $f_\mathit{B2B}^{X}$, please see Chap. ~\ref{background} and Chap. ~\ref{concept}.). The PAP module extrapolates the phase $\psi^{SIS100}$ for $f_{\mathit{B2B}}^{SIS100}$ and $\psi^{SIS18}$ for $f_{\mathit{B2B}}^{SIS18}$ at $t_{\psi}^\mathit{SIS18}=t_{\psi}^\mathit{SIS100}$. The phase difference between two synchronization frequencies is
\begin{equation}
\Delta \phi_\mathit{syn}=\frac{f_{\mathit{syn}}^{SIS100}}{f_{\mathit{rev}}^{SIS100}}(\psi^\mathit{SIS100}-\psi^\mathit{SIS18}) \mod 2\pi =10 \times(\psi^\mathit{SIS100}-\psi^\mathit{SIS18}) \mod 2\pi
\label{phase_diff_18to100}
\end{equation}

%Fig.~\ref{Calculation_symble} illustrates some basic definition of symbols for the calculation. 
%\begin{figure}[!htb]
%   \centering   
%   \includegraphics*[width=160mm]{Calculation_symble.jpg}
%   \caption{The illustration of symbols for the calculation.}
%   \label{Calculation_symble}
%\end{figure}
%$\phi_{h=2}^{SIS18}$and $\phi_{h=10}^{SIS100}$ are individual rf phase of SIS18 and SIS100 Reference RF Signals at $t_{\psi}$. The relationship between \gls{symb:h2phase18}, \gls{symb:h10phase100} and $\psi_{h=1/5}^{SIS18}$, $\psi_{h=1}^{SIS100}$ are given by eq.~\ref{SIS18_phase} and eq.~\ref{SIS100_phase}. 

%\begin{equation}
%\phi_{h=2}^{SIS18} =  \frac {\frac{\psi_{h=1/5}^{SIS18}}{360^\circ}\cdot {T_{h=1/5}^{SIS18}} \mod {T_{h=2}^{SIS18}}}{T_{h=2}^{SIS18}}\cdot {360^\circ} \label{SIS18_phase}
%\end{equation}
%\begin{equation}
%\phi_{h=10}^{SIS100} =  \frac {\frac{\psi_{h=1}^{SIS100}}{360^\circ}\cdot {T_{h=1}^{SIS100}} \mod {T_{h=10}^{SIS100}}}{T_{h=10}^{SIS100}}\cdot {360^\circ} \label{SIS100_phase}
%\end{equation}
%substituting $T_{h=2}^{SIS18}\cdot 10=T_{h=1/5}^{SIS18}$, $T_{h=10}^{SIS100}\cdot 10=T_{h=1}^{SIS100}$ into eq.\ref{SIS18_phase} and eq.\ref{SIS100_phase} yields
% \begin{equation}
%\phi_{h=2}^{SIS18} =  \frac {\frac{\psi_{h=1/5}^{SIS18}\cdot 10}{360^\circ}\cdot {T_{h=2}^{SIS18}} \mod {T_{h=2}^{SIS18}}}{T_{h=2}^{SIS18}}\cdot {360^\circ} \label{SIS18_phase1}
%\end{equation}
%\begin{equation}
%\phi_{h=10}^{SIS100} =  \frac {\frac{\psi_{h=1}^{SIS100}\cdot 10}{360^\circ}\cdot {T_{h=10}^{SIS100}} \mod {T_{h=10}^{SIS100}}}{T_{h=10}^{SIS100}}\cdot {360^\circ} \label{SIS100_phase1}
%\end{equation}

Here we explain the inevitable uncertainty of the phase extrapolation and rf frequency modulation. 
\begin{itemize}
\item Uncertainty of the phase extrapolation

The longer the time is used for the phase extrapolation, the more precise the extrapolated phase will be. If the PAP module use \SI{500}{us} to extrapolate the phase, the uncertainty of the extrapolated phase (denoted as \gls{symb:uncertainty_time}) is \SI{100}{ps} ~\cite{ferrand_development_????}, see eq. ~\ref{jitter_measure_t}. The uncertainty of the extrapolated phase from the time domain to the phase domain (denoted as \gls{symb:un_h1phase100}) is calcuated by eq. ~\ref{jitter_measure_p}.  
\begin{equation} 
\delta t_\psi^X= 100ps
\label{jitter_measure_t}
\end{equation}
\begin{equation} 
\delta \psi^{SIS18}=\delta\psi^{SIS100}=
\frac {100ps}{T_\mathit{rev}^\mathit{SIS100}} \times {360^{\circ}}\approx \pm0.006^\circ
\label{jitter_measure_p}
\end{equation}
 
Based on eq.~\ref{phase_diff_18to100} and eq.~\ref{jitter_measure_p}, the uncertainty of the phase difference between the SIS18 and SIS100 synchronization frequencies is
\begin{equation}
\Delta \phi_\mathit{syn}\approx10 \times[0.006^\circ-(-0.006^\circ)] \mod 2\pi\approx 0.12^\circ
\end{equation}

%\begin{equation}
%\begin{aligned}
%\delta \phi_{h=2}^{SIS18} = \sqrt {(\frac{\partial \phi_{h=2}^{SIS18}}{\partial \psi_{h=2}^{SIS18}} \delta \psi_{h=2}^{SIS18})^2}=\sqrt {(10 \cdot \delta \psi_{h=2}^{SIS18})^2}=0.06^\circ
%\label{phi_jitter1}
%\end{aligned}
%\end{equation}
%\begin{equation}
%\delta \phi_{h=10}^{SIS100} = \sqrt {(\frac{\partial \phi_{h=10}^{SIS100}}{\partial \psi_{h=1}^{SIS100}} \delta \psi_{h=10}^{SIS100})^2}=\sqrt {(10 \cdot \delta \psi_{h=10}^{SIS100})^2}=0.06^\circ
%\label{phi_jitter2}
%\end{equation}

\item Uncertainty of the rf frequency modulation

For the rf frequency modulation, the uncertainty is $0.2^\circ$ at \SI{5.4}{MHz} ~\cite{laier_funktional-spezifikation_2011}. We calculate the uncertainty in the time domain, see eq.~\ref{freq_jitter_t}.
\begin{equation}
\delta \Delta f_\mathit{rf} =\delta \Delta f_\mathit{syn}= \frac{0.2^\circ}{360^\circ} \times {\frac{1}{5.4MHz}}\approx 100ps
\label{freq_jitter_t}
\end{equation}
%
%The precision of the rf frequency is 0.05Hz. 
%\begin{equation}
%\delta \Delta f = 0.05Hz
%\label{freq_jitter_f}
%\end{equation}



\end{itemize}
%%%%%%%%%%%%%%%%%%%%%%%%%%%%%%%%%%%%%%%%%%%%%%%%%%%%%%%%%%%%%%%%%%%%%%%%%%%%%%%
\subsubsection{Best Estimate of Alignment and probable Range of Alignment for Phase Shift Method}





%Different relation between $\phi_{h=2}^{SIS18}$ and $\phi_{h=10}^{SIS100}$ requires different phase adjustment for SIS18. Fig.~\ref{phase_shift} illustrates all scenarios of their relation and the required phase adjustment for each scenario. We would like to introduce a phase shift of up to $\pm 180^\circ$. The blue and red line represents the phase of SIS100 and SIS18 Reference RF Signal. The clockwise arrow from the SIS18 to SIS100 rf phase represents the negative phase adjustment for SIS18 and the anticlockwise represents the positive phase adjustment. The required phase adjustment of SIS18 is denoted by $\Delta \phi_{shift}$.
%
%
%\begin{itemize}
%    \item Scenario (a): $\phi_{h=10}^{SIS100}\in [0,90^\circ)$, see Fig.~\ref{frequency_beating} (a).
%
%	\begin{itemize}
%		\item $\phi_{h=10}^{SIS100}< \phi_{h=2}^{SIS18}< \phi_{h=10}^{SIS100} +180^\circ$, which denotes by the yellow semicircle in Fig.~\ref{frequency_beating} (a). The phase adjustment is
%    \begin{equation}
%			\Delta \phi_{shift}=-(\phi_{h=2}^{SIS18} - \phi_{h=10}^{SIS100})
%    \end{equation}
%    		\item $\phi_{h=2}^{SIS18} < \phi_{h=10}^{SIS100}$ or  $\phi_{h=2}^{SIS18} >\phi_{h=10}^{SIS100} +180^\circ$, which denotes by the white semicircle in Fig.~\ref{frequency_beating} (a). The phase adjustment is
%    \begin{equation}
%			\Delta \phi_{shift}= 360^\circ - \phi_{h=2}^{SIS18} + \phi_{h=10}^{SIS100}
%    \end{equation}
%	\end{itemize}
%\begin{figure}[H]
%   \centering   
%   \includegraphics*[width=130mm]{phase_shift_synch_window_cal.jpg}
%   \caption{Scenarios for the phase shift method.}
%   \label{phase_shift}
%\end{figure}
%    \item Scenario (b): $\phi_{h=10}^{SIS100}\in [90,180^\circ)$, see Fig.~\ref{frequency_beating} (b). 
%
%	\begin{itemize}
%		\item $\phi_{h=10}^{SIS100}< \phi_{h=2}^{SIS18}< \phi_{h=10}^{SIS100} +180^\circ$, which denotes by the yellow semicircle in Fig.~\ref{frequency_beating} (b). The phase adjustment is
%	    \begin{equation}		
%\Delta \phi_{shift}=-(\phi_{h=2}^{SIS18} - \phi_{h=10}^{SIS100})
%    \end{equation}
%    		\item $\phi_{h=2}^{SIS18} < \phi_{h=10}^{SIS100}$ or  $\phi_{h=2}^{SIS18} >\phi_{h=10}^{SIS100} +180^\circ$, which denotes by the white semicircle in Fig.~\ref{frequency_beating} (b).  The phase adjustment is
%    \begin{equation}			
%\Delta \phi_{shift}=360^\circ - \phi_{h=2}^{SIS18} + \phi_{h=10}^{SIS100}
%    \end{equation}
%	\end{itemize}
%    \item Scenario (c): $\phi_{h=10}^{SIS100}\in [180,270^\circ)$, see Fig.~\ref{frequency_beating} (c). The phase adjustment is
%
%	\begin{itemize}
%		\item $\phi_{h=2}^{SIS18} > \phi_{h=10}^{SIS100}$ or  $\phi_{h=2}^{SIS18} < \phi_{h=10}^{SIS100} +180^\circ - 360^\circ $, which denotes by the yellow semicircle in Fig.~\ref{frequency_beating} (c). The phase adjustment is
%    \begin{equation}			
%\Delta \phi_{shift}=-(360^\circ - \phi_{h=10}^{SIS100}+ \phi_{h=2}^{SIS18})
%    \end{equation}
%    		\item $\phi_{h=10}^{SIS100}-180^\circ < \phi_{h=2}^{SIS18}< \phi_{h=10}^{SIS100}$, which denotes by the white semicircle in Fig.~\ref{frequency_beating} (c). The phase adjustment is
%    \begin{equation}			
%\Delta \phi_{shift}=\phi_{h=10}^{SIS100}-\phi_{h=2}^{SIS18}
%    \end{equation}
%	\end{itemize}
%    \item Scenario (d): $\phi_{h=10}^{SIS100}\in [270,360^\circ)$, see Fig.~\ref{frequency_beating} (d).
%
%	\begin{itemize}
%		\item $\phi_{h=10}^{SIS100}-180^\circ < \phi_{h=2}^{SIS18}< \phi_{h=10}^{SIS100}$, which denotes by the yellow semicircle in Fig.~\ref{frequency_beating} (d). The phase adjustment is 
%	    \begin{equation}	
%\Delta \phi_{shift}=\phi_{h=10}^{SIS100}-\phi_{h=2}^{SIS18}	
%    \end{equation}
%    		\item $\phi_{h=2}^{SIS18} > \phi_{h=10}^{SIS100}$ or  $\phi_{h=2}^{SIS18} < \phi_{h=10}^{SIS100} +180^\circ - 360^\circ $ , which denotes by the white semicircle in Fig.~\ref{frequency_beating} (d). 
%    \begin{equation}			
%\Delta \phi_{shift}=-(360^\circ - \phi_{h=10}^{SIS100}+ \phi_{h=2}^{SIS18})
%    \end{equation}
%	\end{itemize}
%\end{itemize}

The duration of the rf frequency modulation for the phase shift is $T$, so the best estimate of alignment is expressed by 

\begin{equation}
t_\mathit{best} = t_{\psi}^X + T \label{Phase_win}
\end{equation}
The uncertainty of the phase extrapolation $t_{\psi}^X$ is $\pm$\SI{100}{ps}, see eq.~\ref{jitter_measure_t}. The uncertainty of the phase shift is casued by the rf frequency modulation, whose uncertainty is \SI{100}{ps}, see eq.~\ref{freq_jitter_t}. The phase shift uncertainty equals to the uncertainty of the phase shift duration time, namely $\delta T = 100ps$. Both cause an uncertainty in the best estimate of alignment $t_{best}$.
\begin{equation}
%\begin{aligned}
%\delta t_{best} =\sqrt {(\frac {\partial t_{best}}{\partial t_{\psi}^X}\delta t_{\psi}^X)^2 + (\frac {\partial t_{best}}{\partial T}\delta T)^2} \\
% =\sqrt {(\delta t_{\psi}^X)^2+T^2} \approx \sqrt { 100ps^2+100ps^2}\approx 140ps \label{Phase_uncertainty}
%\end{aligned}
\delta t_\mathit{best} =\sqrt {(\frac {\partial t_{best}}{\partial t_{\psi}^X}\delta t_{\psi}^X)^2 + (\frac {\partial t_{best}}{\partial T}\delta T)^2} 
 =\sqrt {(\delta t_{\psi}^X)^2+(\delta T)^2}\approx 140ps \label{Phase_uncertainty}
\end{equation}


The uncertainty of the alignment for the phase shift method is about \SI{140}{ps}. So the proper range of alignment is [$t_{best}$-\SI{140}{ps}, $t_{best}$+\SI{140}{ps}] for $U^{28+}$ B2B transfer from the SIS18 to the SIS100.
%%%%%%%%%%%%%%%%%%%%%%%%%%%%%%%%%%%%%%%%%%%%%%%%%%%%%%%%%%%%%%%%%%%%%%
\subsubsection{Best Estimate of Alignment and probable Range of Alignment for Frequency Beating Method}
%Fig.~\ref{frequency_beating} illustrates two scenarios for the frequency beating method. With the frequency beating method, SIS18 can only achieve positive phase adjustment, which is denoted by \gls{symb:phase_just_frequency_beating}. E.q.~\ref{sync_time} shows the best estimate of alignment for the phase adjustment of $\Delta \phi_{adjustment}$.
%\begin{equation}
%	 t_{best} = t_{\psi}+\frac {\Delta \phi_{adjustment}}{{360^\circ} \cdot {\Delta f}} \label {sync_time}
%   \end{equation}
%where \gls{symb:beating_freq} is the beating frequency.
%\begin{figure}[!htb]
%   \centering   
%   \includegraphics*[width=90mm]{frequency_beating_synch_window_cal.jpg}
%   \caption{Two scenarios for the frequency beating method.}
%   \label{frequency_beating}
%\end{figure}
%
%According to the relation between $\phi_{h=2}^{SIS18}$ and $\phi_{h=10}^{SIS100}$, there are two scenarios, see Fig.~\ref{frequency_beating}.
%\begin{itemize}
%    \item Scenario (a): $\phi_{h=2}^{SIS18} < \phi_{h=10}^{SIS100}$
%	\begin{equation}
%	 \Delta \phi_{adjustment} = \phi_{h=10}^{SIS100} - \phi_{h=2}^{SIS18}\label {great}
%   \end{equation}
%   Replacing $\Delta \phi_{adjustment}$ in eq.~\ref{sync_time} with eq.~\ref{great}, we have
%	\begin{equation}
%	 t_{best} =t_{\psi} +\frac {\phi_{h=10}^{SIS100} - \phi_{h=2}^{SIS18}}{{360^\circ} \cdot {\Delta f}} \label {beating_win_1}
%   \end{equation}
%     \item  Scenario (b): $\phi_{h=2}^{SIS18} \ge \phi_{h=10}^{SIS100}$
%	\begin{equation}
%	 \Delta \phi_{adjustment} = 360^\circ - (\phi_{h=2}^{SIS18}-\phi_{h=10}^{SIS100}) \label {less}
%   \end{equation}
%  Replacing $\Delta \phi_{adjustment}$ in eq.~\ref{sync_time} with eq.~\ref{less}, we have
%
%	\begin{equation}
%	 t_{best} =t_{\psi} +\frac {360^\circ - (\phi_{h=2}^{SIS18}-\phi_{h=10}^{SIS100})}{{360^\circ} \cdot {\Delta f}} \label {beating_win_2}
%   \end{equation}
%\end{itemize}
%Based on these two scenarios, we could deduce the formula for the best estimate of alignment. 
%	\begin{equation}
%	 t_{best} =t_{\psi} +\frac {{\Delta n} \cdot {360^\circ} - (\phi_{h=2}^{SIS18}-\phi_{h=10}^{SIS100})}{{360^\circ} \cdot {\Delta f}} \label {beating_win_2}
%   \end{equation}
%where $\bigtriangleup{n}$ equals 0 when  $\phi_{h=2}^{SIS18} < \phi_{h=10}^{SIS100}$ and equals 1 when  $\phi_{h=2}^{SIS18} \ge \phi_{h=10}^{SIS100}$.
The time of the phase alignment is determined by the phase difference, the target phase difference and the beating frequency, see eq. ~\ref{best_align_beating}.
\begin{equation}
t_\mathit{best}= t_\psi^\mathit{X}+\frac{\Delta \phi_\mathit{shift}}{2\pi}\cdot\frac{1}{|f_{\mathit{syn}}^\mathit{src}-f_{\mathit{syn}}^\mathit{trg}|}
\label{best_align_beating}
\end{equation}
where $\Delta \phi_\mathit{shift}$ is the required phase difference, the relation between $\Delta \phi_\mathit{shift}$ and $\Delta \phi_\mathit{syn}$ is explained in Chap. ~\ref{concept}. Becasue $\Delta \phi_\mathit{shift}$ and $\Delta \phi_\mathit{syn}$ have a linear relationship and the linear slop is $1$, the uncertainty of $\Delta \phi_\mathit{shift}$ (denoted as $\delta \Delta \phi_\mathit{shift}$) is equal to $\delta \Delta \phi_\mathit{syn}$, namely $\delta \Delta \phi_\mathit{shift}\approx 0.12^\circ$.   

The uncertainty of the alignment is the result of the \gls{glos:error_pro} of the phase extrapolation and rf frequency detune, see eq.~\ref{beating_uncertainty}. Because the rf frequency detune has the long term stability, $\int\delta \Delta f_\mathit{syn}$=0, the uncertainty caused by rf frequency detune is 0. The uncertainty of the phase extrapolation is taken into consideration by $\Delta \phi_\mathit{shift}$. $\Delta f$ = \SI{200}{Hz} is taken as an example of the frequency detune. 
\begin{equation}
\begin{aligned}
\delta t_{best} =\sqrt {(\frac {\partial t_{best}}{\partial \phi_{h=2}^{SIS18}}\delta \phi_{h=2}^{SIS18})^2 + (\frac {\partial t_{best}}{\partial \phi_{h=10}^{SIS100}}\delta \phi_{h=10}^{SIS100})^2+(\frac {\partial t_{best}}{\partial \Delta f}\delta \Delta f)^2} \\
 =\sqrt {(\frac{-1}{{2\pi} \cdot {\Delta f}}\delta \phi_{h=2}^{SIS18})^2+(\frac{1}{{2\pi} \cdot {\Delta f}}\delta \phi_{h=10}^{SIS100})^2+(-\frac{{\Delta n} \cdot {2\pi} - (\phi_{h=2}^{SIS18}-\phi_{h=10}^{SIS100})}{{2\pi} \cdot {\Delta f}^2}\delta \Delta f)^2} \\
\le \sqrt {(\frac{-1}{{2\pi} \cdot {200}}0.06^\circ)^2+(\frac{1}{{2\pi} \cdot {200}}0.06^\circ)^2+0}\\
\approx 1.178us \label{beating_uncertainty}
\end{aligned}
\end{equation}
From eq.~\ref{beating_uncertainty} we could get the uncertainty of the alignment is 1.178us, so the probable range of alignment is [$t_{best} – 1.178us, t_{best}+ 1.178us$].
%%%%%%%%%%%%%%%%%%%%%%%%%%%%%%%%%%%%%%%%%%%%%%%%%%%%%%%%%%%%%%%%%%%%%%
\subsubsection{Calculation of Synchronization Window and its Accuracy}
In the last section, we get the probable range of alignment, within which the two Reference Rf Signals could be aligned with each other. The synchronization window is used to select the revolution frequency marker for the extraction and injection kicker firing, which is closest to the probable range of alignment, See Fig.~\ref{accuracy_syn_win}. For the selection, the length of the synchronization window must be a least one SIS100 revolution period. The best estimate of the start of the synchronization window is exactly half revolution period before the selected revolution frequency marker. The blue and orange rectangles represent two scenarios of the probable range of alignment. In Fig.~\ref{accuracy_syn_win}, the $2^{nd}$ revolution frequency marker is the closest one to the probable range of alignment. The best estimate of the start of the synchronization window aligns with the negative zero crossing point of the revolution marker signal.

\begin{figure}[!htb]
   \centering   
   \includegraphics*[width=160mm]{accuracy_syn_win.jpg}
   \caption{The illustration of the synchronization window and its accuracy.}
   \label{accuracy_syn_win}
\end{figure}

For SIS100, the rf phase of the revolution frequency is $\psi_{h=1}^{SIS100}$ at $t_{\psi}$. We could calculate the rf phase \gls{symb:phase_s_alignment} of the revolution frequency at the start of the probable rang of alignment, $t_{best}$-$\delta t_{best}$.
\begin{equation}
\begin{aligned}
\psi_{s\_alignment}=\frac{(t_{best}-\delta t_{best}-t_{\psi}- \frac{360^\circ-\psi_{h=1}^{SIS100}}{360^\circ} \cdot {T_{h=1}^{SIS100}}) \mod T_{h=1}^{SIS100}}{T_{h=1}^{SIS100}}\cdot {360^\circ} 
\label{phase_after_syn}
\end{aligned}
\end{equation}

For the calculation of the best estimate of the start of the synchronization window, there are two scenarios. \gls{symb:win_correction} is the time correction for the start of the probable range of alignment to the best estimate of the start of the synchronization  window, see Fig.~\ref{accuracy_syn_win}.
\begin{itemize}
\item $\psi_{s\_alignment}\in [0^\circ,180^\circ)$, the orange rectangle in Fig.~\ref{accuracy_syn_win}
\begin{equation}
\begin{aligned}
\Delta t_{win \_correct}=\frac{\psi_{s\_alignment}}{360^\circ}\cdot T_{h=1}^{SIS100}+\frac{T_{h=1}^{SIS100}}{2}
\end{aligned}
\end{equation}
\begin{equation}
\begin{aligned}
WIN_{start}= t_{best}- \delta t_{best}-\Delta t_{win \_correct}
\end{aligned}
\end{equation}


\item $\psi_{s\_alignment}\in [180^\circ,360^\circ)$, the blue rectangle in Fig.~\ref{accuracy_syn_win}

\begin{equation}
\begin{aligned}
\Delta t_{win \_correct}=\frac{\psi_{s\_alignment}-180^\circ}{360^\circ}\cdot T_{h=1}^{SIS100}
\end{aligned}
\end{equation}
\begin{equation}
\begin{aligned}
WIN_{start}= t_{best}- \delta t_{best}-\Delta t_{win \_correct}
\end{aligned}
\end{equation}

\end{itemize}

The actual start of the synchronization window is impossible to be exactly at the best estimate of the start of the synchronization window because of the precision and trueness~\cite{_statistical_????}. The \gls{glos:precision} is defined as the closeness of agreement between the actual start of the synchronization window of different SCUs and the \gls{glos:trueness} as the closeness of agreement between the average actual start of the synchronization window of different SCUs and the best estimation start of the synchronization window. The precision comes from the random error, e.g. IO port \gls{TTL} signal rising oscillation. The trueness is the systematic error, e.g. FPGA process time. The \gls{glos:accuracy} is defined as the closeness of agreement between the observed start and the best estimate of the start of the synchronization window, which is the sum of the precision and trueness. The B2B transfer system will be used for many transfers for FAIR. Therefore, we have to find the most stringent accuracy requirement. The shortest revolution period of the target machine is \SI{433}{\ns}, which comes from RIB transfer from CR to HESR. We keep 10ns as a forbidden range, which means that the actual start is not allowed \SI{10}{\ns} before and after the revolution frequency marker. The green region in Fig.~\ref{accuracy_syn_win} represents the safety margin for the start of the synchronization window. So the accuracy of the start of the synchronization window must meet the requirement calculated by eq. ~\ref{accu}.
\begin{equation}
\begin{aligned}
Accuracy=\pm\frac{433-10 \cdot 2}{2}\approx \pm \SI{200}{\ns}\label{accu}
\end{aligned}
\end{equation}

%%%%%%%%%%%%%%%%%%%%%%%%%%%%%%%%%%%%%%%%%%%%%%%%%%%%%%%%%%%%%%%%%%%%%%%%%%%%%%%%%%%%%%%%%%%%%%%%%%%%%%
\subsection{Characterization of WR Network}
Within this dissertation, a network analyzer, a Xena traffic generator\footnote{\url{http://xenanetworks.com/layer-2-3-platform/}}, is used to characterize the properties of the WR network, which are relevant to the B2B transfer. The Xena traffic generator offers a new class of professional Layer 2-3 Gigabit Ethernet test platform. It is used to measure the \gls{glos:frame_loss_rate}\footnote{The ratio of the number of the lost frames to the number of the theoretic received frames of a tested port.}, the \gls{glos:latency}\footnote{The time interval between the time of Xena port receiving frame and the time of another Xena port sending frame.} and the \gls{glos:jitter}\footnote{The absolute value of the difference between the latency of two consecutive received frames belonging to the same stream from one Xena port to another Xena port. \newline\url{http://www.xenanetworks.com/wp-content/uploads/Measuring_Frame_latency_Variation.pdf}} for a network. The Xena traffic generator sends traffic streams with a unique stream ID and receives the identical traffic streams for identifying the latency, jitter and packet loss. For the measurements reported here, the following types of traffic are considered.

\begin{itemize}
    \item DM Broadcast 

The DM forwards broadcast timing frames downwards to all FECs. The average bandwidth for the DM broadcast is \SI{100}{Mbit/s}. The burst\footnote{A group of consecutive frames with shorter interframe gaps than frames arriving before or after the burst of frames.} speed is 12 packets per \SI{100}{\micro\second}.
 		\item DM Unicast 

The DM sends \SI{10}{Mbit/s} unicast timing frames to some specified FECs at the burst speed of 3 packets per \SI{300}{\micro\second}.
	\item B2B Unicast

The B2B source SCU sends two \gls{glos:timing_frame}s upwards to the DM within \SI{10}{\ms} for each cycle. Every supercycle contains four cycles. The maximum supercycle repetition frequency for FAIR is the repetition frequency of the $U^{28+}$ supercycle, \SI{2.82}{\Hz}. The bandwidth is $2.82\times4\times2\times880<$ \SI{20}{kbit/s}. 
	\item B2B Broadcast

Maximum 10 B2B broadcast timing frames are sent within \SI{10}{\ms} for each cycle. So the bandwidth is $2.82\times4\times10\times880<$ \SI{100}{kbit/s}.

	\item Management Traffic

The average bandwidth for the management traffic is \SI{10}{Mbit/s}. It broadcasts packets with random Ethernet frame length from 64 bytes to 1518 bytes. 
\end{itemize}

The requirements for the B2B Broadcast and Unicast traffic are summarized in Tab.~\ref{requirement_network}.
\begin{table}[!htb]
\newcommand{\tabincell}[2]{\begin{tabular}{@{}#1@{}}#2\end{tabular}}
\caption{B2B transfer requirements on the WR network}
\label{requirement_network}
\begin{center}
    \begin{tabular}{ | c | c | c | c | c | c | }
    \hline
     \tabincell{c}{} & \tabincell{c}{Frame \\ Loss Rate} & \tabincell{c}{Upper bound latency \\ of WR network} &\tabincell{c}{ Upper bound latency\\ per WR switch layer} \\ \hline
       \tabincell{c}{B2B \\ Broadcast} & $<10^{-12}$ & \SI{500}{\us} & \SI{60}{\us} \\ \hline
		\tabincell{c}{B2B \\ Unicast} & $<10^{-12}$ & \SI{500}{\us} & \SI{60}{\us}\\ \hline
    \end{tabular}
\end{center}
\end{table}

For the WR network for FAIR, three VLANs with different priorities are applied according to the importance of the traffic. The DM Broadcast, DM Unicast and B2B Unicast are asigned to the VLAN 7 with the highest priority. The B2B Broadcast is asigned to the VLAN 6 with the secondary high priority and the Management Traffic is asigned to the VLAN 5 with the lowest priority.

\subsubsection{WR Network Test Setup}

\begin{figure}[H]
   \centering   
   \includegraphics*[width=160mm]{GSI_use_case.png}
   \caption{The WR network test setup.}
   \label{GSI_use_case.jpg}
\end{figure}
Based on the mentioned traffic, the measurement setup is built, see Fig.~\ref{GSI_use_case.jpg}. Four WR switches are connected to the port 1 to 18 of the Xena traffic generator. All ports of four WR switches are assigned to three VLANs, VLAN 5, VLAN 6 and VLAN 7. Tab. ~\ref{test_setup_network} shows the bandwidth, VLAN, VLAN priority and usage of the traffic of each Xena port in details. The test is running for 45 days. More test configuration and results, please see ``Testing the WR Network of the FAIR General Machine Timing System`` ~\cite{prados_testing_2016}.
\renewcommand{\multirowsetup}{\centering} 
\begin{table}[!htb]
\newcommand{\tabincell}[2]{\begin{tabular}{@{}#1@{}}#2\end{tabular}}
\caption{The connection between the traffic generator and WR switches}
\label{test_setup_network}
\begin{center}
    \begin{tabular}{ | c | c | c | c | c | c | }
    \hline
	  \rowcolor[gray]{0.5}
     \tabincell{c}{Switch} & \tabincell{c}{Xena \\ Port} & \tabincell{c}{Traffic} &\tabincell{c}{ VLAN} &\tabincell{c}{Priority} &\tabincell{c}{Usage}\\ \hline
       \multirow{6}*{{\tabincell{c}{WR \\switch \\ 1}}}& Port 1 & \SI{100}{Mbit/s} 110bytes & 7 & 7 & DM Broadcast \\ \cline{2-6}
		 &Port 2 & Rx traffic &  &  &  \\ \cline{2-6}
		 &Port 3 &\SI{10}{Mbit/s} 110bytes & 7 & 7 & DM Unicast \\ \cline{2-6}
   		 &Port 4 & Rx traffic &  &  &  \\ \cline{2-6}
		 &Port 5 & Rx traffic &  &  &  \\ \cline{2-6}
		 &Port 6 & \SI{1}{Mbit/s} 64 - 1518 bytes & 5 & 5 &  \tabincell{c}{Management \\ Broadcast} \\ \hline
    \multirow{4}*{{\tabincell{c}{WR \\switch \\ 2}}}& Port 7 & \SI{2}{Mbit/s} 64 - 1518 bytes& 5 & 5 &  \tabincell{c}{Management \\ Broadcast} \\ \cline{2-6}
	& Port 8 & Rx traffic &  &  & \\ \cline{2-6}
	& Port 9 & Rx traffic &  &  & \\ \cline{2-6}
   & Port 10 & \SI{1}{Mbit/s} 64 - 1518 bytes& 5 & 5 &  \tabincell{c}{Management \\ Broadcast} \\ \hline
	\multirow{4}*{{\tabincell{c}{WR \\switch \\ 3}}}& Port 11 & Rx traffic &  &  & \\ \cline{2-6}
	& Port 12 & Rx traffic &  &  & \\ \cline{2-6}
   & Port 13 & \SI{2}{Mbit/s} 64 - 1518 bytes& 5 & 5 &  \tabincell{c}{Management \\ Broadcast} \\ \cline{2-6}
	& Port 14 & \SI{1}{Mbit/s} 64 - 1518 bytes& 5 & 5 &  \tabincell{c}{Management \\ Broadcast} \\ \hline
	\multirow{4}*{{\tabincell{c}{WR \\switch \\ 4}}}& Port 15 & \SI{1}{Mbit/s} 64 - 1518 bytes& 5 & 5 &  \tabincell{c}{Management \\ Broadcast} \\ \cline{2-6}
   & Port 16 & \SI{100}{kbit/s} 110bytes & 6 & 6 & B2B Broadcast \\ \cline{2-6}
	& Port 17 & \SI{20}{kbit/s} 110bytes & 7 & 7 & B2B Unicast \\ \cline{2-6}
	& Port 18 & \SI{2}{Mbit/s} 64 - 1518 bytes& 5 & 5 &  \tabincell{c}{Management \\ Broadcast} \\ \hline
    
    \end{tabular}
\end{center}
\end{table}

\subsubsection{Test Result of Frame Loss Rate }

The frame loss rate of the stream from the port 17 to the port 1 is measured for the B2B Unicast frames. The frame loss rate of the stream from the port 16 to other ports is measured for the B2B Broadcast traffic. The B2B Unicast frames have no frame loss. For the B2B Broadcast frames, the frame loss rate from the port 16 to the port 11, 12, 17 and 18 is $1.78\times 10^{-8}$, see Fig. ~\ref{packet_loss}. 

\begin{figure}[H]
   \centering   
   \includegraphics*[width=160mm]{packet_loss.png}
   \caption{Frame loss rate of the B2B Broadcast frames.}
   \label{packet_loss}
\end{figure}



\subsubsection{Test Result of Latency and Jitter }


\begin{itemize}
    \item Latency and jitter for the B2B Broadcast frames

For the B2B Broadcast frames, the latency and jitter of the stream from the port 16 to other ports is measured. 
		\begin{itemize}
    		\item[-] Average latency and jitter

Fig. ~\ref{average_latency_jitter} shows the test result for the average latency and jitter for the B2B Broadcast frames. Tab. \ref{avg latency jitter} shows the average latency and jitter of different WR switch layers. They meet the requirements of the B2B transfer. 
\begin{figure}[H]
   \centering   
   \includegraphics*[width=160mm]{average_latency_jitter.png}
   \caption{The average latency and jitter for B2B Broadcast frames.}
   \label{average_latency_jitter}
\end{figure}
\begin{table}[H]
\newcommand{\tabincell}[2]{\begin{tabular}{@{}#1@{}}#2\end{tabular}}
\caption{The average latency and jitter of the B2B Broadcast frames}
\label{avg latency jitter}
\begin{center}
    \begin{tabular}{ | c | c | c | c | c | c | }
    \hline
     & \tabincell{c}{WR switch\\4}  & \tabincell{c}{WR switch\\4, 3} &\tabincell{c}{WR switch\\4, 3, 2} &\tabincell{c}{WR switch\\4, 3, 2, 1} \\ \hline
       \tabincell{c}{Avg \\ latency} & \SI{6}{\us} & \SI{8}{\us} & \SI{11}{\us} & \SI{14}{\us}\\ \hline
		\tabincell{c}{Avg \\ jitter} & \SI{0}{\ns} & \SI{0}{\ns} & \SI{0}{\ns} & \SI{0}{\ns}\\ \hline
    \end{tabular}
\end{center}
\end{table}


			\item[-] Maximum Latency and jitter

Fig. ~\ref{Max_latency_jitter} shows the test result for the maximum latency and jitter for the B2B Broadcast frames. Tab. \ref{max latency jitter} shows the maximum latency and jitter of different WR switch layers. They meet the requirements of the B2B transfer.

\begin{figure}[H]
   \centering   
   \includegraphics*[width=160mm]{Max_latency_jitter.png}
   \caption{The maximum latency and jitter for B2B Broadcast frames.}
   \label{Max_latency_jitter}
\end{figure}
\begin{table}[H]
\newcommand{\tabincell}[2]{\begin{tabular}{@{}#1@{}}#2\end{tabular}}
\caption{The maximum latency and jitter of the B2B Broadcast frames}
\label{max latency jitter}
\begin{center}
    \begin{tabular}{ | c | c | c | c | c | c | }
    \hline
     & \tabincell{c}{WR switch\\4}  & \tabincell{c}{WR switch\\4, 3} &\tabincell{c}{WR switch\\4, 3, 2} &\tabincell{c}{WR switch\\4, 3, 2, 1} \\ \hline
       \tabincell{c}{Max \\ latency} & \SI{28}{\us} & \SI{34}{\us} & \SI{37}{\us} & \SI{41}{\us}\\ \hline
		\tabincell{c}{Max \\ jitter} & \SI{25}{\us} & \SI{25}{\us} & \SI{27}{\us} & \SI{30}{\us}\\ \hline
    \end{tabular}
\end{center}
\end{table}

		\end{itemize}
    \item Latency and jitter for the B2B Unicast frames

For the B2B unicast frames, the latency and jitter of the stream from the port 17 to the port 1 are measured. 

		\begin{itemize}
    		\item[-] Average latency and jitter

For the B2B Unicast frames, 4 WR switch network has approximate \SI{11}{\us} average latency and \SI{0}{\us} average jitter. 

\begin{figure}[H]
   \centering   
   \includegraphics*[width=160mm]{Avg_latency_jitter_unicast.png}
   \caption{The average latency and jitter for B2B Unicast frames.}
   \label{Avg_latency_jitter_unicast}
\end{figure}

			\item[-] Maximum Latency and jitter

For the B2B unicast frames, 4 WR switch network has approximate \SI{23}{\us} maximum latency and \SI{13}{\us} maximum jitter.

\begin{figure}[H]
   \centering   
   \includegraphics*[width=160mm]{Max_latency_jitter_unicast.png}
   \caption{The maximum latency and jitter for B2B Unicast frames.}
   \label{Max_latency_jitter_unicast}
\end{figure}

		\end{itemize}
\end{itemize}



\subsubsection{Conclusion}

Tab. ~\ref{result} shows the result of the test. The latency and jitter meet the requirements of the B2B Broadcast and B2B Unicast traffic. But the frame loss rate for the B2B Broadcast frames dosen't meet the requirement. The firmware of the WR switch is still under development by CERN.

\begin{table}[H]
\newcommand{\tabincell}[2]{\begin{tabular}{@{}#1@{}}#2\end{tabular}}
\caption{The result of the WR network test for the B2B transfer}
\label{result}
\begin{center}
    \begin{tabular}{ | c | c | c | c | c | c | }
    \hline
     \tabincell{c}{} & \tabincell{c}{Frame \\ Loss Rate} & \tabincell{c}{Average \\Latency }&\tabincell{c}{Maximum \\Latency}& \tabincell{c}{Average \\Jitter}&\tabincell{c}{Maximum \\Jitter } \\ \hline
       \tabincell{c}{B2B \\ Broadcast} &  $7.12\times10^{-8}$ & \SI{6}{\us}/switch & \SI{28}{\us}/switch &  \SI{0}{\us}/switch & \SI{25}{\us}/switch\\ \hline
		\tabincell{c}{B2B \\ Unicast} 	 & \SI{0}{\percent} & \tabincell{c}{\SI{11}{\us}/4switch \\ \SI{3}{\us}/switch}& \tabincell{c}{\SI{23}{\us}/4switch\\\SI{6}{\us}/switch}&  \tabincell{c}{\SI{0}{\us}/4switch\\\SI{0}{\us}/switch} & \tabincell{c}{\SI{13}{\us}/4switch\\\SI{4}{\us}/switch}\\ \hline
    \end{tabular}
\end{center}
\end{table}

For the B2B transfer system, the upper bound latency of the frames on the WR network is \SI{500}{\us} and the upper bound latency for each WR layer is \SI{60}{us}, see Tab.\ref{requirement_network}. The latency of the WR network is decided by the layers of WR switches and the length of the optical fiber. The latency of the optical fiber is about \SI{204}{\meter/\us}~\cite{_calculating_2012} and the longest distance in the FAIR campus is around \SI{2}{\kilo\meter}, so the latency of a \SI{2}{\kilo\meter} optical fiber is about \SI{10}{\us}. The layers of WR switches play a more important role in the latency. 

\begin{itemize}
    \item B2B Broadcast

		Here we calculate the tolerate layer of the WR switch between the B2B source \gls{SCU} and the B2B target SCU, between the B2B source SCU and the source trigger SCU and between the B2B source SCU and the target trigger SCU.  
		\begin{equation}
		\begin{aligned}
			\frac{\SI{500}{\us}-\SI{10}{\us}}{\SI{60}{\us/switch}}> 8
		\label {num_switch_b}
		\end{aligned}
		\end{equation}
	\item B2B Unicast

		Here we calculate the tolerate layer of the WR switch between the B2B source SCU and the DM.
		\begin{equation}
		\begin{aligned}
			\frac{\SI{500}{\us}-\SI{10}{\us}}{\SI{60}{\us/switch}}> 8
		\label {num_switch_b}
		\end{aligned}
		\end{equation}
\end{itemize}

%%%%%%%%%%%%%%%%%%%%%%%%%%%%%%%%%%%%%%%%%%%%%%%%%%%%%%%%%%%%%%%%%%%%%%%%%%%%%%%%%%%%%%%%%%%%%%%%%%%%%%%%
\section{Kicker systematic Investigation}
The SIS18 extraction kicker magnet consists of nine kicker units. In the existing topology, five kicker units are installed in the $1^{st}$ crate and the other four units are in the $2^{nd}$ crate. The width of each kicker unit is \SI{0.25}{m} and the distance between two kicker units is \SI{0.09}{m}. The distance between two crates is \SI{19.167}{m}. The SIS100 injection kicker magnet consists of six kicker units, which are equally located. The width of each kicker unit is \SI{0.22}{m} and the distance between two units is \SI{0.23}{m}. For the B2B transfer, the rise time of SIS18 extraction kicker and SIS100 injection kicker unit are \SI{90}{ns} and 1/20 of the revolution period. The rise time of kickers must fit within bunch gaps, 25$\%$ of the rf period of the cavity rf frequency ~\cite{udo_injection_2014, liebermann_sis100_2013}. The bunch gap is denoted by \gls{symb:G}. All the analysis in this section dose not consider the jitter of the kicker trigger signal. Here we are discussing about the following possibilities. 
\begin{itemize}
    \item For SIS18, whether the kicker units in the $2^{nd}$ crate could be fired a fixed delay after the firing of the kicker units in the $1^{st}$ crate for ion beams over the whole range of stable isotopes. 
    \item For SIS100, whether the kicker units could be fired instantaneously. 
\end{itemize} 

\subsection{SIS18 Extraction Kicker Magnet}
\begin{figure}[H]
   \centering   
   \includegraphics*[width=160mm]{kicker.jpg}
   \caption{Three scenarios for the delay of SIS18 extraction kicker.}
   \label{kicker}
\end{figure}
Here we take three ion beams, $H^+, U^{28}$ and $U^{73+}$, to check the possibility, because the boundary ion species have the most stringent requirements. Fig.~\ref{kicker} shows three scenarios of the firing delay between two crates. Beam is firstly kicked by kicker units in the $1^{st}$ crate and than kicked by the units in the $2^{nd}$ crate to the transfer line. The yellow and red ellipse represents the position of the bunches, when the kicker units in the $1^{st}$ and $2^{nd}$ crate are fired. The number in the ellipse is used to tell different bunches. The head of the bunch is at the right side. The bunch 2 is firstly kicked. Here we assume that the kicker units in the same crate are triggered instantaneous. d denotes the distance between two crates. \gls{symb:L} denotes the distance from the leftmost to the rightmost kicker unit. \gls{symb:D} denotes the sum distance of \gls{symb:d} and the $2^{nd}$ crate. d equals to 19.167 meter. L equals to 22.047m = d + 9$\cdot 0.25m + 7\cdot$ 0.09m. D equals to 20.437m = d + 4$\cdot 0.25m + 3\cdot$ 0.09m.

Fig.~\ref{kicker} (a) is the easiest scenario. The kicker units in the $1^{st}$ crate are fired when the tail of the bunch 1 passes by the $1^{st}$ crate completely. The kicker units in the $2^{nd}$ crate are fired when the tail of the bunch 1 passes by the $2^{nd}$ crate completely. The delay for the firing two crates in this scenario is D/$\beta$c. 

Fig.~\ref{kicker} (b) shows the scenario of the maximum delay between the firing of two crates. The kicker units in the $1^{st}$ crate are fired when the tail of the bunch 1 passes by the $1^{st}$ crate completely. The kicker units in the $2^{nd}$ crate are fired 90ns before the head of the bunch 2 passes by it. The delay equals to G+d/$\beta$c-90ns.

Fig.~\ref{kicker} (c) shows the scenario of the minimum delay. The kicker units in the $1^{st}$ crate are fired 90ns before the head of the bunch 2 passes by it. The kicker units in the $2^{nd}$ crate are fired when the bunch 1 passes by the $2^{nd}$ crate. The delay is L/\gls{symb:b}\gls{symb:c}-G+90ns.



Tab. ~\ref{kicker_delay} shows delay for three scenarios and related parameters. The fixed delay is determined primarily by the boundary delay range from $H^+, U^{28}$ and $U^{73+}$ beams, the delay range for other heavy ion species beams must be contained in these boundary range. According to the result, a fixed delay is available for firing kicker units in two crate for different beams. e.g. 80ns.   
\begin{table}[H]
\newcommand{\tabincell}[2]{\begin{tabular}{@{}#1@{}}#2\end{tabular}}
\caption{The delay for firing two crates of SIS18 extraction kicker}
\label{kicker_delay}
\begin{center}
    \begin{tabular}{ | c | c | c | c | c | c | c | }
    \hline
    Beam & $\beta$ &  \tabincell{c}{time\\ L/$\beta$c } &\tabincell{c}{bunch gap \\ G } & \tabincell{c}{minimum delay \\ L/$\beta$c-G+90ns} & \tabincell{c}{delay \\ D/$\beta$c} & \tabincell{c}{maximum delay \\ G+d/$\beta$c-90ns}\\ \hline
    $H^+$ & 0.982 &75ns &  184ns & 0ns & 69ns & 163ns  \\ \hline
    $U^{28+}$ &0.568 & 130ns &  159ns & 61ns &120ns & 189ns \\ \hline
    $U^{73+}$ & 0.872 & 84ns & 104ns & 70ns & 78ns & 92ns \\ \hline
    \end{tabular}
\end{center}
\end{table}

\subsection{SIS100 Injection Kicker Magnet}
Two bunches from SIS18 will be continuously injected into two rf buckets after the other in SIS100. See Fig.~\ref{kicker_SIS100}. The yellow ellipse represents the circulating bunch in SIS100 and the red one represents the bunch to be injected. The head of the bunch is at the left side. The preparation of the SIS100 injection kicker must be done during the bunch gap and it must be established for at least one SIS18 revolution period. For the instantaneous firing, all kicker units are fired only if the tail of the circulating bunch passes the leftmost kicker unit. The kicker pass time is the time needed for the tail of a bunch to pass from the rightmost unit to the leftmost kicker unit. The rise time of the kicker unit is 1/20 of the revolution period ~\cite{udo_injection_2014}. Therefor the preparation time is the sum of the kicker pass time and rise time. The distance from the rightmost to the leftmost kicker unit is 3.79m, 6$\cdot 0.22m + 5\cdot $0.23m. If the preparation time is shorter than bunch gap, all kicker units could be fired instantaneous. Tab. ~\ref{kicker_SIS100} shows the preparation time for $H^+, U^{28} and U^{73+}$ beams and their bunch gap. The preparation time is much shorter than the bunch gap. So the kicker units could be fired instantaneous. 

\begin{figure}[H]
   \centering   
   \includegraphics*[width=160mm]{kicker_SIS100.jpg}
   \caption{SIS100 injection kicker.}
   \label{kicker_SIS100}
\end{figure}

\begin{table}[H]
\newcommand{\tabincell}[2]{\begin{tabular}{@{}#1@{}}#2\end{tabular}}
\caption{The delay for firing SIS00 injection kicker}
\label{kicker_SIS100}
\begin{center}
    \begin{tabular}{ | c | c | c | c | c | c  |}
    \hline
    Beam & $\beta$ &  \tabincell{c}{kicker pass\\ time L/$\beta$c} & \tabincell{c}{Rise time \\ 1/20$\cdot T_{rev}^{SIS100}$}& \tabincell{c}{Preparation time \\ L/$\beta$c+1/20$\cdot T_{rev}^{SIS100}$} & \tabincell{c}{bunch gap \\ 2.25$\cdot T_{rev}^{SIS100}$}\\ \hline
    $H^+$     & 0.982 & 3ns  &  184ns & 187ns & 828ns   \\ \hline
    $U^{28+}$  & 0.568 & 22ns &  318ns   & 333ns  & 1431ns  \\ \hline
    $U^{73+}$ & 0.872 & 15ns &   207ns & 222ns &  932ns \\ \hline
    \end{tabular}
\end{center}
\end{table}
%%%%%%%%%%%%%%%%%%%%%%%%%%%%%%%%%%%%%%%%%%%%%%%%%%%%%%%%%%%%%%%%%%%%%%%%%%%%%%%%%%%%%%%%%%%%%%%%%%%%%%%%
\section{Test Setup for Data Collection, Merging and Redistribution}

In this section, the test setup for the FAIR B2B transfer system is described, focusing mainly on the timing aspects.  

\subsection{Functional Requirement}
The test setup must achieve the following functional requirement.
\begin{itemize}
\item[-] After receiving CMD\_B2B\_START, both the B2B source and target SCUs collect the extrapolated phase equivalent data locally. The equivalence is a timestamp of the positive zero-crossing of the simulated Reference RF Signal of the SIS18 and the SIS100. 
\item[-] The B2B target SCU transfers the frame TGM\_PHASE\_TIME containing the timestamp to the B2B source SCU.
\item[-] After receiving the data, the B2B source SCU calculates the synchronization window.
\item[-] The B2B source SCU sends the frame TGM\_SYNCH\_WIN containing the start timestamp of the synchronization window to the WR network.
\item[-] After receiving the frame, the trigger SCU produces a TTL output indicating the start of the synchronization window. 
\end{itemize}

\subsection{Test Setup}

\begin{figure}[H]
   \centering   
   \includegraphics*[width=160mm]{schematic_setup.jpg}
   \caption{Schematic of the test setup.}
   \label{setup}
\end{figure}

Fig.~\ref{setup} shows the schematic of the test setup. In this test setup, two MODEL DS345 Synthesized Function Generators\footnote{\url{http://www.thinksrs.com/downloads/PDFs/Manuals/DS345m.pdf}} are used to simulate the Reference RF Signals of the SIS18 and that of the SIS100. The DS345 of the SIS18 is synchronized to an internal \SI{10}{\MHz} clock, which works as an external reference clock for the DS345 of SIS100. The B2B source SCU, the B2B target SCU and the trigger SCU are connected to a WR switch, which connects to the timing network. A \gls{PC}\footnote{A Linux personal computer is installed with the standard TR tools and library. \newline\url{https://www-acc.gsi.de/wiki/Timing/TimingSystemNodesCurrentRelease}} is used as a DM to produce the B2B start timing frame CMD\_B2B\_START. Besides, it monitors the status of the B2B transfer programs in all SCUs. The oscilloscope is used to monitor the alignment of two simulated Reference RF Signals within the synchronization window provided by the trigger SCU.   

Fig.~\ref{testsetup_text} shows the front and back view of the test setup. The DS345 of the SIS18 produces the analog sine signal of \SI{1.572200}{\MHz} frequency for the oscilloscope and the DS345 of the SIS100 produces the analog sine signal of \SI{1.572000}{\MHz} for the oscilloscope, which are achieved by the LEMO cables (green lines in Fig.~\ref{testsetup_text}). The DS345 of the SIS18 produces an analog TTL signal for the B2B source SCU, whose rising edges are synchronized to the positive zero-crossings of the sine wave of \SI{1.572200}{\MHz} and the DS345 of the SIS100 produces an0 analog TTL signal for the B2B target SCU, whose rising edges are synchronized to the positive zero-crossings of the sine wave of \SI{1.572000}{\MHz}. The connections are achieved by the LEMO cables (red lines in Fig.~\ref{testsetup_text}). So the beating frequency is \SI{200}{\Hz} and the synchronization period is \SI{5}{\ms}. The B2B source SCU, the B2B target SCU and the trigger SCU are connected to a WR switch by the optical fiber (yellow lines in Fig.~\ref{testsetup_text}). The WR switch is connected to the PC and the WR network. The output of the synchronization window from the B2B trigger SCU is connected to the oscilloscope by the LEMO cable (green line in Fig.~\ref{testsetup_text}). 

\begin{figure}[!htb]
   \centering   
   \includegraphics*[width=160mm]{testsetup_text.jpg}
   \caption{The front and back view of the test setup.}
   \label{testsetup_text}
\end{figure}

Compared with the final scenario, there are some difference of the test setup.
\begin{itemize}
\item
The SIS18 and SIS100 DS345 will be replaced by the PAP modules, which are installed in the B2B source and target SCUs as SCU slaves. 
\item 
All devices are installed in different racks. The SIS18 source SCU and B2B trigger SCU of the extraction kicker are installed in SIS18 and the SIS18 target SCU and B2B trigger SCU of the injection kicker are installed in SIS100. The connection is done via the WR network. 
\item 
The B2B source SCU has several other SCU slaves, e.g. the Phase Shift Module (PSM) for the phase shift. 
\item 
The B2B trigger SCU considers not only the synchronization window, but also the kicker delay compensation from the SM. Besides, it has several SCU slaves, which coordinate the correct B2B extraction and injection kicker with other systems, e.g. the MPS.
\end{itemize}

\subsection{Firmware}

The B2B source, B2B target and trigger SCUs have different firmware running on their soft \gls{CPU}, LM32\footnote{LatticeMico32 is a 32-bit microprocessor soft core from Lattice Semiconductor optimized for field-programmable gate arrays (\gls{FPGA}s).}. The firmware are activated by the  B2B start timing frame, CMD\_START\_B2B, which indicates the source and target synchrotrons of the B2B transfer. 
%%%%%%%%%%%%%%%%%%%%%%%%
\begin{itemize}
\item Firmware for the B2B source SCU
\begin{figure}[!htb]
   \centering   
   \includegraphics*[width=160mm]{flow_chart_src.jpg}
   \caption{Flow chart of the firmware for B2B source SCU.}
   \caption*{Flow chart of the firmware for B2B source SCU. ``Step`` is represented as ``S`` in the figure.}
   \label{flow_chart_src}
\end{figure}

The firmware for the B2B source SCU is the core program of the B2B transfer system. See Fig. ~\ref{flow_chart_src}. 

 	\begin{itemize}
		\item[-]Step 1. The program waits for the CMD\_START\_B2B timing frame.
% 		\item[-]Step 2. When it receives the timing frame CMD\_START\_B2B, it collects the predicted phase and checks whether it is within a proper range of $0^\circ$ to $360^\circ$. If not, it sends a timing frame TGM\_B2B\_ERROR to the WR network and goes back to the step 1, which indicates the data error.
 		\item[-]Step 2. When it receives the timing frame CMD\_START\_B2B, it reads the extrapolated phase, the corresponding timestamp and the phase deviation slope from the PAP module.
		\item[-]Step 3. It waits for the TGM\_PHASE\_TIME timing frame from the B2B target SCU, which contains the extrapolated phase, the corresponding timestamp and the slope of the phase deviation.
		\item[-]Step 4. When it receives the timing frame TGM\_PHASE\_TIME within a specified timeout interval, it calculates the synchronization window, the phase shift/jump value and the phase correction value. Or it sends a timing frame TGM\_B2B\_ERROR to the WR network and goes back to the step 1, which indicates the timeout error of the frame. Besides, it checks whether the phase correction is in the range of $0^\circ$ to $360^\circ$ , the required phase shift in the range of $-180^\circ$ to $180^\circ$ and the start of the synchronization window not in the forbidden range. If at least one of them is not correct, it sends a timing frame TGM\_B2B\_ERROR to the WR network and goes back to the step 1, which indicates the calculation error. 
		\item[-]Step 5. It sends the timing frame TGM\_SYNCH\_WIN and TGM\_PHASE \_CORRECTION to the WR network. TGM\_SYNCH\_WIN indicates the start of the synchronization window and TGM\_PHASE\_CORRECTION is used for the trigger SCUs for the reproduction of the bucket label signal.
		\item[-]Step 6. It gives the phase correction and phase shift/jump values to corresponding modules.
		\item[-]Step 7. It waits for the timing frame TGM\_KICKER\_TIME\_S from the source trigger SCU and TGM\_KICKER\_TIME\_T from the target trigger SCU, which contains the extraction/injection kicker trigger and firing timestamp. When it does not receive the timing frames within a specified timeout interval, it sends a timing frame TGM\_B2B\_ERROR to the WR network and goes back to the step 1, which indicates the timeout error of the frame.
		\item[-]Step 8. When it receives the timing frames mentioned in the step 7 within a specified timeout interval, it checks the B2B transfer status and sends TGM\_B2B\_STATUS to the WR network and goes to the step 1. The B2B transfer is successful, if all of the following checks are correct. Or the B2B transfer is failure. 
\begin{itemize}
	\item Trigger time $<$ firing time of the extraction kicker of the source synchrotron

	\item Trigger time $<$ firing time of the injection kicker of the target synchrotron

	\item Firing time of the extraction kicker $<$ firing time of the injection kicker
\end{itemize}
 

	\end{itemize}
%%%%%%%%%%%%%%%%%%%%
\item Firmware for the B2B target SCU
\begin{figure}[H]
   \centering   
   \includegraphics*[width=160mm]{flow_chart_trg.jpg}
   \caption{Flow chart of the firmware for B2B target SCU.}
	\caption*{Flow chart of the firmware for B2B target SCU. ``Step`` is represented as ``S`` in the figure.}
   \label{flow_chart_trg}
\end{figure}
Fig. ~\ref{flow_chart_trg} (a) shows the flow chart of the program of the B2B target SCU.
 	\begin{itemize}
		\item[-]Step 1. The program waits for the CMD\_START\_B2B timing frame.
 		\item[-]Step 2. When it receives the timing frame CMD\_START\_B2B, it collects the predicted phase.
		\item[-]Step 3. It sends the TGM\_PHASE\_TIME timing frame to the B2B source SCU and goes back to the step 1.
	\end{itemize}
%%%%%%%%%%%%%%%%%%%%%
\item Firmware for the trigger SCU

Fig. ~\ref{flow_chart_trg} (b) shows the flow chart of the program of the source trigger SCU. For the target trigger SCU, the flow chat is same only with the different name of the timing frame TGM\_KICKER\_TIME\_T.
 	\begin{itemize}
		\item[-]Step 1. The program waits for the CMD\_START\_B2B timing frame.
		\item[-]Step 2. The program waits for the TGM\_PHASE\_CORRECTION timing frame.
		\item[-]Step 3. The program gives the phase correction value to the corresponding module for the bucket label signal reproduction.
 		\item[-]Step 4. When it receives the timing frame CMD\_START\_B2B, it waits for the timing frame TGM\_SYNCH\_WIN to indicate the synchronization window for the kicker trigger.
		\item[-]Step 5. After the beam extraction, it collects the trigger and firing timestamp. 
		\item[-]Step 6. It sends the TGM\_KICKER\_TIME\_S timing frame to the B2B source SCU and goes back to the step 1.
	\end{itemize}

\end{itemize}
%%%%%%%%%%%%%%%%%%%%%%
\subsection{Time Constraints}
For the B2B transfer system, the time constraints are very important and strict. Fig. ~\ref{time_constraint} shows the time constraint of the system. The CMD\_START\_B2B is executed at \gls{symb:t_b2b}. The RF phase prediction needs \SI{500}{\us}, so the B2B source and target SCUs collect the phase data at $t_{B2B}$ + \SI{500}{\us} and need about \SI{450}{\ns} for the data collection. The B2B source SCU receives the timing frame TGM\_PHASE\_TIME at around $t_{B2B}$ + \SI{500}{\us} + \SI{450}{\ns} + \SI{500}{\us} $\approx$ $t_{B2B}$ + \SI{1}{\ms}. The second \SI{500}{\us} is the upper bound latency of the WR network. After that, the B2B source SCU needs about \SI{100}{\us} for the calculation, the sending of the timing frame TGM\_SYNCH\_WIN and TGM\_PHASE\_CORRECTION and data transferring to the corresponding module. TGM\_SYNCH\_WIN is sent at around $t_{B2B}$ + \SI{1}{\ms} + \SI{100}{\us} $\approx$ $t_{B2B}$ + \SI{1.1}{\ms}. The trigger SCU receives TGM\_PHASE\_CORRECTION and TGM\_SYNCH\_WIN at around $t_{B2B}$ + \SI{1.1}{\ms} + \SI{500}{\us} $\approx$ $t_{B2B}$ + \SI{1.6}{\ms}. The \SI{500}{\us} is the latency of the WR network. The start of the synchronization window must be later than $t_{B2B}$ + \SI{1.1}{\ms} + 2$\cdot$\SI{500}{\us} $\approx$ $t_{B2B}$ + \SI{2.1}{\ms}, because the TGM\_SYNCH\_WIN must be transferred back to the DM and the DM transfers it further to the beam instrumentation devices via WR network. The upward to DM transfer needs maximum \SI{500}{\us} and the transfer from the DM to BI needs another \SI{500}{\us}.  The upper bound B2B transfer time is \SI{10}{\ms}, which is decided by the duration of the stable beam. There is no hard real time for the collection of the trigger and firing timestamps and timing frame TGM\_KICKER\_TIME\_S sending, we give \SI{1}{\ms} for the source trigger SCU to do this task and the source trigger SCU sends TGM\_KICKER\_TIME\_S at around $t_{B2B}$ + \SI{10}{\ms} + \SI{1}{\ms} $\approx$ $t_{B2B}$ + \SI{11}{\ms}. The same time constraints is also for the target trigger SCU. The B2B source SCU receives TGM\_KICKER\_TIME\_S and TGM\_KICKER\_TIME\_T at around $t_{B2B}$ + \SI{11}{\ms} + \SI{500}{\us} $\approx$ $t_{B2B}$ + \SI{11.5}{\ms}. The \SI{500}{\us} is the latency of the WR network. The B2B source SCU sends TGM\_B2B\_STATUS at around $t_{B2B}$ + \SI{11.5}{\ms} + \SI{100}{\us} $\approx$ $t_{B2B}$ + \SI{11.6}{\ms}. The BI devices receives the timing frame TGM\_B2B\_STATUS at around $t_{B2B}$ + \SI{11.6}{\ms} + 2$\cdot$\SI{500}{\us} $\approx$ $t_{B2B}$ + \SI{12.6}{\ms}.

\begin{landscape}
\begin{figure}[!htb]
   \centering   
   \includegraphics*[width=200mm]{flow_chart_time.jpg}
   \caption{The time constraints of the B2B transfer system.}
   \caption*{The sent and received timing frame pairs have the same color. The test setup realizes the steps in the blue rectangle. (not drawn to accurate timescale) }
   \label{time_constraint}
\end{figure}
\end{landscape}

\subsection{Test Result}
Because some modules of the B2B transfer system are still under the development, the test setup realizes parts of the whole function, mainly concentrated on the data collection from two simulated Reference RF signals, the calculation of the synchronization window and the distribution of the start of the synchronization window. The steps with the blue rectangle in Fig.~\ref{time_constraint} are realized in this test setup. The test result of the B2B programs on B2B source, B2B target and trigger SCUs are shown as follows. 

\begin{lstlisting}[language={[ANSI]C}, keywordstyle=\color{blue!70}, commentstyle=\color{red!50!green!50!blue!50}, frame=shadowbox, rulesepcolor=\color{red!20!green!20!blue!20}]

U28+ B2B transfer from SIS18 to SIS100 => Source B2B SCU
=============================================
SIS18: Frequency of the Reference RF Signal = 1.572200MHz
SIS100: Frequency of the Reference RF Signal = 1.572000MHz 
SIS18: Period of the Reference RF Signal = 636051(ps)
SIS100: Period of the Reference RF Signal = 636132(ps)

>>>>>>>>>>>>>>>>>>>>>>>>> Receive CMD_START_B2B from WR network
Timestamp of the Reference RF Signal from SIS18 (accuracy to 1ns)
GMT: Thu, Jan 8, 1970, 21:07:27.445405856

>>>>>>>>>>>>>>>>>>>>>>>>> Receive TGM_PHASE_TIME from WR network
Timestamp of the Reference RF Signal from SIS100 (accuracy to 1ns)
GMT: Thu, Jan 8, 1970, 21:07:27.445364560

Beating time: 5 (ms)
Synchronization time: 4.622818 (ms)
The number of the SIS18 revolution for the synchronization: 3634
Start of the synchronization window: GMT: Thu, Jan 8, 1970, 21:07:27.450028674

<<<<<<<<<<<<<<<<<<<<<<<<< Send TGM_SYNCH_WIN to WR network
\end{lstlisting}

\begin{lstlisting}[language={[ANSI]C}, keywordstyle=\color{blue!70}, commentstyle=\color{red!50!green!50!blue!50}, frame=shadowbox, rulesepcolor=\color{red!20!green!20!blue!20}]

U28+ B2B transfer from SIS18 to SIS100 => Target B2B SCU
=============================================
>>>>>>>>>>>>>>>>>>>>>>>>> Receive CMD_START_B2B from WR network
Timestamp of the Reference RF Signal from SIS100 (accuracy to 1ns)
GMT: Thu, Jan 8, 1970, 21:07:27.445364560

<<<<<<<<<<<<<<<<<<<<<<<<< Send TGM_PHASE_TIME to WR network
\end{lstlisting}

\begin{lstlisting}[language={[ANSI]C}, keywordstyle=\color{blue!70}, commentstyle=\color{red!50!green!50!blue!50}, frame=shadowbox, rulesepcolor=\color{red!20!green!20!blue!20}]

U28+ B2B transfer from SIS18 to SIS100 => Trigger SCU
=============================================
Waiting for timing frames...
>>>>>>>>>>>>>>>>>>>>>>>>> Receive TGM_SYNCH_WIN from WR network
Event execution timestamp: GMT 1970-01-08 21:07:27.450028674
\end{lstlisting}

After both B2B source and target programs receive the CMD\_START\_B2B frame, they trigger another unit connected to the System-on-Chip\footnote{A system-on-chip is an integrated circuit that integrates all components of a computer or other electronic system into a single chip.}  (SoC) bus to get the timestamp of the next zero crossing point of the DS345 sine waves, which is simulated as an equivalent to the predicted phase. All timestamp are shown in the format of Greenwich Mean Time (GMT). The timestamp got by the B2B source SCU is Thu, Jan 8, 1970, 21:07:27 0.445405856 second and the timestamp got by the B2B target SCU is Thu, Jan 8, 1970, 21:07:27 0.445364560 second, see Line 10 and 14 of the test result of the B2B source SCU. The time difference between two timestamps is \SI{41.296}{\us}. The frequency difference between SIS18 and SIS100 Reference RF Signals is \SI{200}{Hz}. It means that there are 200 more periods of the SIS18 Reference RF Signal within one second compared with the SIS100 Reference RF Signal. Every \SI{5}{ms} (1/\SI{200}{Hz}) SIS18 Reference RF Signal has one period more than that of SIS100. The time is calculated by eq. ~\ref {syn_time}, indicating the alignment of the zero crossing of two DS345 sine waves of SIS18 and SIS100. The time is named as ``synchronization time``, denoted by $\Delta t$.

\begin{equation}
\begin{aligned}
\frac{T^{SIS18}_{h=2}}{1/(f^{SIS18}_{h=2}-f^{SIS100}_{h=10})}=\frac{41.296us\mod T^{SIS100}_{h=10}}{\Delta t}
\label {syn_time}
\end{aligned}
\end{equation}

\begin{equation}
\Delta t = \SI{4.622818}{\ms}
\end{equation}

The number of the SIS18 Reference RF Signal periods for the synchronization is calculated as
\begin{equation}
\frac{\Delta t}{T^{SIS18}_{h=2}}=7268
\end{equation}
we could get that the beating time \gls{symb:d_t} is \SI{4.622818}{\ms} and the number of the SIS18 Reference RF Signal periods for the synchronization is 7268 for the test.

%After both B2B source and target programs receive the $CMD\_START\_B2B$ frame, they trigger another unit connected to the System-on-Chip\footnote{A system-on-chip is an integrated circuit that integrates all components of a computer or other electronic system into a single chip.}  (SoC) bus to get the timestamp of the next zero crossing point of the DS345 sine waves, which is simulated as an equivalent to the predicted phase. The triggers of the B2B source and target SCUs are not simultaneous, namely the B2B source and target SCU do not get the timestamp of the adjacent zero crossing points of two RF simulated sine signals, see Line 10 and 14 of the test result of the B2B source SCU. All timestamp are shown in the format of Greenwich Mean Time (GMT). The timestamp got by the B2B source SCU is Thu, Jan 8, 1970, 21:07:27 0.445405856 second and the timestamp got by the B2B target SCU is Thu, Jan 8, 1970, 21:07:27 0.445364560 second. The time difference between two timestamps is \SI{41.296}{\us}. There are two reasons for the asynchronous triggers.
%
%\begin{itemize}
%	\item
%The SoC bus might be granted to other program and B2B program must wait until it is free.
%	\item
%The behaviour of the user friendly messages of the LM32 programs causes the non real time of the programs.
%\end{itemize}
%
%The difference between timestamps of the adjacent zero crossing points, 592ns, is the remainder resulting from 41.296us dividing SIS18 revolution period \SI{636051}{\ps}. Based on eq. ~\ref{syn_time} and eq. ~\ref{syn_num}, 
%\begin{equation}
%\begin{aligned}
%\frac{T^{SIS18}_{h=2}}{5ms}=\frac{592ns}{\Delta t}
%\label {syn_time}
%\end{aligned}
%\end{equation}
%
%\begin{equation}
%\begin{aligned}
%\frac{\Delta t}{T^{SIS18}_{h=1}}=3634
%\label {syn_num}
%\end{aligned}
%\end{equation}
%we could get that the beating time \gls{symb:d_t} is \SI{4.622818}{\ms} and the number of the SIS18 revolution period is 3634 for the test. 
%
%For the real application of the B2B transfer system, in order to guarantee the time constraints of the B2B programs, see Fig. ~\ref{time_constraint}, the B2B source, target and trigger SCUs run only their corresponding B2B program. The SoC bus is occupied only by the B2B program. Besides, the programs running on LM32 are forbidden to print out any user friendly messages.



