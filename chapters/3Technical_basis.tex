
For the FAIR accelerator complex, synchronization of the B2B transfer will be realized by the FAIR control system and the Low-Level RF (LLRF) system. For the synchronization of LLRF system, the General Machine Timing (\gls{GMT}) system is complemented and linked to the Bunchphase Timing System (BuTiS). Machine Protection System (MPS) protects SIS100 and subsequent accelerators or experiments from damage. Hence, the B2B transfer system for FAIR coordinates with the MPS system. 
\section{FAIR control system}
The \gls{FAIR} control system takes advantage of collaborations with CERN in using proven framework solutions like Front-End System Architecture (\gls{FESA})~\cite{hoffmann_fesafront-end_2008}, LHC Software Architecture (\gls{LSA}), White Rabbit (\gls{WR}), etc ~\cite{huhmann_fair_2013}. It consists of the equipment layer, middle layer and application layer. The equipment layer consists of equipment interfaces, GMT and software representations of the equipment FESA. The middle layer provides service functionality both to the equipment layer and the application layer through the IP control system network. LSA is used for the Settings Management (SM). The application layer combines the applications for operators as \gls{GUI} applications or command line tools. The application layer and the middle layer only request what the FAIR accelerator complex should do and transmit set values to the equipment layer. The actual beam production is controlled by the GMT. The GMT system is synchronized to BuTiS. The \gls{SM} supplies the schedule for the GMT by LSA ~\cite{huhmann_fair_2013, beck_new_2012}.

\subsection{BuTiS}
Bunch Phase Timing System (BuTiS) serves as a campus-wide clocks distribution system with sub nanosecond resolution and stability over distances of several hundred meters while maintaining 100ps per km timing stability ~\cite{moritz_butisdevelopment_2006}. Two BuTiS reference clocks 10 MHz and C2 200 MHz and a trigger identification pulse T0 at 100 kHz are generated centrally in the BuTiS center. A star-shaped optical fiber distribution network transfers these signals to BuTiS receivers all over the FAIR campus. A BuTiS receiver and a local reference synthesizer are installed in each supply room to produce the BuTiS reference clocks, which are in phase at every BuTiS receiver. For this purpose, a measurement setup in the BuTiS center continuously measures the optical signal transmission delay between the BuTiS center and the different BuTiS receivers. This measurement information is used to shift the phases of the signals generated in each local reference synthesizer for the delay compensation. The main task of BuTiS is the supply of the reference clock signals for \gls{glos:Rrf} in each rf supply room ~\cite{moritz_butisdevelopment_2006, zipfel_recent_2011}.

\subsection{GMT}
The GMT is contained in the equipment layer. The main tasks of the GMT system are time synchronization of more than 2000 Front-End Controllers (\gls{FEC}) with nanosecond accuracy, distribution of timing messages and subsequent generation of real-time actions by the nodes of the timing system ~\cite{beck_new_2012}. The GMT consists of the Timing Master (\gls{TM}), the White Rabbit (WR) timing network and integrated nodes. The timing master's interface to the upper layers, e.g. online schedule monitor, is modeled as a FESA device. The timing master is a logical device, containing the data master (\gls{DM}), the clock master (\gls{CM}) and the management master (\gls{MM}). The data master receives a schedule for the operation of the FAIR accelerator complex from the Settings Management and provides the real-time scheduler by broadcasting timing messages to the WR timing network, which will be received and executed by the corresponding node at the designated time. The clock master is a dedicated White Rabbit switch. It is the topmost switch layer of the WR timing network and provides the grandmaster clock and timestamps which are distributed to all other nodes in the timing network. The clock master derives its clock from the BuTiS clocks and timestamps distributed are phase locked to BuTiS clocks. The GMT could deliver BuTiS T0 and C2 clocks to any nodes and nodes are capable to timestamp clock edges. All active components including receiver nodes and switches are registered to the MM. The MM monitors and manages the active components of the GMT system ~\cite{beck_general_????}.

The DM sends timing messages\footnote{\url{https://www-acc.gsi.de/wiki/Timing/TimingSystemEvent}} to nodes, which contains Event ID, Format ID, Group ID, Event No, Sequence ID, Beam Process ID, parameter, timestamp and so on ~\cite{beck_timing_2015}. A timing message is sent across the WR network, so it must be contained in the Ethernet frame. An Ethernet frame including one timing message has a length of \SI{110}{byte}, which is called the timing frame in this dissertation.

%Fig. ~\ref{Timing_message} shows the format of the timing message . 
%\begin{figure}[H]
%   \centering   
%   \includegraphics*[width=160mm]{Timing_message.jpg}
%   \caption{The format of the timing message.}{~\cite{beck_timing_2015}}
%   \label{Timing_message}
%\end{figure}
%The timing message contains 
%\begin{itemize}
%	\item WB Addr (\SI{32}{bit}): Wishbone address to which on the node the data shall be written.
%	\item Payload (\SI{256}{bit})
%		\begin{itemize}
%			\item EventID (\SI{64}{bit}): Index of the schedule step.
%		\begin{itemize}
%			\item Format ID (FID) (\SI{4}{bit}): Serves to distinguish between different formats of the timing message.
%			\item Group ID (GID)(\SI{12}{bit}): Identifies a group of equipment, such as a synchrotron or a transfer line.
%			\item Event No (EVTNO) (\SI{12}{bit}): Specifies a command to be executed.
%			\item Sequence ID (SID) (\SI{12}{bit}): A sequence is analogous to the concept of a ``virtual accelerator``. 
%			\item Beam Process ID (BPID) (\SI{14}{bit}): A beam process defines a process which must not be interrupted, e.g. a acceleration ramp. 
%			\item Reserved (\SI{10}{bit})
%		\end{itemize}
%			\item Param (\SI{64}{bit}): An additional parameter with event specific meaning.
%			\item TEF (\SI{32}{bit}): Timing Extension Field containing fine delay information and other data.
%			\item Reserved (\SI{32}{bit}): 
%			\item Timestamp (\SI{64}{bit}): In units of \SI{8}{\ns} clock cycles since 1 January 1970.
%		\end{itemize}
%\end{itemize}
%A timing message is sent across the WR network, so it must be contained in the ethernet frame. An ethernet frame including one timing message has a length of \SI{110}{byte}, which is called the timing frame in this document.

\subsection{FESA}
The \gls{FESA}\footnote{\url{https://www-acc.gsi.de/wiki/FESA/WhatIsFESA}} is a framework used to fully integrate the large amount of front-end equipments into the accelerator control system. FESA was developed by CERN and has already been implemented into the \gls{CERN} control system. Now it is developed further in collaboration with GSI for the FAIR project. FESA develops FESA classes, the equipment-type specific front-end software ~\cite{hoffmann_fesafront-end_2008}. For a specific type of equipments, a FESA class implementation accesses to the control interface of the equipments. The FESA class models the equipment as device, so the FESA output is called device class. One device class can instantiate several devices and thus generally handles several independent pieces of equipments.  FESA provides JAVA based graphical user interfaces (GUI) to design, deploy, instantiate and test the device classes. For the FAIR project the necessary interaction with the timing nodes is realized in a lab-specific timing library of the FESA framework.
The FEC use FESA to implement generic and equipment specific functions in form of the device classes. Interaction with the equipment is synchronized with the GMT system. 

For time multiplexed operation of the accelerators, the FESA framework supports defining multiplexed properties. Before an accelerator schedule is started, the setting properties of FESA classes are pre-supplied by LSA from SM for all scheduled beams with specific settings accordingly. At runtime, FESA’s real time software actions are triggered by timing message, the actual beam specific data is then selected based on information carried by the timing message and send to the equipment. 

\subsection{Settings Management}
The Settings Management (\gls{SM}) is based on a physics model for accelerator optics, parameter space and overall relations between parameters and between accelerators, which is located in the middle layer of the control system. It supports off-line generation of synchrotron settings, sending these settings to all involved devices, and programming the schedule of the timing system ~\cite{huhmann_fair_2013}. The SM uses the LSA framework. A standardized \gls{API} allows accessing data in a common way as basis for generic client applications for all accelerators. Using the LSA-API, trim-applications can coherently modify synchrotron settings ~\cite{huhmann_fair_2013}. E.g. the service generates timing constraints (e.g. ramp curve) as well as the equipment’s data settings (e.g. field) for all devices derived from physics parameters (e.g. beam energy). For FAIR the framework is extended to model the overall schedule of all accelerators. Beams are described as Beam Production Chains to allow a description from beam source to beam target for settings organization and data correlation.

\section{LLRF system}
The FAIR low-level rf (\gls{LLRF}) system will be used in the existing synchrotrons SIS18 and \gls{ESR}, as well as in the FAIR synchrotrons SIS100 and SIS300 and in \gls{CR}, \gls{NESR}, and accumulator ring (\gls{RESR}). It supports fast ramp rates and large frequency span for the acceleration of a variety of ion species, It supports different RF manipulations, including operation at different harmonic numbers, barrier bucket generation, bunch compression and longitudinal feedback. ~\cite{klingbeil_new_2011}. 

RF cavities are driven by one of Reference RF Signals, which are supplied in every supply room . Fig.~\ref{local_cavity_syn} shows the local cavity synchronization system, which synchronizes the local Cavity Direct Digital Synthesizer (DDS) unit to the Reference RF Signal. The cavity gets the RF signal from a local Cavity \gls{DDS} unit, which receives RF Frequency Ramps from the Central Control System (\gls{CCS}). A Digital Signal Processor (\gls{DSP})-System measures the phase difference between the Reference RF Signal and the gap voltage of the cavity. In the DSP system, a closed-loop control algorithm is implemented, which generates frequency corrections for the local \gls{glos:cavity_DDS} unit. This process is called local synchronization loop, which ensures that the phase of the gap voltage follows the phase of the Reference RF signal ~\cite{klingbeil_new_2011}. 
\begin{figure}[H]
   \centering   
   \includegraphics*[width=110mm]{local_cavity_syn.png}
   \caption{Local Cavity Synchronization}{~\cite{klingbeil_new_2011}}
   \label{local_cavity_syn}
\end{figure}

In order to damp coherent longitudinal rigid dipole oscillations, the beam phase control loop is used. The phase difference between the beam signal and the Reference RF Signal is fed back via an FIR filter. The beam signal is obtained by a fast current transformer or a beam position monitor. The filter output is converted in a phase-correction and forwarded to the Group DDS. The corrections are added to the phase of the  frequency ramp in the Cavity DDS, which results in a change of the phase of the gap voltage and thus a feedback to the beam ~\cite{baudrenghien_low-level_2010}. Unfortunately, the actual beam phase control loop in SIS18 is not able to damp incoherent longitudinal rigid dipole oscillations. For SIS100, a bunch-by-bunch longitudinal rf feedback loop will be developed. The bunch-by-bunch longitudinal rf feedback loop generates a correction voltage in dedicated feedback cavities for a specified bunch ~\cite{gross_bunch-by-bunch_2015}. 



\begin{figure}[!htb]
   \centering   
   \includegraphics*[width=160mm]{ref_rf_dis.png}
   \caption{Reference RF Signal distribution system}{~\cite{klingbeil_new_2011}}
   \label{ref_rf_dis}
\end{figure}
Each supply room has a Reference RF Signal distribution system shown in Fig.~\ref{ref_rf_dis}. The Reference RF Signals in different supply rooms are synchronized by BuTiS. BuTiS 200MHz C2 and 100kHz T0 clock signals are generated by BuTiS receivers in different supply rooms in phase. In Fig.~\ref{ref_rf_dis}, a number of Group DDS units are located in each supply room, which are synchronized by BuTiS local reference. The Group DDS signals can be routed to the different cavity systems by a Switch Matrix. All cavities in a synchrotron could be providing with the same Group DDS signal. The cavities at different harmonic numbers could be realized by using Group DDS signals with different harmonic numbers and by adjusting the harmonic number at the Cavity DDS accordingly. The Group DDS concept allows to synchronize a variety of cavities in a very flexible way ~\cite{klingbeil_new_2011}.  The path from the Group DDS 1 to Cavity 1 marked with the red line in Fig.~\ref{ref_rf_dis} is realized by the local cavity synchroniyation in Fig.~\ref{local_cavity_syn}.  The \gls{glos:vit_DDS} is a virtual position around the ring, to which the Reference RF Signal corresponds.

All the cavities of SIS18 are driven from one supply room. The SIS100 cavities will be gathered in five acceleration sections, each of them is driven by a dedicated supply room. 

\section{\gls{MPS} system}
The Fast Beam Abort System (\gls{FBAS}) protects SIS100 and subsequent accelerators or experiments from damage by mislead beams, uncontrolled radioactive contamination and so on. Thereby, the individual equipment is assumed self-protecting, which could triggers accelerator safety critical actions, such as an emergency beam dump, a shutdown of magnets or a beam injection inhibit. In case of relevant equipment failures or other inappropriate equipment states, a MPS signal stems from this equipment. MPS propagates these signals by a proprietary net and combines them with the machine state or operator interventions. These signals must be processed to ensure the timeliness of the triggered actions. The entering of a safe equipment state is subordinate to a FBAS action ~\cite{mandakovic_fair_????}.


\section{Comparison}

Based on the FAIR existing infrastructures, the B2B transfer system for FAIR is unique from other existing B2B transfer systems (~\cite{ferrand_synchronization_2015, ezura_beam-dynamics_2008}). The uniqueness is the phase difference between two rf systems of the source and target synchrotrons are achieved based on the campus distributed reference signals with picosecond precision, which are in phase and have same frequency. The campus distributed reference signals are synchronized with BuTiS T0 clock. They are named Synchronization Reference Signal in this document. The phase difference between the Synchronization Reference Signal and the rf signal of the rf system is measured locally and transferred from one synchrotron to another via the WR network. The existing B2B transfer system for CERN measures the phase difference between two synchrotrons by the direct cable transfer of the rf signal of one synchrotron to another synchrotron ~\cite{ferrand_synchronization_2015}. The phase measurement between two rf systems of the B2B transfer system for FAIR is more stable and precise, which is less influenced by the external environment, e.g. temperature influence on the direct cable connection. It does not constraint by the distance between two synchrotrons.

Besides, the B2B transfer system for FAIR is more flexible, which supports various complex beam transfer for FAIR.  It is capable to transfer different species beam from one \gls{glos:machine_cycle} to another.  It is capable to parallel transfer beam through FAIR accelerators. It is capable to transfer the beam between two synchrotrons via FRS or Super FRS. 

It coordinates with the MPS system, which protects SIS100/SIS300 from unacceptable failure or situation. E.g. beam position is out of tolerance, rf cavity failure and so on. If the inhibit signal from MPS is off, the B2B transfer extraction and injection kickers will be fired. If the inhibit signal is on, the injection and extraction kickers will be blocked for firing.  When the emergency signal from MPS is indicated, the beam is capable to be kicked to the beam dump at any time during the B2B transfer process.


%\section{$U^{28+}$ beam from SIS18 to SIS100}
%
%In this document, we use $U^{28+}$ B2B transfer from SIS18 to SIS100 as an example. So the supercycle of $U^{28+}$ beam of SIS18 and stacking of $U^{28+}$ beam of SIS100 are introduced in this section. 
%\begin{figure}[!htb]
%   \centering   
%   \includegraphics*[width=160mm]{SIS18-100-U28.jpg}
%   \caption{$U^{28+}$ beam from SIS18 to SIS100}
%   \label{SIS18-100-U28}
%\end{figure}
%
%In Fig.~\ref{ref_rf_dis}, SIS100 is operated at harmonic number 10, it holds 10 buckets in total, indicated by the row of ellipses in the lower part of the figure. SIS18 is operated at harmonic number 2. The beam is accumulated using four consecutive injections of two bunches each from SIS18 into different buckets. These four consecutive cycles are called super cycle. When the injection from SIS18 is completed, 8 neighbouring buckets in SIS100 are filled. The bucket pattern is defined as the rules of the bucket filling. After that the complete beam of SIS100 is compressed in a single bunch at harmonic number 2.

%%%%%%%%%%%%%%%%%%%%%%%%%%%%%%%%%%%%%%%%%
%\bibliography{main}
%\bibliographystyle{plain}


