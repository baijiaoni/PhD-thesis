Due to the ratio of the circumference of the injection/extraction orbit and the ratio of the harmonic number of the large synchrotron to that of the small synchrotron, there are several user cases of the B2B transfer system for FAIR, see Tab.~\ref{B2B_cases}. In this document, the circumference of the designed orbit of the synchrotron is denoted by C, the revolution frequency and rf cavity frequency by $f_{rev}$ and $f_{rf}$, the beating frequency by \gls{symb:beating_freq}, the RF by $f_{rf}$ and the harmonic number by h. The superscript l or s denotes the large or small synchrotron. $\kappa$ is used to represent integers and $\lambda$ the decimal numbers. Here we define the circumference ratio as the ratio of the circumference of the designed orbit and the harmonic ratio the ratio of the harmonic number of the large synchrotron to that of the small synchrotron.

\begin{table}[!htb]
\newcommand{\tabincell}[2]{\begin{tabular}{@{}#1@{}}#2\end{tabular}}
\caption{FAIR user cases of the B2B transfer system}
\label{B2B_cases}
\begin{center}
    \begin{tabular}{ | c | c | c | c | c | c |}
    \hline
	Circumference ratio type & $C^l/C^s$ & Harmonic ratio type & $h^l$ & $h^s$ & User case of FAIR accelerators\\ \hline
     	\multirow{3}*{{\tabincell{c}{$C^l/C^s=\kappa$ \\Integer}}} & 5 & $h^l/h^s=\kappa$ & 10 & 2 & $U^{28+}$ B2B transfer from the SIS18 to the SIS100 \\ \cline{2-6}
& 5& $h^l/h^s\neq\kappa$ & 10 & 1 & $H^{+}$ B2B transfer from the SIS18 to the SIS100 \\ \cline{2-6}
& 2& $h^l/h^s\neq\kappa$ & 1 & 1 & B2B transfer from ESR to CRYRING \\ \hline
     	\multirow{4}*{\tabincell{c}{$C^l/C^s=\kappa\pm \lambda$ \\Not Integer}}& &$h^l/h^s=\kappa$ &&& B2B transfer from the SIS18 to the ESR \\ \cline{2-6}
&2-0.003 & $h^l/h^s\neq\kappa$& 4&1 & Heavy ion B2B transfer from SIS18 to ESR \\ \cline{2-6}
&2-0.003 & $h^l/h^s\neq\kappa$& 1&1 & $H^{+}$/Heavy ion B2B transfer from SIS18 to ESR \\ \cline{2-6}
&5-0.107 & $h^l/h^s\neq\kappa$& 4&1 & Heavy ion B2B transfer from SIS18 to ESR \\ \hline
    \end{tabular}
\end{center}
\end{table}

%%%%%%%%%%%%%%%%%% Circumference Integer %%%%%%%%%%%%%%%%%%%%%%%%%%%%%%%%
\section{ Circumference ratio is an ideal integer}
If the Ratio of the circumference of the designed orbit of the large synchrotron to that of the small synchrotron is an ideal integer, we have the following relation. 
\begin{equation}
\frac{C^l}{C^s}=\kappa \label{circumference_ratio_int}
\end{equation}
From the circumference ratio, the revolution frequency ratio of two synchrotrons can be calculated.
\begin{equation}
\frac{f_{rev}^{l}}{f_{rev}^{s}}=\frac{1}{\kappa} \label{rev_freq_ratio_int}
\end{equation}
Based on eq.~\ref{rev_freq_ratio_int} and harmonic number, the $f_{rf}$ are calculated by eq.~\ref{rf_freq_s_int} and eq.~\ref{rf_freq_l_int}
\begin{equation} 
f_{rf}^{s}= h^s \times f_{rev}^{s}=h^s \times \kappa \times f_{rev}^{l} \label{rf_freq_s_int}
\end{equation}
\begin{equation} 
f_{rf}^{l}= h^l \times f_{rev}^{l} \label{rf_freq_l_int}
\end{equation}
%%%%%%%%%%%%%%%%%% Harmonic = circumference ratio Integer %%%%%%%%%%%%%%%%%%%%%%%%%%%%%%%%
\subsection{Harmonic ratio equals to the circumference ratio}
When the ratio of the harmonic number of the large synchrotron to that of the small synchrotron equals to the circumference ratio, we have the following relation.
\begin{equation}
\frac {h^{l}}{h^{s}}=\frac {C^{l}}{C^{s}}= \kappa  \label{harmonic_1_int}
\end{equation}
Substituting eq.~\ref{harmonic_1_int} into eq.~\ref{rf_freq_l_int}, the following relation is deduced. 
\begin{equation}
f_{rf}^{s}= f_{rf}^{l}\label{equ_rf_freq}
\end{equation}

In this scenario, the rf cavity frequencies of two synchrotrons are same, which is the user case of the $U^{28+}$ B2B transfer from the SIS18 to the SIS100. Four batches of $U^{28+}$ at \SI{200}{meV/\atomicmassunit} are injected into continous eight out of ten buckets of SIS100. Each batch consists of two bunches. We know $\kappa=5$, $h^l=10$ and $h^s=2$, so the cavity rf frequencies of two synchrotrons comply with eq.~\ref{equ_rf_freq}.

For the RF synchronization, both phase shift and frequency beating methods are applicable for the small or large or both synchrotrons. When the target synchrotron is empty, the phase will be shifted for the target synchrotron by the phase jump. After the synchronization, the phase difference between the SIS18 and SIS100 revolution frequency markers equals to the sum of $t_{src}$, $t_{trg}$ and TOF. When the 1st and 2nd buckets are to be filled, $d_{pattern}$=0. When the 3rd and 4th buckets, $d_{pattern}$=one SIS18 revolution period. When the 5th and 6th buckets, $d_{pattern}$= 2 $\times$ one SIS18 revolution period. When the 7th and 8th buckets, $d_{pattern}$= 3 $\times$ one SIS18 revolution period. With the phase shift method, we have an infinite synchronization window in the ideal situation, within which each bunch could be injected into its corresponding bucket at any time. 

%%%%%%%%%%%%%%%%%% Harmonic != circumference ratio Integer %%%%%%%%%%%%%%%%%%%%%%%%%%%%%%%%
\subsection{Harmonic ratio does not equal to the circumference ratio} 
When the ratio of the harmonic number of the large synchrotron to that of the small synchrotron does not equal to the circumference ratio, we have the following relation.
\begin{equation}
\frac {h^{l}}{h^{s}}\neq \frac {C^{l}}{C^{s}}= \kappa  \label{harmonic_1_noint}
\end{equation}
We assume 
\begin{equation}
\frac {h^{l}}{h^{s} \times \kappa}= \frac {m}{n}  \label{number_noint}
\end{equation}
where m and n are used to represent integers.

Eq.~\ref{rf_freq_l_int} divides eq.~\ref{rf_freq_s_int}, we get
\begin{equation}
\frac{f_{rf}^{l}}{f_{rf}^{s}}= \frac{h^l}{h^s \times \kappa} \label{freq_divide}
\end{equation}

Substituting eq.~\ref{number_noint} into eq.~\ref{freq_divide}, the following relation is deduced. 
\begin{equation}
\frac{f_{rf}^{l}}{f_{rf}^{s}}= \frac{m}{n}
\end{equation}

In this scenario, two synchrotrons have different rf cavity frequencies, which is the user case of the $H^{+}$ B2B transfer from the SIS18 to the SIS100. Four batches of $U^{28+}$ at \SI{4}{GeV/\atomicmassunit} are injected into continous four out of ten buckets of SIS100. Each batch consists of one bunch. We know $\kappa=5$, $h^l=10$ and $h^s=1$, so 
\begin{equation}
\frac{f_{rf}^{l}}{f_{rf}^{s}}= \frac {h^{l}}{h^{s} \times \kappa}= \frac{10}{1 \times 5}=\frac{2}{1}=\frac {m}{n}
\end{equation}

For the RF synchronization, both phase shift and frequency beating methods are applicable. In order to inject into the odd and even number buckets, there are two scenarios of the phase difference between the SIS18 and SIS100 revolution frequency markers after the synchronization.
\begin{itemize}
	\item Injection into the odd number buckets
		
		The phase difference between the SIS18 and SIS100 revolution frequency markers equals to the sum of $t_{src}$, $t_{trg}$ and TOF. When the 1st bucket is to be filled, $d_{pattern}$=0. When the 3rd bucket is to be filled, $d_{pattern}$=2 $\times$ SIS100 revolution period. 
	\item Injection into the even number buckets
	
		The phase difference between the SIS18 and SIS100 revolution frequency markers equals to $t_{src}$+$t_{trg}$+TOF- $T_{rf}^{100}$. When the 2nd bucket is to be filled, $d_{pattern}$=One SIS100 revolution period. When the 4th bucket is to be filled, $d_{pattern}$=3 $\times$ SIS100 revolution period. 

\end{itemize}
With the phase shift method, we have an infinite synchronization window. 

For the frequency beating method, two slightly different frequencies are $f_{rf}^{s}/n+\Delta f$ and $f_{rf}^{s}/m$, where $\Delta f$ is the beating frequency. Because m > n, it is easier to detune the RF Reference Signal of the small synchrotron (SIS18) by $n\times\Delta f$ than that of the large synchrotron (SIS100) by $m\times\Delta f$.

%%%%%%%%%%%%%%%%%% Circumference Not Integer %%%%%%%%%%%%%%%%%%%%%%%%%%%%%%%%
\section{ Circumference ratio is not an ideal integer}
If the Ratio of the circumference of the designed orbit of the large synchrotron to that of the small synchrotron is not an ideal integer, we have the following relation, where $\lambda<10^{-2}$.
\begin{equation}
\frac{C^l}{C^s}=\kappa \pm \lambda = \frac{m}{n}\pm \lambda \label{circumference_ratio_noint}
\end{equation}
From the circumference ratio, the revolution frequency ratio of two synchrotrons can be calculated.
\begin{equation}
\frac{f_{rev}^{l}}{f_{rev}^{s}}=\frac{1}{\kappa\pm \lambda} \label{rev_freq_ratio_noint}
\end{equation}
Based on eq.~\ref{rev_freq_ratio_noint} and harmonic number, the $f_{rf}$ are calculated by eq.~\ref{rf_freq_s_noint} and eq.~\ref{rf_freq_l_noint}
\begin{equation} 
f_{rf}^{s}= h^s \times f_{rev}^{s}=h^s \times (\kappa\pm \lambda) \times f_{rev}^{l} \label{rf_freq_s_noint}
\end{equation}
\begin{equation} 
f_{rf}^{l}= h^l \times f_{rev}^{l} \label{rf_freq_l_noint}
\end{equation}
%%%%%%%%%%%%%%%%%% Harmonic = circumference ratio Integer %%%%%%%%%%%%%%%%%%%%%%%%%%%%%%%%
\subsection{Harmonic ratio equals to the circumference ratio}
When the ratio of the harmonic number of the large synchrotron to that of the small synchrotron does not equal to the integer part of the circumference ratio, we have the following relation.
\begin{equation}
\frac {h^{l}}{h^{s}}= \kappa  \label{harmonic_1_noint}
\end{equation}
Substituting eq.~\ref{harmonic_1_noint} into eq.~\ref{rf_freq_l_noint} and eq.~\ref{rf_freq_s_noint}, the following relation is deduced. 
\begin{equation} 
f_{rf}^{s}= =h^s \times (\kappa\pm \lambda) \times f_{rev}^{l} =h^s \times \kappa \times f_{rev}^{l}+h^s \times \lambda \times f_{rev}^{l}=h^l\times f_{rev}^{l} +h^s \times \lambda \times f_{rev}^{l}= f_{rf}^{l}+h^s \times \lambda \times f_{rev}^{l}\label{equ_rf_freq_noint}
\end{equation}

In this scenario, the rf cavity frequencies of two synchrotrons are different, which is the user case of the B2B transfer from the SIS18 to the ESR. Four batches of $U^{28+}$ at \SI{200}{meV/\atomicmassunit} are injected into continous eight out of ten buckets of SIS100. Each batch consists of two bunches. We know $\kappa=5$, $h^l=10$ and $h^s=2$, so the cavity rf frequencies of two synchrotrons comply with eq.~\ref{equ_rf_freq}.

For the RF synchronization, both phase shift and frequency beating methods are applicable. After the synchronization, the phase difference between the SIS18 and SIS100 revolution frequency markers equals to the sum of $t_{src}$, $t_{trg}$ and TOF. When the 1st and 2nd buckets are to be filled, $d_{pattern}$=0. When the 3rd and 4th buckets, $d_{pattern}$=one SIS18 revolution period. When the 5th and 6th buckets, $d_{pattern}$= 2 $\times$ one SIS18 revolution period. When the 7th and 8th buckets, $d_{pattern}$= 3 $\times$ one SIS18 revolution period.  With the phase shift method, we have an infinite synchronization window in the ideal situation, within which each bunch could be injected into its corresponding bucket at any time.

%%%%%%%%%%%%%%%%%% Harmonic != circumference ratio Integer %%%%%%%%%%%%%%%%%%%%%%%%%%%%%%%%
\subsection{Harmonic ratio does not equal to the circumference ratio} 
When the ratio of the harmonic number of the large synchrotron to that of the small synchrotron does not equal to the circumference ratio, we have the following relation.
\begin{equation}
\frac {h^{l}}{h^{s}}\neq \kappa  \label{harmonic_1_iinoint}
\end{equation}
We know 
\begin{equation}
\frac {f_{rf}^{l}}{f_{rf}^{s}}= \frac {h^l f_{rev}^{l}}{h^s f_{rev}^{s}} =\frac {h^l}{h^s (\kappa\pm \lambda)}\label{number_iinoint}
\end{equation}
Substituting eq.~\ref{circumference_ratio_noint} into eq.~\ref{number_iinoint}, we get

\begin{equation}
\frac {f_{rf}^{l}}{f_{rf}^{s}}= \frac {h^l n}{h^s(h^s m\pm \lambda n)}\label{number_iinoint}
\end{equation}
namely 
\begin{equation}
\frac {f_{rf}^{s}}{h^s m}\pm\frac{\lambda f_{rev}^{l}}{m}= \frac {f_{rf}^{l}}{h^l n}\label{cir_noint_har_noeq}
\end{equation}

In this scenario, two rf cavity frequencies are slightly different. Two frequencies are $\frac {f_{rf}^{s}}{h^s m}$ and $\frac {f_{rf}^{l}}{h^l n}$. The  beating frequency is $\pm\frac{\lambda f_{rev}^{l}}{m}$. 

This is the user case of the heavy ion B2B transfer from the SIS18 to the ESR. Continous two of four bunches at \SI{30}{meV/\atomicmassunit} are injected into the barrier bucket of the injection orbit of ESR. The large synchrotron is SIS18 and the small one is ESR. We know $\kappa=\frac{m}{n}=2$, $\lambda=-0.003$, $h^l=4$ and $h^s=1$, so eq.~\ref{cir_noint_har_noeq} is expressed as
\begin{equation}
\frac {f_{rf}^{s}}{1 \times 2}-\frac{0.003 f_{rev}^{l}}{2}= \frac {f_{rf}^{l}}{4 \times 1}
\end{equation}

For the frequency beating method, the slightly different frequencies are $\frac {f_{rf}^{s}}{1 \times 2}=342.826KHz$ and $\frac {f_{rf}^{l}}{4 \times 1}=343.255KHz$, the beating frequency is $-\frac{0.003 f_{rev}^{l}}{2}=515Hz$.


