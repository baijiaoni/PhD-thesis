Due to the ratio of the circumference of the injection/extraction orbit and the ratio of the harmonic number of the large synchrotron to that of the small synchrotron, there are several user cases of the B2B transfer for FAIR, see Tab.~\ref{B2B_cases}. In this document, the circumference of the injection/extraction orbit of the synchrotron is denoted by \gls{symb:C_param}, the revolution frequency and rf cavity frequency by \gls{symb:rev_freq} and \gls{symb:cavity_freq}, the beating frequency by \gls{symb:beating_freq} and the harmonic number by \gls{symb:harmonic_param}. The superscript X could be either ``l`` or ``s`` denoting the large or small synchrotron. \gls{symb:integer} is used to represent integers and \gls{symb:decimal} the decimal numbers. Here we define the \gls{glos:cir_ratio} as the ratio of the circumference of the injection/extraction orbit of the large synchrotron to that of the small synchrotron and the \gls{glos:har_ratio} the ratio of the harmonic number of the large synchrotron to that of the small synchrotron.

\begin{landscape} 
\begin{table}[!htb]
\newcommand{\tabincell}[2]{\begin{tabular}{@{}#1@{}}#2\end{tabular}}
\caption{B2B transfer}
\label{B2B_cases}
\begin{center}
    \begin{tabular}{ | c | c | c | c | c | c | c | c|}
    \hline
	\tabincell{c}{Circumference\\ratio} & & \tabincell{c}{Harmonic\\ ratio} &  $f_{rf}^{l}/f_{rf}^{s}$& Bucket label&\tabincell{c}{Frqeuency beating\\Two frequencies\tablefootnote{Two slightly different frequencies}}& \tabincell{c}{Frqeuency beating\\Mismatch\tablefootnote{Bunch and bucket center mismatch}} & Phase shift method\\ \hline
     	\multirow{2}*{{\tabincell{c}{$C^l/C^s=\kappa$ \\Integer}}} & & $h^l/h^s=\kappa$ & $\frac{h^l}{h^s\times \kappa}$=1 & $f_{rf}^{l} or f_{rf}^{s}$&$f_{rf}^{l}$ and $f_{rf}^{s}+\Delta f$ &$\frac{2T_{rev}^{l}}{1/\Delta f}\times360^\circ$ or $\frac{2T_{rev}^{s}}{1/\Delta f}\times360^\circ$& \\ \cline{2-8}

& &$h^l/h^s\neq\kappa$ &$\frac{h^l}{h^s\times \kappa}$ & $\frac{f_{rf}^{l}}{h^l} or \frac{f_{rf}^{s}}{h^s\times \kappa}$ & $\frac{f_{rf}^{l}}{h^l} and \frac{f_{rf}^{s}}{h^s\times \kappa}+\Delta f$ &$\frac{2\times h^l \times T_{rev}^{l}}{1/\Delta f}\times360^\circ$ or $\frac{2\times (h^s\times \kappa)\times T_{rev}^{s}}{1/\Delta f}\times360^\circ$& \\ \hline
											
     	\multirow{2}*{\tabincell{c}{$C^l/C^s=$\\ $\iota+ \lambda$ \\Not Integer}}&\tabincell{c}{$\iota+ \lambda$ \\close to integer\\($\iota$ is integer)} & $\frac{h^l}{h^s\times (\iota+ \lambda)}$& $\frac{f_{rf}^{l}}{h^l} or \frac{f_{rf}^{s}}{h^s\times \iota}$ & $\frac{f_{rf}^{l}}{h^l} and \frac{f_{rf}^{s}}{h^s\times \iota}$ &$\frac{2\times h^l \times T_{rev}^{l}}{1/\Delta f}\times360^\circ$ or $\frac{2\times (h^s\times \iota)\times T_{rev}^{s}}{1/\Delta f}\times360^\circ$&  \\ \cline{2-8}


 &\multirow{1}*{\tabincell{c}{$\iota+ \lambda$ \\far away from integer\\($\iota$ is expressed by $\frac{m}{n}$)}} & $\frac{h^l}{h^s \times (m/n+ \lambda)}\tablefootnote{$\frac{f_rf^{l}}{f_{rf}{s}}=\frac{h^l f_rev^{l}}{h^s  f_{rev}^{s}}=\frac{h^l C^{s}}{h^s C_l}=\frac{h^l}{h^s (m/n+\lambda)}=\frac{h^l\times n}{h^s \times m+ h^s \times\lambda\times n}$}$ &$\frac{f_{rf}^{l}}{h^l\times n} or \frac{f_{rf}^{s}}{h^s\times m}$ & $\frac{f_{rf}^{l}}{h^l\times n} and \frac{f_{rf}^{s}}{h^s\times m}$ &$\frac{2\times (h^l\times n) \times T_{rev}^{l}}{1/\Delta f}\times360^\circ$ or $\frac{2\times (h^s\times m)\times T_{rev}^{s}}{1/\Delta f}\times360^\circ$&  \\ \hline
 
    \end{tabular}
\end{center}
\end{table}
\end{landscape} 



\begin{landscape} 
\begin{table}[!htb]
\newcommand{\tabincell}[2]{\begin{tabular}{@{}#1@{}}#2\end{tabular}}
\caption{FAIR user cases of the B2B transfer}
\label{B2B_cases}
\begin{center}
    \begin{tabular}{ | c | c | c | c | c | c | c | c |}
    \hline
	\tabincell{c}{Circumference\\ratio} & & \tabincell{c}{Harmonic\\ ratio} & $C^l/C^s$ &$h^l$ & $h^s$ &  $f_{rf}^{l}/f_{rf}^{s}$\tablefootnote{$\frac{f_rf^{l}}{f_{rf}{s}}=\frac{h^l f_rev^{l}}{h^s  f_{rev}^{s}}=\frac{h^l C^{s}}{h^s C_l}$} & User case of FAIR accelerators\\ \hline
     	\multirow{3}*{{\tabincell{c}{$C^l/C^s=\kappa$ \\Integer}}} & & $h^l/h^s=\kappa$ & 5& 10 & 2 & $\frac{h^l}{h^s\times \kappa}=\frac{10}{2\times 5}=1$ & $U^{28+}$ \tabincell{c}{B2B transfer \\from SIS18 to SIS100} \\ \cline{2-8}
& &\multirow{2}*{$h^l/h^s\neq\kappa$} & 5& 10 & 1 & $\frac{h^l}{h^s\times \kappa}=\frac{10}{1\times 5}=2$ &\tabincell{c}{$H^{+}$ B2B transfer \\from SIS18 to SIS100} \\ \cline{4-8}
											& & & 2& 1 & 1 & $\frac{h^l}{h^s\times \kappa}=\frac{1}{1\times 2}=\frac{1}{2}$ &\tabincell{c}{B2B transfer\\ from ESR to CRYRING} \\ \hline
     	\multirow{6}*{\tabincell{c}{$C^l/C^s=$\\ $\iota+ \lambda$ \\Not Integer}}&\multirow{3}*{\tabincell{c}{$\iota+ \lambda$ \\close to integer\\($\iota$ is integer)}}  & &2-0.003& 4&1 & $\frac{h^l}{h^s\times (\iota+ \lambda)}=\frac{4}{1\times (2-0.003)}$ & \tabincell{c}{h=4 B2B transfer \\from SIS18 to ESR} \\ \cline{3-8}
 								  &	  & &2-0.003& 1&1 &$\frac{h^l}{h^s\times (\iota+ \lambda)}=\frac{1}{1\times (2-0.003)}$ & \tabincell{c}{h=1 B2B transfer\\ from SIS18 to ESR}\\ \cline{3-8}
 									&  & &1+0.021\tablefootnote{}& 5&1 &$\frac{f_{rf}^{Ext}}{f_{rf}^{Inj}}=\iota+ \lambda=1+0.021$ & \tabincell{c}{$H^{+}$ B2B transfer\\ from SIS100 to CR -----} \\ \cline{2-8}

 &\multirow{3}*{\tabincell{c}{$\iota+ \lambda$ \\far away from integer\\($\iota$ is expressed by $\frac{m}{n}$)}}
											& & 2.5-0.054& 2&1 & $\frac{f_{rf}^{Inj}}{f_{rf}^{Ext}}=\frac{m}{n}+ \lambda=\frac{5}{2}-0.054$ & \tabincell{c}{RIB B2B transfer \\from SIS100 to CR----} \\ \cline{3-8}
%										&	& &2.6-0.003& 1&1 &$\frac{h^l\times n}{h^s \times m+ h^s \times\lambda\times n}\tablefootnote{$\frac{f_rf^{l}}{f_{rf}{s}}=\frac{h^l f_rev^{l}}{h^s  f_{rev}^{s}}=\frac{h^l C^{s}}{h^s C_l}=\frac{h^l}{h^s (m/n+\lambda}=\frac{h^l\times n}{h^s \times m+ h^s \times\lambda\times n}$}=\frac{1\times 5}{1 \times 13- 1\times 0.003\times 5}$ &\tabincell{c}{B2B transfer\\ from CR to HESR} \\ \hline
&	& &2.6-0.003& 1&1 &$\frac{h^l}{h^s \times (m/n+ \lambda)}\tablefootnote{$\frac{f_rf^{l}}{f_{rf}{s}}=\frac{h^l f_rev^{l}}{h^s  f_{rev}^{s}}=\frac{h^l C^{s}}{h^s C_l}=\frac{h^l}{h^s (m/n+\lambda)}=\frac{h^l\times n}{h^s \times m+ h^s \times\lambda\times n}$}=\frac{1}{1 \times (13/5-0.003)}$ &\tabincell{c}{B2B transfer\\ from CR to HESR} \\ \cline{3-8}

&& & 1.8+0.048& 1&1 & $\frac{f_{rf}^{Inj}}{f_{rf}^{Ext}}=\frac{m}{n}+ \lambda=\frac{9}{5}+0.048$ & \tabincell{c}{B2B transfer \\from SIS18 to ESR via FRS---} \\ \hline
 

%\begin{center}
%    \begin{tabular}{ | c | c | c | c | c | c | c |}
%    \hline
%	\tabincell{c}{Circumference\\ratio} & & \tabincell{c}{Harmonic\\ ratio} &  $C^l/C^s$ & $h^l$ & $h^s$ & User case of FAIR accelerators\\ \hline
%     	\multirow{3}*{{\tabincell{c}{$C^l/C^s=\kappa$ \\Integer}}} & & $h^l/h^s=\kappa$ & 5& 10 & 2 & $U^{28+}$ \tabincell{c}{B2B transfer \\from SIS18 to SIS100} \\ \cline{2-7}
%& &\multirow{2}*{$h^l/h^s\neq\kappa$} & 5& 10 & 1 &\tabincell{c}{$H^{+}$ B2B transfer \\from SIS18 to SIS100} \\ \cline{4-7}
%											& & & 2& 1 & 1 & \tabincell{c}{B2B transfer\\ from ESR to CRYRING} \\ \hline
%     	\multirow{5}*{\tabincell{c}{$C^l/C^s=$\\ $\iota+ \lambda$ \\Not Integer}}&\multirow{3}*{\tabincell{c}{$\iota+ \lambda$ \\close to integer}}  & &2-0.003& 4&1 & \tabincell{c}{h=4 B2B transfer \\from SIS18 to ESR} \\ \cline{3-7}
% 								  &	  & &2-0.003& 1&1 & \tabincell{c}{h=1 B2B transfer\\ from SIS18 to ESR}\\ \cline{3-7}
% 									&  & &1+0.021\tablefootnote{}& 5&1 & \tabincell{c}{$H^{+}$ B2B transfer\\ from SIS100 to CR} \\ \cline{2-7}
%
% &\multirow{2}*{\tabincell{c}{$\iota+ \lambda$ \\far away from integer}}
%											& & 2.5-0.054& 2&1 & \tabincell{c}{RIB B2B transfer \\from SIS100 to CR} \\ \cline{3-7}
%										&	& &2.6-0.003& 1&1 & \tabincell{c}{B2B transfer\\ from CR to HESR} \\ \hline
    \end{tabular}
\end{center}
\end{table}
\end{landscape} 
%%%%%%%%%%%%%%%%%% Circumference Integer %%%%%%%%%%%%%%%%%%%%%%%%%%%%%%%%
\section{ Circumference ratio is an ideal integer}

If the ratio of the circumference of the injection/extraction orbit of the large synchrotron to that of the small synchrotron is an ideal integer, we have the following relation. 
\begin{equation}
\frac{C^l}{C^s}=\kappa \label{circumference_ratio_int}
\end{equation}
From the circumference ratio, the revolution frequency ratio of two synchrotrons can be calculated.
\begin{equation}
\frac{f_{rev}^{l}}{f_{rev}^{s}}=\frac{1}{\kappa} \label{rev_freq_ratio_int}
\end{equation}
Based on eq.~\ref{rev_freq_ratio_int} and harmonic number, the $f_{rf}^{X}$ is calculated by eq.~\ref{rf_freq_s_int} and eq.~\ref{rf_freq_l_int}
\begin{equation} 
f_{rf}^{s}= h^s \times f_{rev}^{s}=h^s \times \kappa \times f_{rev}^{l} \label{rf_freq_s_int}
\end{equation}
\begin{equation} 
f_{rf}^{l}= h^l \times f_{rev}^{l} \label{rf_freq_l_int}
\end{equation}
%Diving eq.~\ref{rf_freq_l_int} by eq.~\ref{rf_freq_s_int}, we get
%\begin{equation} 
%\frac{f_{rf}^{l}}{f_{rf}^{s}}= \frac{h^l}{h^s \times \kappa} \label{rf_freq_ratio}
%\end{equation}

%%%%%%%%%%%%%%%%%% Harmonic = circumference ratio Integer %%%%%%%%%%%%%%%%%%%%%%%%%%%%%%%%
\subsection{Harmonic ratio equals to the circumference ratio}
\label{sec:cir_no_int}
When the ratio of the harmonic number of the large synchrotron to that of the small synchrotron equals to the circumference ratio, we have the following relation.
\begin{equation}
\frac {h^{l}}{h^{s}}=\frac {C^{l}}{C^{s}}= \kappa  \label{harmonic_1_int}
\end{equation}
Substituting eq.~\ref{harmonic_1_int} into eq.~\ref{rf_freq_l_int}, the following relation is deduced. 
\begin{equation}
f_{rf}^{l}= h^s \times \kappa \times f_{rev}^{l} \label{equ_rf_freq1}
\end{equation}
Compared eq.~\ref{equ_rf_freq1} with eq.~\ref{rf_freq_s_int}, we get
\begin{equation}
f_{rf}^{s}= f_{rf}^{l}\label{equ_rf_freq}
\end{equation}

In this scenario, the rf cavity frequencies of two synchrotrons are same, which is the user case of the $U^{28+}$ B2B transfer from SIS18 to SIS100. Four batches of $U^{28+}$ at \SI{200}{MeV/\atomicmassunit} are injected into continous eight out of ten buckets of SIS100. Each batch consists of two bunches. The large synchroton is SIS100 and the small one SIS18. $\kappa=5$, $h^{SIS100}=10$ and $h^{SIS18}=2$, so the cavity rf frequencies of two synchrotrons comply with eq.~\ref{equ_rf_freq}. More details, please see Appendix \ref{sec:18to100}.

For the RF synchronization, both phase shift and frequency beating methods are applicable for the small or large synchrotrons. There is no difference of the implementation of two methods either on the large or small synchrotron, because they implement their species dependent rf frequency modulation profiles for a same required phase shift and same frequency dutune for the frequency beating method. only when the target synchrotron is empty, the phase will be shifted for the target synchrotron by the phase jump. With the phase shift method, the phase advance between two synchrotrons is a constant, so the synchronization window is ideally infinitely long, within which two synchrotrons remain perfect synchronized. Bunches can be transferred at any time within the window. With the frequency beating method of \SI{200}{Hz} frequency detune, the length of the synchronization window is $2 \times T_{rf}^{SIS100}=\SI{1.272}{us}$ and the mismatch between the bunch and bucket center is $\pm0.05^\circ$.

After the synchronization, the phase difference between the SIS18 and SIS100 revolution frequency markers equals to the sum of $t_{src}$, $t_{trg}$ and $t_{TOF}$. The SIS100 revolution frequency marker works for the bucket label. When the 1st and 2nd buckets are to be filled, $t_{pattern}$=0. When the 3rd and 4th buckets, $t_{pattern}$=one SIS18 revolution period. When the 5th and 6th buckets, $t_{pattern}$= 2 $\times$ one SIS18 revolution period. When the 7th and 8th buckets, $t_{pattern}$= 3 $\times$ one SIS18 revolution period. 

%%%%%%%%%%%%%%%%%% Harmonic != circumference ratio Integer %%%%%%%%%%%%%%%%%%%%%%%%%%%%%%%%
\subsection{Harmonic ratio does not equal to the circumference ratio} 
\label{sec:cir_no_int1}
When the ratio of the harmonic number of the large synchrotron to that of the small synchrotron does not equal to the circumference ratio, we have the following relation.
\begin{equation}
\frac {h^{l}}{h^{s}}\neq \frac {C^{l}}{C^{s}}= \kappa  \label{harmonic_1_noint}
\end{equation}
%We assume 
%\begin{equation}
%\frac {h^{l}}{h^{s} \times \kappa}= \frac {m}{n}  \label{number_noint}
%\end{equation}
%where m and n are used to represent integers.

Eq.~\ref{rf_freq_l_int} divides eq.~\ref{rf_freq_s_int}, we get
\begin{equation}
\frac{f_{rf}^{l}}{f_{rf}^{s}}= \frac{h^l}{h^s \times \kappa} \label{freq_divide}
\end{equation}

%Substituting eq.~\ref{number_noint} into eq.~\ref{freq_divide}, the following relation is deduced. 
%\begin{equation}
%\frac{f_{rf}^{l}}{f_{rf}^{s}}= \frac{m}{n}
%\end{equation}

In this scenario, the rf cavity frequency of one synchrotron is integer times of that of the other synchrotron for FAIR accelerators. Both phase shift and frequency beating methods are applicable for the RF synchronization. There is no difference of the implementation of the phase shift method either on the large or small synchrotron, because they implement their species dependent rf frequency modulation profiles for a same required phase shift. Only when the target synchrotron is empty, the phase jump is applied to the target synchrotron. With the phase shift method, we have an infinite synchronization window. 

For the frequency beating method, from eq.~\ref{freq_divide}, we get
\begin{equation}
\frac{f_{rf}^{l}}{h^l}= \frac{f_{rf}^{s}}{h^s \times \kappa} 
\end{equation}
If we detune $\Delta f$ for $\frac{f_{rf}^{l}}{h^l}$ of the large synchrotron, the rf cavity frequency $ f_{rf}^{l}$ must detune $\Delta f \times h^l$. If we detune $\Delta f$ for $\frac{f_{rf}^{s}}{h^s \times \kappa}$ of the small synchrotron, the rf cavity frequency $ f_{rf}^{s}$ must detune $\Delta f \times (h^s \times \kappa)$. According to the realtion between $h^l$ and $h^s \times \kappa$, we have the following two cases.
\begin{itemize}
	\item $h^l > h^s \times \kappa \rightarrow \Delta f \times h^l > \Delta f \times (h^s \times \kappa)$ 

The frequency detune for the rf cavity frequency of the small synchrotron is smaller than that of the large synchrotron, so the frequency detune is preferred for the small synchrotron.
	\item $h^l < h^s \times \kappa \rightarrow \Delta f \times h^l < \Delta f \times (h^s \times \kappa)$

The frequency detune for the rf cavity frequency of the large synchrotron is smaller than that of the small synchrotron, so the frequency detune is preferred for the large synchrotron.
\end{itemize}

\subsubsection{User case of the $H^{+}$ B2B transfer from SIS18 to SIS100}
Four batches of $H^{+}$ at \SI{4}{GeV/\atomicmassunit} are injected into continous four out of ten buckets of SIS100. Each batch consists of one bunch. The large synchrotron is SIS100 and the small one SIS18. $\kappa=5$, $h^{SIS100}=10$ and $h^{SIS18}=1$, substituting into eq.~\ref{freq_divide}.
\begin{equation}
\frac{f_{rf}^{SIS100}}{f_{rf}^{SIS18}}= \frac {h^{SIS100}}{h^{SIS18} \times \kappa}= \frac{10}{1 \times 5}=\frac{2}{1}
\end{equation}

For the frequency beating method, the frequency detune is preferred for SIS18 becuase of $h^{SIS100} > h^{SIS18} \times \kappa$. With \SI{200}{Hz} frequency detune of SIS18, the length of the synchronization window is $2 \times T_{rf}^{SIS100}=\SI{0.736}{us}$ and the mismatch between the bunch and bucket center is $\pm0.03^\circ$.


In order to inject into the odd and even number buckets, there are two scenarios of the phase difference between the SIS18 and SIS100 revolution frequency markers after the synchronization.
\begin{itemize}
	\item Injection into the odd number buckets
		
		The phase difference between the SIS18 and SIS100 revolution frequency markers equals to $t_{src}$+$t_{trg}$+ $t_{TOF}$. When the 1st bucket is to be filled, $t_{pattern}$=0. When the 3rd bucket is to be filled, $t_{pattern}$=2 $\times$ SIS100 revolution period. 
	\item Injection into the even number buckets
	
		The phase difference between the SIS18 and SIS100 revolution frequency markers equals to $t_{src}$+$t_{trg}$+$t_{TOF}$- $T_{rf}^{100}$. When the 2nd bucket is to be filled, $t_{pattern}$=1 $\times$ SIS100 revolution period. When the 4th bucket is to be filled, $t_{pattern}$=3 $\times$ SIS100 revolution period. 

\end{itemize}

The SIS100 revolution frequency marker works for the bucket label. More details, please see Appendix \ref{sec:18to100}.

\subsubsection{User case of the B2B transfer from ESR to CRYRING}
Only one bunch is injected into one bucket of CRYRING. The large synchrotron is SIS18 and the small one is CRYRING. $\kappa=2$, $h^{ESR}=1$ and $h^{CRYRING}=1$, substituting into eq.~\ref{freq_divide}. 
\begin{equation}
\frac{f_{rf}^{ESR}}{f_{rf}^{CRYRING}}= \frac {h^{ESR}}{h^{CRYRING} \times \kappa}= \frac{1}{1 \times 2}=\frac{1}{2}
\end{equation}

For the RF synchronization, the phase jump for CRYRING is preferred, because CRYRING is empty before the injection. The 1/2 CRYRING revolution frequency marker works for the bucket label. The phase difference between the ESR and 1/2 CRYRING revolution frequency markers equals to $t_{src}$+$t_{trg}$+$t_{TOF}$ after the synchronization. 

%%%%%%%%%%%%%%%%%% Circumference Not Integer %%%%%%%%%%%%%%%%%%%%%%%%%%%%%%%%
\section{ Circumference ratio is not an ideal integer}
If the ratio of the circumference of the injection/extraction orbit of the large synchrotron to that of the small synchrotron is not an ideal integer, $\kappa$ could be expressed as $\iota + \lambda$ and we have the following relation.
\begin{equation}
\frac{C^l}{C^s}=\iota + \lambda \label{circumference_ratio_noint}
\end{equation}
From the circumference ratio, the revolution frequency ratio of two synchrotrons can be calculated.
\begin{equation}
\frac{f_{rev}^{l}}{f_{rev}^{s}}=\frac{1}{\iota+ \lambda} \label{rev_freq_ratio_noint}
\end{equation}
Based on eq.~\ref{rev_freq_ratio_noint} and harmonic number, the $f_{rf}^{X}$ are calculated by eq.~\ref{rf_freq_s_noint} and eq.~\ref{rf_freq_l_noint}
\begin{equation} 
f_{rf}^{s}= h^s \times f_{rev}^{s}=h^s \times (\iota+ \lambda) \times f_{rev}^{l} \label{rf_freq_s_noint}
\end{equation}
\begin{equation} 
f_{rf}^{l}= h^l \times f_{rev}^{l} \label{rf_freq_l_noint}
\end{equation}

We could get the relation between $f_{rf}^{s}$ and $f_{rf}^{l}$ by dividing eq.~\ref{rf_freq_s_noint} by eq.~\ref{rf_freq_l_noint}.
\begin{equation} 
\frac{f_{rf}^{l}}{f_{rf}^{s}}=\frac{h^l}{h^s \times (\iota+ \lambda)}=\frac{h^l}{h^s \times \iota+ h^s \times \lambda}\label{close_to_interger}
\end{equation}

In this scenario, two rf cavity frequencies begin beating automatically. So the frequency beating method is preferred. The synchronization window depends on the beating frequency. The beating frequency corresponding to this mismatch must not be too large in order to guarantee a long enough synchronization window, but also not too small to satisfy the constraint of the maximum synchronization time.
%%%%%%%%%%%%%%%%%% Harmonic = circumference ratio Integer %%%%%%%%%%%%%%%%%%%%%%%%%%%%%%%%
\subsection{Circumference ratio is close to an ideal integer}
When the circumference ratio of the large synchrotron to that of the small synchrotron is very close to an ideal integer, $\iota$ in eq. ~\ref{circumference_ratio_noint} is an integer and $\lambda$ has the order of magnitude $10^{-2}$.
%\begin{equation}
%\frac {h^{l}}{h^{s}}= \iota  \label{harmonic_1_noint}
%\end{equation}
%Substituting eq.~\ref{harmonic_1_noint} into  eq.~\ref{rf_freq_s_noint}, the following relation is deduced. 
%\begin{equation} 
%%f_{rf}^{s}=h^s \times \iota \times f_{rev}^{l}+ h^s \times \lambda \times f_{rev}^{l}=h^l\times f_{rev}^{l} + h^s \times \lambda \times f_{rev}^{l} \label{equ_rf_freq_noint}
%\end{equation}
%Subtituting eq.~\ref{rf_freq_l_noint} into eq.~\ref{equ_rf_freq_noint}, we get
%\begin{equation} 
%f_{rf}^{s}=f_{rf}^{l}+ h^s \times \lambda \times f_{rev}^{l}\label{equ_rf_freq_noint1}
%\end{equation}


In eq.~\ref{close_to_interger}, $h^s\times\lambda $ is much smaller than $h^s \times \iota$ and $h^l$, so the frequency beating method is preferred. The slightly different frequencies are $\frac{f_{rf}^{s}}{h^s \times \iota}$ and $\frac{f_{rf}^{l}}{h^l}$ and the beating frequency is the difference between $\frac{f_{rf}^{s}}{h^s \times \iota}$ and $\frac{f_{rf}^{l}}{h^l}$. If the small synchrotron is the target synchrotron, $\frac{f_{rf}^{s}}{h^s \times \iota}$ works as for the bucket label, or $\frac{f_{rf}^{l}}{h^l}$ works for the bucket label. 

Besides, it is also grouped to this scenario, that the ratio between the extraction and injection caivity rf frequencies is close to an ideal interger when the beam passes some target (e.g. FRS, Pbar) between two synchrotrons. See eq.~\ref{close_to_interger1} 
\begin{equation} 
\frac{f_{rf}^{Ext}}{f_{rf}^{Inj}}=\iota+ \lambda\label{close_to_interger1}
\end{equation}
For the frequency beating method, the slight different frequencies are $\frac{f_{rf}^{Ext}}{\iota}$ and $f_{rf}^{Inj}$ and the beating frequency is the difference between $\frac{f_{rf}^{Ext}}{\iota}$ and $f_{rf}^{Inj}$. 

\subsubsection{User case of h=4 B2B transfer from SIS18 to ESR} 
The user case of the heavy ion B2B transfer from SIS18 to ESR is this scenario. Continous two of four bunches are injected into the barrier bucket of the injection orbit of ESR. The beam is accumulated in ESR. The large synchrotron is SIS18 and the small one is ESR. We know $\iota=2$, $\lambda=-0.003$, $h^{SIS18}=4$ and $h^{ESR}=1$, so eq.~\ref{close_to_interger} is expressed as
\begin{equation}
\frac {f_{rf}^{SIS18}}{f_{rf}^{ESR}}= \frac {4}{1 \times(2- 0.003)}=\frac{4}{1.997}
\end{equation}

The slightly different frequencies are chosen, e.g. $\frac{f_{rf}^{SIS18}}{4}=\SI{343.300}{\kHz}$ and $\frac{f_{rf}^{ESR}}{2}=\SI{342.826}{\kHz}$ for \SI{30}{MeV/\atomicmassunit} heavy ion. Then the beating frequency is \SI{474}{\Hz}. The 1/2 ESR revolution frequency marker works for the bucket label. The length of the synchronization window is $2\times 2\times T_{rf}^{ESR}$ = \SI{5.834}{\us} and the mismatch between the bunch and bucket center is better than $\pm0.51^\circ$. More details, please see Appendix \ref{sec:18toESR}.  After the synchronization, the phase difference between the 1/4 SIS18 and 1/2 ESR revolution frequency markers equals to $t_{src}$+$t_{trg}$+ $t_{TOF}$.

In the real operation, ESR uses different methods, e.g. barrier bucket or unstable fixed point, to accumulate beam instead of normal bucket.  Presently two general schemes of the particle accumulation are possible: moving or fixed barrier RF bucket. In the scheme with moving barrier RF bucket, the bunch is injected in the longitudinal gap prepared by two barrier pulses. The injected beam becomes coasting after switching off the barrier voltages and merges with
the previously stacked beam. The barrier voltages are switched on and moved away from each other to prepare the empty space for the next beam injection. In the fixed barrier bucket scheme, one prepares a stationary voltage distribution consisting of two barrier pulses of opposite sign. The resulting stretched rf potential separates the longitudinal phase space into a stable and an unstable region. After injection onto the unstable region (potential maximum), the particles circulate along all phases and cooling application leads to their capture in the stable region of the phase space (potential well). After some time of the beam cooling the unstable region is free for a next injection without losing of the stored beam. With the barrier bucket, the bunch should be injected into the longitudinal gap or the unstable region of the barrier bucket.

%Put it to the next section!!!
%When the heavy ion beam is transferred to a target, e.g. fragment separator (FRS), the energy of the RIB varies in a wide range. The slightly different frequencies are RIB energy depedent. Here we use an applied case as an example, the energy before the FRS is \SI{550}{MeV/\atomicmassunit} and after is \SI{400}{MeV/\atomicmassunit}. The different frequencies are   $\frac{f_{rf}^{SIS18}}{5}=\SI{223.891}{\kHz}$ and $\frac{f_{rf}^{ESR}}{9}=\SI{219.642}{\kHz}$ and the beating frequency is \SI{4.249}{\kHz}$. The length of the synchronization window is \SI{0.235}{\ms}$ and the mismatch between the bunch and bucket center is less than $\pm0.7^\circ$. More details, please see Appendix \ref{sec:18toESRviaFRS}. The 1/9 ESR revolution frequency marker works for the bucket label. After the synchronization, the phase difference between the 1/5 SIS18 and 1/9 ESR revolution frequency markers depends on the accumulation method. 

\subsubsection{User case of h=1 B2B transfer from SIS18 to ESR} 
One bunch is injected into the barrier bucket of the injection orbit of ESR. The beam is accumulated in ESR. The large synchrotron is SIS18 and the small one is ESR. We know $\iota=2$, $\lambda=-0.003$, $h^{SIS18}=1$ and $h^{ESR}=1$, so eq.~\ref{number_iinoint2} is expressed as
\begin{equation}
\frac {f_{rf}^{SIS18}}{f_{rf}^{ESR}}= \frac {1 }{1 \times( 2- 0.003)}=\frac{1}{1.997}
\end{equation}

For the frequency beating method, the slightly different frequencies are chosen, e.g. $\frac{f_{rf}^{SIS18}}{1}=\SI{989.756}{\kHz}$ and $\frac{f_{rf}^{ESR}}{2}=\SI{988.388}{\kHz}$ for \SI{400}{MeV/\atomicmassunit} proton/heavy ion. Then the beating frequency is \SI{1368}{\Hz}. The length of the synchronization window is $2\times 2 \times T_{rev}^{ESR}= \SI{2.034}{\us}$ and the mismatch between the bunch and bucket center is better than $\pm0.50^\circ$. More details, please see Appendix \ref{sec:18toESR}. The 1/4 ESR revolution frequency marker works for the bucket label. After the synchronization, the phase difference between the 1/2 SIS18 and 1/4 ESR revolution frequency markers depends on the accumulation method.

\subsubsection{User case of $H^{+}$ B2B transfer from SIS100 to CR} 
Only one out of five bunches of proton is extracted from SIS100 and goes to Pbar, then antiproton is produced and injected into one bucket of CR. The large synchrotron is SIS100 and the small one is CR, $h^{SIS100}=5$ and $h^{CR}=1$. Here we take an example, that the proton energy before the Pbar is \SI{28.8}{GeV/\atomicmassunit} and the antiproton energy after the Pbar is \SI{3}{GeV/\atomicmassunit}. Substituting the extraction and injection energy into eq.~\ref{close_to_interger1}, we get
\begin{equation} 
\frac{f_{rf}^{Ext}}{f_{rf}^{Inj}}=\iota+ \lambda=1+0.021
\end{equation}

The slightly difference frequencies are $\frac{f_{rf}^{Ext}}{1}=\SI{1.345}{\MHz}$ and $\frac{f_{rf}^{Inj}}{1}=\SI{1.318}{\MHz}$ and the beating frequency is \SI{27}{\kHz}. The length of the synchronization window is $2\times T_{rf}^{CR}= \SI{1.518}{us}$ and the mismatch between the bunch and bucket center is better than $\pm7.38^\circ$. The CR is empty before the injection, so the phase jump is preferred for CR. More details, please see Appendix \ref{100toCR}.

%%%%%%%%%%%%%%%%%% Harmonic != circumference ratio Integer %%%%%%%%%%%%%%%%%%%%%%%%%%%%%%%%
\subsection{Circumference ratio is far away from an ideal integer} 
When the circumference ratio of the large synchrotron to that of the small synchrotron is far away from an ideal integer, $\iota$ in eq.~\ref{circumference_ratio_noint} could be denoted by $\frac{m}{n}$ (m and n are integers) and eq.~\ref{circumference_ratio_noint} could be expressed as

\begin{equation}
\frac{C^l}{C^s}=\frac{m}{n}+ \lambda \label{circumference_ratio_noint1}
\end{equation}

There are various combination of $\frac{m}{n}$ and $\lambda$, $\lambda$ should be with the order of magnitude $10^{-2}$ in order to guarantee the beating slow enough.


Substituting $\iota$ by $\frac{m}{n}$ into eq.~\ref{close_to_interger}, we could get the relation between $f_{rf}^{s}$ and $f_{rf}^{l}$.
\begin{equation} 
\frac{f_{rf}^{l}}{f_{rf}^{s}}=\frac{h^l\times n}{h^s \times m+ h^s \times\lambda\times n}\label{close_to_interger1}
\end{equation}

With the frequency beating method, the slightly different frequencies are $\frac{f_{rf}^{l}}{h^l\times n}$ and $\frac{f_{rf}^{s}}{h^s\times m}$ and the beating frequency is the difference between $\frac{f_{rf}^{l}}{h^l\times n}$ and $\frac{f_{rf}^{s}}{h^s\times m}$.

Besides, it is also grouped to this scenario, that the ratio between the extraction and injection caivity rf frequencies is far away from an ideal interger when the beam passes some target (e.g. FRS, Pbar) between two synchrotrons. The ratio between extraction and injection energy can be expressed as
\begin{equation} 
\frac{f_{rf}^{Ext}}{f_{rf}^{Inj}}=\frac{m}{n}+ \lambda\label{close_to_interger11}
\end{equation}
For the frequency beating method, the slight different frequencies are $\frac{f_{rf}^{Ext}}{m}$ and $\frac{f_{rf}^{Inj}}{n}$ and the beating frequency is the difference between $\frac{f_{rf}^{Ext}}{m}$ and $\frac{f_{rf}^{Inj}}{n}$. 


%\begin{equation}
%\frac {h^{l}}{h^{s}}\neq \kappa  \label{harmonic_1_iinoint}
%\end{equation}
%According to the relation between the revolution and rf cavity frequencies, we know 
%\begin{equation}
%\frac {f_{rf}^{l}}{f_{rf}^{s}}= \frac {h^l f_{rev}^{l}}{h^s f_{rev}^{s}}\label{number_iinoint}
%\end{equation}
%Substituting eq.~\ref{rev_freq_ratio_noint} into eq.~\ref{number_iinoint}
%\begin{equation}
%\frac {f_{rf}^{l}}{f_{rf}^{s}}= \frac {h^l}{h^s (\kappa+ \lambda)}\label{number_iinoint2}
%\end{equation}
%Substituting eq.~\ref{circumference_ratio_noint} into eq.~\ref{number_iinoint}, we get
%
%\begin{equation}
%\frac {f_{rf}^{l}}{f_{rf}^{s}}= \frac {h^l n}{h^s( m+ \lambda n)}\label{number_iinoint1}
%\end{equation}
%namely 
%\begin{equation}
%\frac {f_{rf}^{s}}{h^s m}+\frac{\lambda f_{rev}^{l}}{m}= \frac {f_{rf}^{l}}{h^l n}\label{cir_noint_har_noeq}
%\end{equation}
%
%In this scenario, two rf cavity frequencies are different, so the frequency beating method is preferred. 
%Two frequencies are $\frac {f_{rf}^{s}}{h^s m}$ and $\frac {f_{rf}^{l}}{h^l n}$. The  beating frequency is $+\frac{\lambda f_{rev}^{l}}{m}$. 

\subsubsection{User case of B2B transfer from CR to HESR} 

One bunch of CR is injected into one bucket of HESR. The beam is accumulated in HESR. The large synchrotron is HESR and the small one is CR. $\frac{m}{n}=2.6=\frac{13}{5}$, $\lambda=-0.003$, $h^{HESR}=1$ and $h^{CR}=1$, so eq.~\ref{close_to_interger1} is expressed as
\begin{equation} 
\frac{f_{rf}^{HESR}}{f_{rf}^{CR}}=\frac{1\times 5}{1 \times 13- 1 \times 0.003\times 5}
\end{equation}

%The B2B transfer from CR to HESR is impossible to be achieved by the frequency beating method because there exist no slightly different frequencies between two cavity rf frequencies. Besides, the bunch is stochastic cooling by electrons in CR. So only the phase shift for HESR is the only synchroniyation method. 
The slightly different frequencies are chosen, e.g. $\frac{f_{rf}^{HESR}}{5}=\SI{101.426}{\kHz}$ and $\frac{f_{rf}^{CR}}{13}=\SI{101.290}{\kHz}$ for \SI{3}{GeV/\atomicmassunit} antiproton. Then the beating frequency is \SI{136}{\Hz}. The 1/5 HESR revolution frequency marker works for the bucket label. The length of the synchronization window is $2\times 5\times T_{rf}^{HESR}$ = \SI{19.719}{\us} and the mismatch between the bunch and bucket center is better than $\pm0.48^\circ$. For the RIB, e.g. \SI{740}{MeV/\atomicmassunit}, $\frac{f_{rf}^{HESR}}{5}=\SI{86.608}{\kHz}$ and $\frac{f_{rf}^{CR}}{13}=\SI{86.493}{\kHz}$ for \SI{3}{GeV/\atomicmassunit} antiproton. Then the beating frequency is \SI{113}{\Hz}. The 1/5 HESR revolution frequency marker works for the bucket label. The length of the synchronization window is $2\times 5\times T_{rf}^{HESR}$ = \SI{23.090}{\us} and the mismatch between the bunch and bucket center is better than $\pm0.47^\circ$. More details, please see Appendix \ref{sec:CRtoHESR}. After the synchronization, the phase difference between the 1/13 CR and 1/5 HESR revolution frequency markers depends on the accumulation method.


\subsubsection{User case of RIB B2B transfer from SIS100 to CR} 
Only one out of two bunches is extracted from SIS100 and goes to Super FRS, then RIB is produced and injected into one bucket of CR. The large synchrotron is SIS100 and the small one is CR. $h^{SIS100}=2$ and $h^{CR}=1$. Here we take an example, that the energy of the heavy ion beam before the Super FRS is \SI{1.5}{GeV/\atomicmassunit} and the RIB energy after the Super FRS is \SI{740}{MeV/\atomicmassunit}. Substituting the extraction and injection energy into eq.~\ref{close_to_interger11}, we get
\begin{equation} 
\frac{f_{rf}^{Inj}}{f_{rf}^{Ext}}=\frac{m}{n}+ \lambda=\frac{5}{2}-0.054
\end{equation}

The slightly difference frequencies are $\frac{f_{rf}^{Ext}}{2}=\SI{0.230}{\MHz}$ of SIS100 and $\frac{f_{rf}^{Inj}}{5}=\SI{0.225}{\MHz}$ of CR and the beating frequency is \SI{5}{\kHz}. The length of the synchronization window is $2\times 5 \times T_{rf}^{CR}= \SI{8.890}{us}$ and the mismatch between the bunch and bucket center is better than $\pm8.01^\circ$. The CR is empty before the injection, so the phase jump is preferred for CR. More details, please see Appendix \ref{100toCR}.

\subsubsection{User case of B2B transfer from SIS18 to ESR via FRS} 
Only one bunch is extracted from SIS18 and goes to FRS, then RIB is produced and injected into one bucket of ESR. Here we take an applied case as an example, that the energy of the heavy ion beam before the FRS is \SI{550}{MeV/\atomicmassunit} and the RIB energy after the FRS is \SI{400}{MeV/\atomicmassunit}. Substituting the extraction and injection energy into eq.~\ref{close_to_interger11}, we get
\begin{equation} 
\frac{f_{rf}^{Inj}}{f_{rf}^{Ext}}=\frac{m}{n}+ \lambda=\frac{9}{5}+0.048
\end{equation}

The slightly difference frequencies are $\frac{f_{rf}^{SIS18}}{9}=\SI{215.393}{\kHz}$ of SIS18 and $\frac{f_{rf}^{ESR}}{5}=\SI{219.642}{\kHz}$ of ESR and the beating frequency is \SI{4.227}{\kHz}. The length of the synchronization window is $2\times 9 \times T_{rf}^{ESR}= \SI{9.106}{us}$ and the mismatch between the bunch and bucket center is better than $\pm6.92^\circ$. More details, please see Appendix \ref{sec:18toESRvia FRS}.
