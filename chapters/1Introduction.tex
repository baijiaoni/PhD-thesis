

%%%%%%%%%%%%%%%%%%%%%%%%%%%%%%%%%%%%%%%%%%%%%%%%%%%%%%%%%%%%%%
Beams of high energy particles are useful for both fundamental and applied research in the science, and also in many technical and industrial fields unrelated to fundamental research. It has been estimated that there are approximately 30000 accelerators worldwide. Only about $1\%$ of them are research machines with energies above \SI{1}{GeV} ~\cite{noauthor_particle_2017}. In addition, high energy heavy ion beams with high intensity are required for many experiments. In order to get high energy heavy ion beams with high intensity, the acceleration is generally divided into several energy stages: The first energy stage is achieved usually by a linear accelerator followed by a small ring, which is called the ``booster`` and the second stage by a large ring, which is usually called ``main ring``.  The maximum energy of a beam is related to the magnetic rigidity of the dipole magnets used in the accelerator, which is the product of the magnetic field and the bending radius of a particle immersed in the magnetic field. For the Facility for Antiproton and Ion Research (\gls{FAIR}) project under construction at \gls{GSI} Helmholtz center for Heavy Ion Research GmbH (short: GSI)\footnote{Planckstra\ss{}e 1, 64291 Darmstadt, www.gsi.de} ~\cite{eschke_international_2005, noauthor_fair_2011}, the intermediate charge state ions are used to increase the beam intensity by reducing space charge at the booster \gls{SIS18}\footnote{SIS18 stands for SchwerIonen Synchrotron (18 Tm magnetic rigidity).}, causing the larger mass-to-charge ratio. However, the intermediate charge state ion beams can not be accelerated to high enough energy at the booster due to the constraints of the magnetic rigidity. On the other hand, the intermediate charge state ion beams cause the dynamic vacuum challenge by their significantly enhanced cross section for ionization and their high potential for generating ion desorption driven vacuum instabilities. Since the bending radius of the main ring \gls{SIS100}\footnote{SIS100 stands for SchwerIonen Synchrotron (100 Tm magnetic rigidity).} is larger than that of the booster, the magnetic field in the main ring starts the further acceleration at a lower level. Beside, at the main ring, the perfect control over the dynamic vacuum realizes smaller cross sections for charge exchange and the special lattice design optimizes best control of the ionization beam loss ~\cite{spiller_technologies_2016}. Furthermore, with the double ring facility, high average intensity heavy ion beams can be provided with the help of the beam stacking by the multiple injection. Hence, the transfer of beam between rings is of great importance for high energy beams with high intensity.





\begin{figure}[H]
   \centering   
   \includegraphics*[width=150mm]{B2B.pdf}
   \caption{Illustration of a bunch-to-bucket transfer.}
	\caption*{\textsl{\small{Red rectangles represent buckets and blue dots bunches.}}}

   \label{B2B}
\end{figure}

The beam transfer is not arbitrary. A \gls{glos:bunch} of particles running in a ring should be transferred into the correct position of another ring. Fig. ~\ref{B2B} illustrates the transfer of a bunch of particles between two rings. The example in Fig. ~\ref{B2B} is with the circumference ratio between the right and left rings of four. Bunches of particles are transferred from the left ring to the right one. The blue ellipse represents a bunch of particles and the red rectangle represents the allowable area for particles to be injected. The red rectangles are equally spaced around the ring and determined by the rf frequency. The white space between two red rectangles is forbidden for particles. The allowable area (red rectangle) for particles is termed as a ``bucket`` and a bunch of particles (blue ellipse) as a ``bunch``. The definition of a bunch and a bucket from the accelerator physics perspective, please see Chap. 2. There are two buckets at the left ring and every bucket keeps a bunch. There are eight buckets in the right ring and two of them are filled with bunches. The left ring is connected to a transfer beamline by a kicker, which is called the ``kicker 1``. When the kicker $1$ is off, bunches circulate around the ring. When it is on, bunches will be guided from the ring to the transfer beamline at a specific position around the ring, which is called the ``extraction point`` (represented as a black short bar on the left ring). The transfer beamline is connected to the right ring by another kicker, called the ``kicker 2``. When the kicker $2$ is on, bunches will be guided from the transfer beamline to the right ring at a specific position around the ring, which is called the ``injection point`` (represented as a black short bar on the right ring). Generally both kickers are off. The bunch-to-bucket (B2B) transfer is defined as that bunches of the left ring are transferred to the correct buckets at the right ring. For the B2B transfer, bunches at the left ring and buckets at the right ring must have not only a constant but same velocity. Because the circumference of the right ring is four times longer than that of the left ring, bunches run four cycles of the left ring when buckets run one cycle of the right ring. The distance between two bunches of the left ring is equal to the distance between two continuous buckets of the right ring. Besides, the relative position between bunches and buckets must match. Bunches of the left ring are guided to the transfer beamline and transferred to the right ring. They are guided exactly to two empty buckets of the right ring. Every time when a bunch of the left ring passes by the extraction point, a bucket of the right ring will pass by the injection point after a specific time delay, which equals to the time-of-flight of a bunch in the transfer beamline. What's more, the time for the beam guide in the transfer beamline is of great importance, determining which buckets to be filled. In Fig. ~\ref{B2B}, two empty buckets closely following the filled buckets of the right ring need to be filled (represented as the dotted ellipse). The kicker $2$ must be switched on when the first empty bucket following two filled buckets passes the injection point and the kicker $1$ must be switched on a specific time earlier, when a bunch passes by the extraction point. 

The ring is called the ``source ring``, from which the beam is extracted. The ring is called the ``target ring``, into which the beam is injected. From the above illustration, several preconditions are compulsory for the B2B transfer. The first precondition is that bunches of the source ring and buckets of the target ring have a constant speed, namely the revolution frequencies of the two rf systems of the source and target rings must be constant and therewith the constant cavity rf frequencies, which are harmonics of the revolution frequencies. Beam feedback loops on the rf system are usually implemented in order keep the stability of the beam. The constant revolution frequency requires that beam feedback loops must be switched off before the B2B transfer. The second precondition is that bunches and buckets are with a same speed, which requires that the revolution frequency ratio between two rings is equal to the reciprocal of the circumference ratio. When the circumference ratio between two rings is an integer, the phase difference between two revolution frequencies is constant. It means that bunches always pass the extraction position a constant time earlier/later before/after buckets pass the injection position. But the constant phase difference is in general not correct for the transfer. In order to get the correct phase difference, an azimuthal positioning of bunches in the source ring or buckets in the target ring must be adjusted. This is called the ``phase shift method``. After the phase shift, the phase difference of two revolution frequencies is correct and the correct phase difference stays for an infinite time theoretically. Because beam feedback loops are switched off, the beam is stable only for a short period of time. So the beam must be transferred as soon as possible. %The time frame of two revolution periods is used for the transfer. 
When the circumference ratio is not an integer, the phase difference between two revolution frequencies of particles orbiting in the rings varies periodically. Within one period, there must be one point in time when the phase difference between the two rf systems is correct. Before and after this time point, there exists the mismatch between bunches and buckets. The earlier and later than this time point within a period, the larger the mismatch. Waiting for the phase difference to match is called the ''frequency beating method''. For both the phase shift and frequency beating methods, the transfer can only happen when the mismatch is smaller than a tolerable limit, introducing a time frame. The time frame is called the ``synchronization window``, which achieves the ``coarse synchronization`` between the machines.

Bunches are switched from one path to another path by kicker magnets (short: kicker). The extraction kicker kicks bunches out of the source ring to the transfer beamline and the injection kicker kicks them from the transfer beamline into buckets of the target ring. They are located at the extraction position and injection position in Fig. ~\ref{B2B}. When the phase difference between the two rf systems is correct, the extraction kicker can kick bunches out of the source ring at the exact time-of-flight to the transfer beamline before empty buckets pass the injection kicker. With the synchronization window, the extraction and injection kickers must be fired at the correct time in order to transfer bunches into correct empty buckets. The process of the kicker firing at the correct time is termed as the ``fine synchronization``.


\section{Bunch-to-Bucket Transfer Worldwide}
Nowadays, there are several accelerator institutes in the world, who operate the B2B transfer among rings for specific purposes. 
	\gls{CERN}, the European Organization for Nuclear Research, is one of the world's largest and most respected center for scientific research. The Large Hadron Collider (\gls{LHC}) beam injection chain achieves the proton beam with the energy of \SI{7}{TeV}. After accelerated by the Linac2, the beam is adiabatically captured and accelerated at the Proton Synchrotron Booster (\gls{PSB}) as of today. The linear accelerator will change with the Linac4 coming into operation, which injects the beam directly into PSB buckets. Then bunches are further injected into the Proton Synchrotron (\gls{PS}), the Super Proton Synchrotron (\gls{SPS}) and the LHC. For the LHC heavy ion beam injection chain with the achievement of the energy of \SI{2.76}{TeV/u}, beams are first of all transferred into the Low Energy Ion Ring (\gls{LEIR}) from the Linac3 and bunches are further injected into the PS, the SPS and the LHC ~\cite{noauthor_cern_nodate}. For the Japan Proton Accelerator Complex (\gls{J-PARC}), bunches are transferred from the Rapid Cycle Synchrotron (\gls{RCS}) to buckets of the Main Ring (\gls{MR}) ~\cite{noauthor_j-parc_2016}. For the Brookhaven National Laboratory (\gls{BNL}), bunches are transferred from the Booster into buckets of the Alternating Gradient Synchrotron (\gls{AGS}) and bunches of AGS are transferred further into the Relativistic Heavy Ion Collider (\gls{RHIC}) ~\cite{noauthor_brookhaven_2017}. Fermi National Accelerator Laboratory (\gls{Fermilab}) accelerator complex provides high energy proton beams for a broad range of experiments. Proton beams are injected into the Recycler from the \gls{Fermilab} Booster. Then proton bunches are transferred into the Main Injector from the Recycler. The beam is accelerated to the energy of \SI{120}{GeV} for further application ~\cite{noauthor_fermi_2016}. 
%Institute of Modern Physics of the Chinese Academy of Sciences (\gls{IMP}) operates the Heavy Ion Research Facility in Lanzhou (\gls{HIRFL}). Two existing cyclotrons, the Sector Focusing Cyclotron (\gls{SFC}) and the Separated Sector Cyclotron (\gls{SSC}), are used as an injector system for the Cooler Storage Ring main ring (\gls{CSRm}) for the accumulation, cooling and acceleration. Then the beam is extracted from CSRm to produce radioactive ion beams or highly-charged heavy ions, which can be transferred to the Cooler Storage Ring experimental ring (\gls{CSRe}) for many experiments ~\cite{noauthor_institute_2013}.  

\gls{FAIR}, the Facility for Antiproton and Ion Research, is a new international accelerator facility. It is aiming at providing high-energy beams of ions of all elements from hydrogen to uranium with high intensities, as well as beams of rare isotopes
and beams of antiprotons. The new FAIR accelerator complex in its full version will consist of the \gls{SIS100}, the Collector Ring (\gls{CR}) and the High Energy Storage Ring (\gls{HESR}) ~\cite{spiller_fair_2006, steck_advanced_2009}. FAIR has many rings, so the B2B transfer among FAIR ring accelerators is of great importance to accelerate beams to higher energy with high intensity and achieve beams for various experiments. Based on the existing GSI \gls{UNILAC} and \gls{SIS18} serving as injectors, high intensity ion beams over the whole range of stable isotopes will be accelerated in the new heavy ion machine SIS100 to higher energy. The beam from the SIS100 will be transferred to the CR via the antiproton (\gls{Pbar}) target\footnote{The antiproton target is used to produce antiprotons in inelastic collisions of high energy protons with nucleons of a target nucleus.} or the superconducting fragment separator (Super-FRS) \footnote{Super-FRS is used to produce rare isotopes of all elements up to uranium at relativistic energies and spatially separate them within a few hundred nanoseconds.}. The CR has the purpose of stochastic cooling of both secondary rare isotope and antiproton beams and of measuring nuclear masses ~\cite{nolden_collector_2006, bar_technical_2013}. The CR transfers the beam to the HESR for the accumulation. The HESR serves storage ring experiments with high energy antiproton and rare isotope beams ~\cite{toelle_hesr_2007}. The proton and heavy ion beams can also be transported from the SIS18 to the existing GSI Experimental Storage Ring (\gls{ESR}) and further to the GSI storage ring CRYRING@ESR (short: CRYRING) for the atomic and nuclear physics experiment ~\cite{lestinsky_cryring_2015, lestinsky_cryring_2012}. The proton and heavy ion beams can also be transferred from the SIS18 to the ESR via the fragment separator (\gls{FRS})\footnote{An ion-optical device used to focus and separate products from the collision of relativistic ion beams with thin targets.}.

When the circumference ratio between the large and small rings is an integer, e.g. the SIS100 and the SIS18, the phase difference between two revolution frequencies of rings is constant. The frequency is in the \SI{}{MHz} range. In this scenario, both the phase shift method and the frequency beating methods can be used for the match of the phase difference between the two rf systems. When the circumference ratio between FAIR accelerator pairs is not an integer, e.g. the SIS18 and the ESR \footnote{ESR has an injection/extraction orbit, which is \SI{15}{cm} longer than the design orbit. The orbit of ESR in this dissertation means the injection/extraction orbit.}, the phase difference between two revolution frequencies beats automatically. The beating frequency is in the \SI{}{kHz} range. In this scenario, only the frequency beating method is used for the phase match. The synchronization window for the FAIR B2B transfer is in the \SI{}{\micro\second} range. The beams of ion species, from hydrogen to uranium as well as antiproton and rare isotope beams, should be transferred among all rings. And every transfer must be achieved within an upper bound of about \SI{10}{ms} and a B2B injection mismatch in the range between $-1^\circ$ and $+1^\circ$. Both the phase shift and the frequency beating method should be applicable in the upcoming FAIR facility. The B2B transfer system is designed to work in a parallel operation, e.g. the transfer from the SIS18 to the SIS100 and the transfer from the ESR to the CRYRING can be performed at the same time. It is capable to transfer the beam between two rings via the pbar target, the FRS or the Super-FRS. The B2B transfer system must coordinate with the SIS100 fast beam abort system for all unacceptable failure or situation. 




\section{Objectives, Contribution and Structure of the Dissertation}
The development of the concept of the FAIR B2B transfer system is a joint work, mainly done by Thibault Ferrand and me with the support from the CSCO and PBRF departments of GSI. Thibault Ferrand has contributed to the development of the system from the low-level radio frequency (LLRF) perspective in his PhD thesis ~\cite{ferrand_development_nodate}. The contribution to the development of the system from the timing perspective is presented in the dissertation at hand. In addition, my dedication extends to the development of the technical concept of the FAIR B2B transfer system, which has been worked out for the FAIR project ~\cite{bai_f-tc-c-05_2016}. This dissertation concentrates on the introduction of the concept of the FAIR B2B transfer system and its application for FAIR accelerators. In addition, it explains the systematic investigation for the FAIR B2B transfer system in details. 

The dissertation is structured as follows and as depicted in Fig.~\ref{dissertation_structure}.
\begin{figure}[!htb]
   \centering   
   \includegraphics*[width=130mm]{dissertation_structure.jpg}
   \caption{Structure of the dissertation.}
	\caption*{\textsl{\small{Contributions are marked blue and red is team work. Existing systems or theory are not colored.}}}
   \label{dissertation_structure}
\end{figure}

In Chap. ~\ref{background}, the theoretical background for the B2B transfer is reviewed. First of all, the energy match, the phase match and the voltage match between the source and target rings are introduced. Secondly, two rf synchronization methods are discussed from the perspective of beam dynamics for the phase alignment. At the end of this chapter, the synchronization of the extraction and injection kicker magnets are presented.

Chap.~\ref{technical} is concerned with the existing FAIR technical basis for the development of the FAIR B2B transfer system. The B2B transfer system is realized based on the FAIR control system and low-level radio frequency system, so these two systems are introduced. In addition, the FAIR B2B transfer system must coordinate with the Machine Protection System (MPS), which is also introduced in this chapter.  

In Chap.~\ref{concept}, a brief overview on the basic idea of the B2B transfer system is presented. After that the basic procedure of the FAIR B2B transfer is introduced and the realization of each step of the procedure is explained. In addition, the FAIR B2B transfer system is explained from the data flow perspective. The comparison between the FAIR B2B transfer system and the current B2B transfer with the GSI control system is discussed before the chapter ends. 

Chap.~\ref{realization} presents the systematic investigation for the B2B transfer system. Two synchronization methods are analyzed from the perspective of the beam dynamics for the B2B transfer from the SIS18 to the SIS100. In addition, the realization of the B2B transfer system based on the General Machine Timing (GMT) system is investigated, which contains the accuracy of the start of the synchronization window, the characterization of the White Rabbit network for the B2B transfer, the flow chart and the time constraints of the system. Besides, the different trigger scenarios of the SIS18 extraction and SIS100 injection kickers are systematically investigated. Finally, a test setup of the timing system required for the transfer using the frequency beating method is introduced and results of a prototype implemented in firmware running on a soft CPU are presented.

The applications of the FAIR B2B transfer system for FAIR accelerator pairs with the frequency beating method are outlined in Chap.~\ref{application}. The applications are classified into three categories according to the circumference ratio. The circumference ratio between FAIR accelerator rings can be an integer, close to an integer or far away from an integer. For each category, the corresponding FAIR use cases are analyzed. 
