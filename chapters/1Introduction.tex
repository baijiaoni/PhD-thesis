FAIR, Facility for Antiproton and Ion Research, is a new international accelerator facility under construction, which will use antiprotons and ions to perform research in the fields of nuclear, hadron and particle physics, atomic and anti-matter physics, high density plasma physics, and applications in condensed matter physics, biology and the bio-medical sciences\footnote{https://en.wikipedia.org/wiki/Facility_for_Antiproton_and_Ion_Research}. FAIR is aiming at providing high-energy beams with high intensities. Based on the existing GSI Helmholtz Centre for Heavy Ion Research accelerators UNILAC and SIS18 serving as injectors, high intensity ion beams over the whole range of stable isotopes will be accelerated in the new heavy ion machine SIS100 to higher energies. The new FAIR accelerator complex with storage rings consists of SIS100, Collector Ring CR, accumulator/decelerator ring RESR and New Experimental Storage Ring NESR. An additional High Energy Storage Ring HESR serves experiments with high energy antiprotons and rare isotope beams.  
For example the \gls{glos:bunch}-to-\gls{glos:bucket} (\gls{B2B}) transfer system for FAIR aims at supporting beam transfers between the following accelerators:
\begin{itemize}
\item The B2B transfer from SIS18 to SIS100
\item The B2B transfer from SIS18 to ESR
\item The B2B transfer from ESR to CRYRING
\item The B2B transfer from SIS100 to CR
\item The bunch to barrier bucket transfer from CR to HESR and later to RESR
\end{itemize}

The B2B transfer system focuses first of all on the transfer from SIS18 to SIS100, but it will be firstly tested for the transfer from SIS18 to ESR and further to CRYRING.

\section{Objectives, Contribution and Structure of the Dissertation}
This dissertation is structured as follows.

In Chap.2 the basic principles for the B2B transfer are reviewed. First of all, the energy match between the source and target synchrotrons is introduced. Secondly, two RF synchronization methods are discussed from the perspective of beam dynamics. Once more, the bucket label and the extraction/injection kicker synchronization are discussed. At the end of this chapter, the beam indication for the beam instrumentation is explained.

Chap.3 is concerned with the existing FAIR technical basis for the development of the B2B transfer system and the uniqueness of the system. The system is realized based on the FAIR control system and Low-Level RF system, so these two systems are introduced. In addition, the B2B transfer system for FAIR is unique compared with other similar systems, so the uniqueness of the system is given before the chapter ends. 

In Chap.4, a brief overview on the basic procedure of the B2B transfer system is given. The detailed realization of every basic principles mentioned in Chap.2 are presented. The B2B transfer system is explained from the data/signal flow point of view.

The application of the B2B transfer system for FAIR accelerators are outlined in Chap.5. The applications are classified according to the feature of the circumference ratio.  

Chap.6

Chap.7
