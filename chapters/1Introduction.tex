%\centerline{With regard to excellence, it is not enough to know,} 
%\centerline{but we must try to have and use it.}
%\centerline{- Aristotle (384 BC - 322 BC), \textit{Nicomachean Ethics, ch. 9}}
%
%\vspace{3ex}
%Research leading to the discovery of elementary particles and to ideas for the acceleration of such particles to high energy is dotted with particularly important milestones which from time to time set the directions for further experimental and theoretical research~\cite{wiedemann_particle_2015}. Due to the thirst for knowledge of human beings, the particle accelerator has big progress in the past 100 years. First of all, we look back at the history of the particle accelerator.
%
%In 1927 Rolf Wider$\phi$e developed the first linear accelerator, which maked use of an alternating current (AC) voltage and a series of drift tubes. As the particle gets faster and faster, the drift tubes need to be longer and longer. The linear accelerator needs to be very long for particles to be accelerated to high energies. In 1928, Ernest Lawrence of the University of California, inspired by the work of Wider$\phi$e, had the idea of utilizing a curved path for a particle accelerator. A magnetic field perpendicular to the plane of motion of an accelerated particle makes the particle taking a curved path. By studying the simple relationship between the forces acting on the particle, Lawrence realized that the increase in the radius of the path taken by the particle is compensated by the increased velocity of the particle if the magnetic field, the charge of the particle and the particles mass remain constant. With this in mind, he built what became known as a cyclotron. As the speed of the particle increases, the radius of the semi-circular motion of the particle increases until the particles are eventually focused out of the cyclotron as a high energy beam. The physical principles governing the Betatron were first described by Wider$\phi$e in a paper in 1928 and put into practice in 1940 by Donald Kerst. The development of the Betatron was driven by the demand for high energy X-rays and gamma rays for medical and research usage. The Betatron consists of a main ring, a doughnut shaped vacuum chamber, in which electrons, produced by an electron gun within the chamber, are accelerated. The chamber is set up between the two poles of an electromagnet driven by an AC current, which results in a constantly changing magnetic field. The basic principle of the Synchrotron is to maintain the accelerated particles at a constant orbital radius. This is achieved by synchronizing the magnetic field strength with the energy of the accelerated particles. So, as the particles are accelerated and gain energy, the magnetic field is increased, keeping the particles orbit constant. The first Synchrotron to be built was a modified Betatron and was completed by two English physicists, Frank Goward and D. Barnes ~\cite{_accelerators_????}.
%
%Mankind has never stopped learning and practice about the particle accelerators. Nowadays, particle accelerators is playing an important role not only in research application (particle physics, nuclear physics, atomic physics, cosmology and astrophysics, chemistry and biology), but also in element analysis, power engineering, medicine for diagnostics and for therapy and industrial processing~\cite{barbalat_applications_1994}. 

%%%%%%%%%%%%%%%%%%%%%%%%%%%%%%%%%%%%%%%%%%%%%%%%%%%%%%%%%%%%%%
Beams of high energy particles are useful for both fundamental and applied research in the sciences, and also in many technical and industrial fields unrelated to fundamental research. It has been estimated that there are approximately 30000 accelerators worldwide. Of these, only about $1\%$ are research machines with energies above \SI{1}{GeV} \footnote{\url{https://en.wikipedia.org/wiki/Particle_accelerator}}.  As we all known, particles are accelerated by the electric field. The Radio Frequency (RF) system is devoted to generate the electric field at RF cavities around the ring. Particles are accelerated when they pass through RF cavities. Every RF cavity has a limited frequency range,  particles at rest could not be accelerated to several tens of \SI{}{GeV} energy in one ring accelerator.  So the acceleration must be divided into several energy stages: the first energy stage is achieved usually by a small ring, which is called ``booster`` and the second stage by a large ring, which is usally called ``main ring``.  The energy of a beam is determined by the 'magnetic rigidity' of the dipole, which is the multiplication of the magnetic field and the bending radius of a particle immersed in the magnetic field. Compared with the main ring, the booster needs higher power supply of the dipole to achieve the same energy. Hence, it is significant to use  the beam transfer to achieve high energy beam. Particle beam transfer among rings is also used for the production of the high intensity beam with some specific technolgy, e.g. the beam is transferred to a storage ring for the beam accumulation and beam compression. What's more, the beam transfer between two rings is also important for the beam quality.  Interactions  between  the accelerated particles and the residual-gas atoms in the vaccume chamber may degrade the beam quality, resulting in a reduction in the lifetime of the particles \footnote{\url{http://www.chem.elte.hu/foundations/altkem/vakuumtechnika/CERN13.pdf}}. The shorter the interaction, the better the beam quality. The particle beams can be accelerated at the booster at the same time as the beams in the main ring are transferred to some experiments or other rings. Hence, the transfer of particle beam among rings is of great importance.
%a higher energy beam can be used to more finely penetrate and discriminate a particle probe, to study the smaller structure and to create more massive particles by a collision with a target particle \footnote{\url{The Need for Large Accelerators}}. 
%
% High-energy particle beams are used for particle physics experiments in large facilities, both fundamental and applied research in the sciences, and also in many technical and industrial fields unrelated to fundamental research.
%
%The rf system provides longitudinal focusing which constrains the particle motion in the longitudinal phase space to a confined region. The confined region is called the ``rf bucket`` ~\cite{_lhc_????}. The ``bunch`` is the collection of particles captured within one rf bucket  ~\cite{_lhc_????}. The \gls{glos:bunch}-to-\gls{glos:bucket} (\gls{B2B}) transfer is transferring bunches from a source ring accelerator into the center of specified buckets of a target ring accelerator.  
%
%The bunch-to-bucket transfer is best described in the picture of James Bond jumping from a truck to catch the fast rope of a helicopter, see Fig. ~\ref{Helikopter}. The bucket of the source ring accelerator is regared as the truck and James Bond as a bunch within this bucket. The helicopter is seen as a bucket of the target ring accelerator. The truck and the helicopter must move at constant speed (the bunch-to-bucket transfer only when the rf frequencies of two rf systems of the source and target ring accelerators are constant). They should have the same speed. If the truck moves with \SI{600}{\kilo\meter/\hour} and the helicopter with \SI{600}{\kilo\meter/\hour}, then James Bond won't survive (the bunches and the buckets are with same speed, namely two rf systems have correct frequencies). They must be of the right relative position, so that James Bound will catch the fast rope, when he jumps (the phase difference between two rf systems must be correct). Of course the time of James Bond’s jump is of extreme importance (the bunches must be injected into the correct buckets). 
%
%\begin{figure}[H]
%   \centering   
%   \includegraphics*[width=50mm]{Helikopter.jpg}
%   \caption{Bunch-to-bucket transfer described in the picture of James Bond jumping from a truck to a helicopter}
%   \label{Helikopter}
%\end{figure}
\begin{figure}[H]
   \centering   
   \includegraphics*[width=140mm]{B2B.jpg}
   \caption{Bunch-to-bucket transfer illustration.}
   \label{B2B}
\end{figure}

The beam transfer is not arbitrary. A bunch of particles runing in a ring should be transferred into the correct position of another ring. For the sake of simplicity, Fig. ~\ref{B2B} illustrates the transfer of a bunch of particles between two rings with same circumference. Bunch of particles are transferred from the left ring to the right one. The blue ellipse represents a bunch of particles and the red area is forbiddened for the particles. The bunch of particles are equally spaced and so do forbidden area. The forbidden area of the left ring is exactly same as that of the right ring. The number in the empty region between two forbidden area of the right ring represents the order to be filled by a bunch of partcles. The left ring is connected to a track by a switch, which is called ``switch 1``. When the switch $1$ is off, the particles circulates around the ring. When it is on, the particles will be guided from the ring to the track at a specific position around the ring, which is called ``extraction point`` (represented as a black short bar). The track is connected to the right ring by another switch, called ``switch 2``. When the switch $2$ is on, the particles will be guided from the track to the right ring at a specific position around the ring, which is called ``injection point`` (represented as a black short bar). Generally both switches are off. For the proper transfer, two rings must have not only constant linear velocity, but also same linear velocity. Besides, the relative position of two rings must match. A bunch of particles of the left ring is guided to the track and transferred to the right ring. It is guided exactly to the empty region of the right ring. Everytime when a bunch of particles of the left ring passes by the extraction point, a empty region of the right ring will pass by the injection point after a specific time delay, which equals to the time of flight of particles on the track. What's more, the switch of the track is of great importance, which decides the order of empty region to be filled. The empty region labelled with number ``1`` must be filled first of all. The switch $2$ must be switched on when the empty region $1$ passes the injection point and the switch $1$ must be switched on a specific time earlier, when a bunch of particles pass by the extraction point. 

In order keep the stability of the beam, a beam feedback loop on the RF system is implemented. The ring is called ``source ring``, from which the beam is extracted. The ring is called ``target ring``, into which the beam is injected. From the above illustration, several preconditions are compulsory for the beam transfer. The first precondition is that a bunch of particles of the source ring and the empty region of the target ring have a constant speed, namely the revolution frequency of two rf systems of the source and target rings must be constant. The beam feedback loop must be switched off. The second one is that a bunch of particles and an empty space are with same speed, which requires that the revolution frequency ratio between two rings is equle to the circumference ratio. When the circumference ratio between two rings is an integer, the phase difference between two revolution frequencies is constant. It means that a bunch of particles always passes the extraction position a constant time earlier/later before/after the empty space passes the injection position. But the constant phase difference is not correct for the transfer. In order to get the correct phase difference, an azimuthal positioning of the bunch of particles in the source ring or the empty space in the target ring must be adjusted. This is called  ''phase shift method''. After the phase shift, the phase difference of two revolution frequencies is correct and the correct phase difference keeps infinite theoretically. Because the beam feedback loop is switched off, the beam is stable only for a period of time. The time frame of two revolution periods is used for the transfer. When the circumference ratio is not an integer, the phase difference between two revolution frequencies varies from $0^\circ$ to $360^\circ$ periodically. Within one period, there is only one time point when the phase difference is perfect correct. Before and after this time point, there exists the mistmatch between the bunch of particles and the empty space. The earlier and later than this time point, the larger the mismatch. This is called ''frequency beating method''. The transfer can only happen within a time frame, with which the mismatch is smaller than the upper bound. The time frame is called ``synchronization window`` for both the phase shift and frequency beating methods, which achieves the ``coarse synchronization``.

A bunch of particles are switched from one path to another path by dipole magnets, which are called ``kicker magnet`` or ``kicker``. The extraction kicker kicks a bunch of particles out of the source ring to the track and the injection kicker kicks it from the track in the target ring. They are located at the extraction postion and injection position (see Fig. ~\ref{B2B}). After the phase correction between two rf systems, the extraction kicker could kick a bunch of particles in the source ring at the exact time-of-flight of the track before an empty space passes the injection kicker. With the synchronization window, the extraction and injection kickers must be fired at the correct time in order to transfer a bunch of particles into correct empty space. The process of the kicker firing at the correct time is called ``fine synchronization``.

%
%A bunch of particles is constrained by rf system and the number of bunch around the ring is determined by rf system. provides longitudinal focusing which constrains the particle motion in the longitudinal phase space to a confined region. The confined region is called the ``rf bucket`` ~\cite{_lhc_????}. The ``bunch`` is the collection of particles captured within one rf bucket  ~\cite{_lhc_????}. The \gls{glos:bunch}-to-\gls{glos:bucket} (\gls{B2B}) transfer is transferring bunches from a source ring accelerator into the center of specified buckets of a target ring accelerator.  
%%

Nowadays, there are several accelerator institutes in the world, who operate the beam transfer among rings for specific purposes. 
CERN, the European Organization for Nuclear Research, is one of the world's largest and most respected centres for scientific research. The Large Hadron Collider (LHC) beam injection chain achieves the proton beam with the energy of \SI{7}{TeV}. After accelerated by a linear accelerator, bunches are injected into buckets of the Proton Synchrotron Booster (PSB) and further into the Proton Synchrotron (PS), the Super Proton Synchrotron (SPS) and LHC ~\cite{ferrand_synchronization_2015}. For the LHC heavy ion beam injection chain with the achievement of the energy of \SI{2.76}{TeV/u}, bunches are first of all injected into the Low Energy Ion Ring (LEIR) and the following transfer from PSB to LHC is same as proton beam ~\cite{ferrand_synchronization_2015}. For Japan Proton Accelerator Complex (J-PARC), bunches are transferred from the Rapid Cycle Synchrotron (RCS) to buckets of the Main Ring (MR) ~\cite{_j-parc_????}. The Booster of Brookhaven National Laboratory (BNL) transfers bunches to buckets of the Alternating Gradient Synchrotron (AGS) and bunches of AGS are transferred further into the Relativistic Heavy Ion Collider (RHIC) ~\cite{_brookhaven_????}. Fermi National Accelerator Laboratory's accelerator complex provides high energy proton beams for a broad range of experiments. Proton beams are injected into the Recycler from the Fermilab Booster. Then the proton beam enters the Main Injector from the Recycler. The beam is accelerated to the energy of \SI{120}{GeV}. Some of the proton beam from the Booster will be used to produce pions in a specially designed target system. These pions will then decay into particles called muons. These muons will be injected into Muon Delivery Ring. The Muon Delivery Ring delivers these muons into a muon storage ring for further study ~\cite{_fermi_????}. IMP, Institute of Modern Physics of the Chinese Academy of Sciences, operates the Heavy Ion Research Facility (HIRFL) in Lanzhou. The two existing cyclotrons Sector Focusing Cyclotron (SFC) and the Separated Sector Cyclotron (SSC) are used as an injector system for the Cooler Storage Ring main ring (CSRm) for the accumulation, cooling and acceleration. Then the beam is extracted from CSRm to produce radioactive ion beams or highly-charged heavy ions, which can be transferred to the Cooler Storage Ring experimental ring (CSRe) for many experiments ~\cite{_institute_????, man_survey_2002}.  

FAIR\footnote{\url{https://en.wikipedia.org/wiki/Facility_for_Antiproton_and_Ion_Research}}, Facility for Antiproton and Ion Research, is a new international accelerator facility under construction at GSI Helmholtz center for Heavy Ion Research GmbH (short: GSI)\footnote{Planckstrasse 1, 64291 Darmstadt, www.gsi.de} ~\cite{eschke_international_2005, _fair_2011}. It is aiming at providing high-energy beams of ions from antiprotons to uranium with high intensities. The new FAIR accelerator complex with storage rings consists of SIS100, SIS300, Collector Ring CR, accumulator/decelerator ring RESR, New Experimental Storage Ring NESR and Hign Energy Storage Ring HESR ~\cite{spiller_fair_2006, steck_advanced_2008}. FAIR has so many rings, so the bunch-to-bucket transfer among FAIR ring accelerators is of great importance to accelerate beam to higher energy and achieve beam for various experiments. Based on the existing GSI \gls{UNILAC} and SIS18 serving as injectors, high intensity ion beams over the whole range of stable isotopes will be accelerated in the new heavy ion machine SIS100/SIS300 to higher energies. The beam from SIS100 will be transferred to CR via Pbar or Super-Fragment Separator \footnote{\url{http://www.fair-center.eu/public/experiment-program/nustar-physics/superfrs.html}}. CR has the purpose of stochastic precooling of both secondary rare isotope and antiproton beams and of measuring nuclear masses in an isochronous mode ~\cite{nolden_collector_2006, abe_technical_2010}. The CR transfers the beam to HESR and further to RESR for the accumulation. HESR serves experiments with high energy antiprotons and rare isotope beams ~\cite{toelle_hesr_2007}. The proton and heavy ion beam is transported from SIS18 to the existing GSI Experimental Storage Ring (ESR) and further to the first FAIR-storage ring CRYRING@ESR for the atomic and nuclear physics experiment ~\cite{lestinsky_cryring_2015, lestinsky_cryring_2012}. The proton and heavy ion could also be transferred from SIS18 to ESR via the Fragment Separator (\gls{FRS})\footnote{An ion-optical device used to focus and separate products from the collision of relativistic ion beams with thin targets.}.

For many FAIR accelerator pairs, the circumference ratio between the large and small rings is an integer, e.g. SIS100 and SIS18, so the phase difference between two revolution frequencies of rings is constant. The revolution freuqncy is of the order of magnitude of $10^6$  \SI{}{Hz}. In this scenario, the phase shift method must be used for the match of the phase difference. When the circumference ratio between FAIR accelerator paris is not a integer, e.g. SIS18 and ESR, the phase difference between two revolution frequencies adjusts automatically. The frequency of the phase difference variability is of the order of magnitude of $10^3$  \SI{}{Hz}. The synchronization window for FAIR is of the order of magnitude of $10^{-6}$ \SI{}{s}.

For FAIR, the beams of ion species, from hydrogen to uranium, should be transferred among all rings, as well as antiprotons. And every transfer must be achieved within \SI{10}{ms} and the injection mismatch less than $\pm1^\circ$. Both the phase shift and the frequency beating method should be applicable. The transfers are supported to be in parallel, e.g. the transfer from SIS18 to SIS100 and transfer from ESR to CRYRING are performed at the same time. It is cable to transfer the beam between two rings via Fragment Separator (\gls{FRS})\footnote{An ion-optical device used to focus and separate products from the collision of relativistic ion beams with thin targets.} or Super-Fragment Separator (Super FRS). The transfer must coordinate with the SIS100 emergency dump for unacceptable failure or situation. 


%\section{First overview of the FAIR bunch-to-bucket transfer system}
%From the example of James Bond, we know several precondition for the B2B transfer. The first precondition is that the bunch of the source ring and the bucket of the target ring have a constant speed, namely the rf frequency of two rf systems of the source and target ring accelerators must be constant. The second one is that the bunch and bucket are with same speed, which requires that two rf systems must have correct frequencies. 
%
%In order to transfer bunches to buckets, two rf systems must have correct phase difference. The process of achieving the correct phase difference is named as ``synchronization``. There are two methods for the synchronization. The synchronization is achieved by an azimuthal positioning of the bunch in the source synchrotron or the bucket in the target synchrotron. This is so-called ''phase shift method''. When two rf frequencies are slightly different, they are beating, perceived as periodic variations in phase difference, whose rate is the difference between the two frequencies. The synchronization is automatically achieved. This is so-called ''frequency beating method''. The phase shift method provides a theoretically infinite time frame for the B2B transfer and the frequency beating method provides a finite time frame, within which bunches could be transferred into buckets with the bunch-to-bucket center mismatch smaller than the upper bound $\pm1^\circ$. The time frame is called ``synchronization window``, which achieves the ``coarse synchronization``.
%
%The bunch are switched from one path to another path by dipole magnets, which are called ``kicker magnet`` or ``kicker``. The extraction kicker kicks the bunch out of the source ring and the injection kicker kicks the bunch in the target ring. After the synchronization between two rf systems, the extraction kicker must kick the bunch in the source ring at the exact time-of-flight between two rings before the center of a specific bucket passes the injection kicker. With the synchronization window, the extraction and injection kickers must be fired at the correct time in order to transfer bunches into correct buckets. The process of the kicker firing at the correct time is called ``fine synchronization``.
%CERN, the European Organization for Nuclear Research, is one of the world's largest and most respected centres for scientific research.  At CERN, the world’s largest and most complex scientific instruments are used to study the laws of nature. 
%
%The LHC proton beam injection chain achieves the proton beam with the energy of \SI{7}{TeV}. From a source, protons are injected into a linear accelerator, LINAC-2. Protons are accelerated to an kinetic energy of \SI{50}{MeV}. Pre-accelerated proton bunches are then injected into the first synchrotron of the chain, namely Proton Synchrotron Booster (PSB). The PSB is composed of four identical rings stacked on the top of one another. This acceleration stage carries the proton bunches up to \SI{1.4}{GeV}. The protons with an energy of 1.4 Gev are injected into the Proton Synchrotron (PS), which accelerates the protons to the energy of \SI{25}{GeV}. The proton beams are further transferred to the Super Proton Synchrotron (SPS), which served as the final injector for high-intensity proton beams for the Large Hadron Collider (LHC) ~\cite{ferrand_synchronization_2015}.
%
%The LHC heavy ion beam injection chain achieves the heavy ion with the energy of \SI{2.76}{TeV/u}. Heavy ions like lead and argon ions are produced by an ECR ion source \footnote{\url{https://en.wikipedia.org/wiki/Electron_cyclotron_resonance}} and sent to LINAC-3 to reach the energy of \SI{4.2}{MeV/u}, which is required for injection into the Low Energy Ion Ring (LEIR). The LEIR injects ion beam in the PS at the energy of \SI{72.2}{MeV/u}. The beam is accelerated to the energy of \SI{5.9}{GeV/u} and than transferred to SPS. SPS accelerated the beam to \SI{177}{GeV/u} and SPS transfers the beam further to LHC ~\cite{ferrand_synchronization_2015}.
%
%J-PARC, Japan Proton Accelerator Complex, is a high-intensity proton accelerator to provide various secondary particles for a variety of experiments.  The J-PARC accelerator consists of the Linac, the Rapid Cycle Synchrotron (RCS) and the Main Ring (MR). The Linac accelerates negative hydrogen ions up to \SI{181}{MeV} to inject to the RCS. The negative ions are converted to protons at the injection point of the RCS by a charge-stripping foil. The RCS accelerates protons up to \SI{3}{GeV}, then extract them to the Material and Life science Facility (MLF) and the MR. The MR accelerates the protons up to \SI{30}{GeV}, then extract in one turn to a target ~\cite{_j-parc_????}. 
%
%BNL, Brookhaven National Laboratory, specializes in high energy physics, energy technologies, environmental science, and nanotechnology. High energy physics of BNL studies the nature of the particles that constitute matter and radiation . The BNL accelerator complex contains Linac, Booster, Alternating Gradient Synchrotron (AGS) and Relativistic Heavy Ion Collider (RHIC). The Booster accepts ions from Electron Beam Ion Source (EBIS) and protons from the LINAC, and accelerates the beam to the minimum AGS energy before injecting it into the AGS for further acceleration and delivery to RHIC ring ~\cite{_brookhaven_????}.
%
%Fermi National Accelerator Laboratory's accelerator complex comprises seven particle accelerators and storage rings. It produces powerful, high-energy neutrino beam and provides proton beams for a broad range of new and existing experiments. Proton beams enter the Fermilab Booster from the Linac, accelerating to an energy of \SI{8}{GeV}. The Recycler is a kind of staging area for proton beams after they exit the Booster. Once beam enters the Recycler, protons form a more intense beam with the energy of \SI{8}{GeV}. Then the proton beam enters the Main Injector, which situated directly beneath the Recycler. The beam is accelerated to the energy of \SI{120}{GeV}. Some of the proton beam from the Booster will be used to produce pions in a specially designed target system. These pions will then decay into particles called muons. These muons will be injected into Muon Delivery Ring. The Muon Delivery Ring delivers these muons into a muon storage ring for further study ~\cite{_fermi_????}.
%
%IMP, Institute of Modern Physics of the Chinese Academy of Sciences, operates the Heavy Ion Research Facility (HIRFL) in Lanzhou. HIRFL is used to conduct fundamental research in nuclear and atomic physics, which consists of the Sector Focusing Cyclotron (SFC), the Separated Sector Cyclotron (SSC), the Cooler Storage Ring (CSR) and a number of experimental terminals. CSR includes a main ring (CSRm), an experimental ring (CSRe) and radioactive beam line to connect the two rings. The two  existing cyclotrons SFC and SSC will be used as its injector system for CSRm, which is able to accumulate, cool and accelerate beam to higher energy. Then the beam is extracted to produce radioactive ion beams or highly-charged heavy ions. The radioactive ion beams or highly-charged heavy ions can be transferred to CSRe for many experiments ~\cite{_institute_????, man_survey_2002}.  
%
%\section{First overview of the FAIR bunch-to-bucket transfer system}
%From the example of James Bond, we know several precondition for the B2B transfer. The first precondition is that the bunch of the source ring and the bucket of the target ring have a constant speed, namely the rf frequency of two rf systems of the source and target ring accelerators must be constant. The second one is that the bunch and bucket are with same speed, which requires that two rf systems must have correct frequencies. 
%
%In order to transfer bunches to buckets, two rf systems must have correct phase difference. The process of achieving the correct phase difference is named as ``synchronization``. There are two methods for the synchronization. The synchronization is achieved by an azimuthal positioning of the bunch in the source synchrotron or the bucket in the target synchrotron. This is so-called ''phase shift method''. When two rf frequencies are slightly different, they are beating, perceived as periodic variations in phase difference, whose rate is the difference between the two frequencies. The synchronization is automatically achieved. This is so-called ''frequency beating method''. The phase shift method provides a theoretically infinite time frame for the B2B transfer and the frequency beating method provides a finite time frame, within which bunches could be transferred into buckets with the bunch-to-bucket center mismatch smaller than the upper bound $\pm1^\circ$. The time frame is called ``synchronization window``, which achieves the ``coarse synchronization``.
%
%The bunch are switched from one path to another path by dipole magnets, which are called ``kicker magnet`` or ``kicker``. The extraction kicker kicks the bunch out of the source ring and the injection kicker kicks the bunch in the target ring. After the synchronization between two rf systems, the extraction kicker must kick the bunch in the source ring at the exact time-of-flight between two rings before the center of a specific bucket passes the injection kicker. With the synchronization window, the extraction and injection kickers must be fired at the correct time in order to transfer bunches into correct buckets. The process of the kicker firing at the correct time is called ``fine synchronization``.
%
%
%%The bunches must be transferred from the buckets of an injecting machine into the middle of the buckets in the receiving machine, namely energy match and phase match. The red dots (1, 2, 3) on Fig.~\ref{inj_error} show the trajectory followed by the center of the bunch after injection with a phase error (dot 1: correct momentum but displaced horizontally with respect to the bucket centre). The green dots (a, b, c) correspond to an injection with the correct phase but with an energy deviation~\cite{baudrenghien_low-level_2010}. 
%\begin{figure}[!htb]
%   \centering   
%   \includegraphics*[width=120mm]{inj_error.png}
%   \caption{Bunch-to-bucket injection with phase and energy error.}{X-axis: Phase in radian. Y-axis: momentum, energy or frequency deviation
%from synchronism. The red dots (1, 2, 3) show the trajectory followed by the bunch center for an injection phase error. The green dots (a, b, c) correspond to an injection energy error ~\cite{baudrenghien_low-level_2010}.}
%   \label{inj_error}
%\end{figure}
%
%The injection energy or phase error cause dipole oscillation, that the bunch profile does not change but the phase of the center of charge moves back and forth with respect to the stable phase. 
%
%Besides, the rf voltage must also match. Fig.~\ref{voltage_error} shows the capture of a bunch (marked in red) with perfect phase and energy matching. 
%\begin{figure}[!htb]
%   \centering   
%   \includegraphics*[width=80mm]{voltage_error.png}
%   \caption{Bunch-to-bucket injection with voltage error.}{Injection of a bunch (dark red) in the exact center of the bucket but with phase space trajectories mismatched to the two dimensional phase-momentum bunch profile. The result is a quadrupole oscillation at twice the synchrotron frequency and, after filamentation, significant emittance increase ~\cite{baudrenghien_low-level_2010}.}
%   \label{voltage_error}
%\end{figure}
%
%The center of the bunch falls in the middle of the bucket. The bunch has a non-zero length and therefore occupies an area defined by the
%phase space trajectories in the injector. But it is not matched to the phase space trajectories in the receiving machine (the voltage is too high). The particles of the bunch will follow these trajectories, resulting in the evolution shown on the figure: after one-quarter synchrotron period, the bunch length has been reduced (projection on the phase axis) and the momentum spread has been increased. We call
%this a quadrupole oscillation. It is a modulation of the bunch length (and momentum spread) at twice the synchrotron frequency. After filamentation the bunch emittance will be much increased and this must be avoided ~\cite{baudrenghien_low-level_2010}.
%

\section{Objectives, Contribution and Structure of the Dissertation}
This dissertation contributes to the development of the FAIR B2B transfer system from the timing perspective. It concentrates on the introduction of the concept of the system and its application for FAIR accelerators. In addition, it explains the systematic investigation of the FAIR B2B transfer system in details.

The dissertation is structured as follows and as depicted in Fig.~\ref{dissertation_structure}.
\begin{figure}[!htb]
   \centering   
   \includegraphics*[width=130mm]{dissertation_structure.jpg}
   \caption{Structure of the dissertation.}{Contributions are marked blue and red is team work; existing system or theory are not colored.}
   \label{dissertation_structure}
\end{figure}

In Chap.2 the basic principles for the B2B transfer are reviewed. First of all, the energy match between the source and target synchrotrons is introduced. Secondly, two rf synchronization methods are discussed from the perspective of beam dynamics in order for the phase match. Once more, the bucket label and the extraction/injection kicker synchronization are discussed. At the end of this chapter, the beam indication for the beam instrumentation is mentioned.

Chap.3 is concerned with the existing FAIR technical basis for the development of the B2B transfer system and the uniqueness of the system. The B2B transfer system is realized based on the FAIR control system and Low-Level RF system, so these two systems are introduced. In addition, the uniqueness of the B2B transfer system for FAIR is discussed before the chapter ends. 

In Chap.4, a brief overview on the basic idea of the B2B transfer system is presented. After that the basic precedure of the B2B transfer is introduced and the realization of each step of the procedure. In addition, the B2B transfer system is explained from the data flow perspective.

The application of the B2B transfer system for FAIR accelerators are outlined in Chap.5. The applications are classified into two categories according to the feature of the circumference ratio. The ratio of the circumference between many pair of machines in FAIR is not a perfect integer, e.g. SIS18 and ESR (injection orbit), SIS100 and CR, CR and HESR. So the frequencies of two synchrotrons begin beating automatically. For the pairs with the perfect integer ratio of the circumference, e.g. SIS18 and SIS100, the rf frequency of one synchrotron is detuned to get the frequency beating. For each category, the corresponding FAIR applications are presented. 

Chap.6 presents the systematic investigation for the B2B transfer system, mainly focusing on the timing aspect. The calculation of the synchronization window is explained and the transfer of the B2B messages via the WR network is tested. In addition, for the B2B transfer from SIS18 to SIS100, two synchronization methods are analyzed from the perspective of beam dynamics. The SIS18 extraction and SIS100 injection kicker are systematically investigated. Finally, the test setup is presented and the result is analyzed.

%%%%%%%%%%%%%%%%%%%%%%%%%%%%%%%%%%%%%%%%%
%\bibliography{main}
%\bibliographystyle{plain}
