\centerline{With regard to excellence, it is not enough to know,} 
\centerline{but we must try to have and use it.}
\centerline{- Aristotle (384 BC - 322 BC), \textit{Nicomachean Ethics, ch. 9}}

\vspace{3ex}
Research leading to the discovery of elementary particles and to ideas for the acceleration of such particles to high energy is dotted with particularly important milestones which from time to time set the directions for further experimental and theoretical research~\cite{wiedemann_particle_2015}. We look back at the development in particle accelerator history.

The first development in linear accelerators came from Rolf Wider$\phi$e in 1927, when he built a linear accelerator using an alternating current (AC) voltage and a series of drift tubes. As the particle gets faster and faster, the drift tubes need to be longer and longer. Indeed the faster the particle moves, the greater distance it covers in the same amount of time. This is one of the limiting factors of linear accelerators, they need to be very long for particles to be accelerated to high energies. In 1928, Ernest Lawrence of the University of California, inspired by the work of Wider$\phi$e, had the idea of utilizing a curved path for a particle accelerator. A magnetic field perpendicular to the plane of motion of an accelerated particle will result in the particle taking a curved path. By studying the simple relationship between the forces acting on the particle, Lawrence realised that the increase in the radius of the path taken by the particle is compensated for by the increased velocity of the particle if the magnetic field, the charge of the particle and the particles mass remain constant. With this in mind, he built what became known as a cyclotron. As the speed of the particle increases, the radius of the semi-circular motion of the particle increases until the particles are eventually focused out of the cyclotron as a high energy beam. The physical principles governing the Betatron were first described by Wider$\phi$e in a 1928 paper and put in to practice in 1940 by Donald Kerst. The development of the Betatron was driven by the demand for high energy X-rays and gamma rays for medical and research use. The Betatron consists of a main ring, a doughnut shaped vacuum chamber, in which electrons, produced by an electron gun within the chamber, are accelerated. The chamber is set up between the two poles of an electromagnet driven by an AC current, which results in a constantly changing magnetic field. The basic principle of the Synchrotron is to maintain the accelerated particles at a constant orbital radius. This is achieved by synchronising the magnetic field strength with the energy of the accelerated particles. So, as the particles are accelerated and gain energy, the magnetic field is increased, keeping the particles orbit constant. The first Synchrotron to be built was a modified Betatron and was completed by two English physicists, Frank Goward and D. Barnes. Many Synchrotrons were built after this and by 1954.~\cite{_accelerators_????}

Mankind has never stopped learning and practice about the particle accelerators. Nowadays, particle accelerators is playing an important role not only in research application (particle physics, nuclear physics, atomic physics, cosmology and astrophysics, chemistry and biology), but also in element analysis, power engineering, medicine for diagnostics and for therapy and industrial processing~\cite{barbalat_applications_1994}. 


FAIR, Facility for Antiproton and Ion Research~\cite{_fair_2011}, is a new international accelerator facility under construction at GSI Helmholtz Centre for Heavy Ion Research GmbH (short: GSI)\footnote{Plackstrasse 1, 64291 Darmstadt, www.gsi.de}~\cite{_gsi_2011}, which will use antiprotons and ions to perform research in many fields. It is aiming at providing high-energy beams with high intensities. Based on the existing GSI UNILAC and SIS18 serving as injectors, high intensity ion beams over the whole range of stable isotopes will be accelerated in the new heavy ion machine SIS100 to higher energies. The new FAIR accelerator complex with storage rings consists of SIS100, Collector Ring CR, accumulator/decelerator ring RESR and New Experimental Storage Ring NESR. An additional High Energy Storage Ring HESR serves experiments with high energy antiprotons and rare isotope beams. The \gls{glos:bunch}-to-\gls{glos:bucket} (\gls{B2B}) transfer system for FAIR aims at supporting beam transfers between the following accelerators:
%\footnote{https://en.wikipedia.org/wiki/Facility_for_Antiproton_and_Ion_Research}. 
\begin{itemize}
\item The B2B transfer from SIS18 to SIS100
\item The B2B transfer from SIS18 to ESR
\item The B2B transfer from ESR to CRYRING
\item The B2B transfer from SIS100 to CR
\item The B2B\footnote{B2B: Bunch-to-Barrier bucket} transfer from CR to HESR and later to RESR
\end{itemize}

The system supports not only the simple transfer between two accelerators, but also various complex beam transfer for FAIR. It should be capable to transfer different species beam from one machine cycle to another. It should also be capable to parallel transfer beam through FAIR accelerators. It should be able to transfer the beam between two synchrotrons via FRS\footnote{FRS: Fragment Seperator \\An ion-optical device used to focus and separate products from the collision of relativistic ion beams with thin targets.} or Super FRS. The B2B transfer system focuses first of all on the transfer from SIS18 to SIS100, but it will be firstly tested for the transfer from SIS18 to ESR and further to CRYRING.  

The bunches must be transferred from the buckets of an injecting machine into the middle of the buckets in the receiving machine, namely energy match and phase match. The red dots (1, 2, 3) on Fig.~\ref{inj_error} show the trajectory followed by the centre of the bunch after injection with a phase error (dot 1: correct momentum but displaced horizontally with respect to the bucket centre). The green dots (a, b, c) correspond to an injection with the correct phase but with a energy deviation~\cite{baudrenghien_low-level_2010}. 
\begin{figure}[!htb]
   \centering   
   \includegraphics*[width=120mm]{inj_error.png}
   \caption{Phase space at injection. X-axis: Phase in radian. Y-axis: momentum, energy or frequency deviation
from synchronism. The red dots (1, 2, 3) show the trajectory followed by the bunch centre for an injection phase error. The green dots (a, b, c) correspond to an injection energy error.~\cite{baudrenghien_low-level_2010}}
   \label{inj_error}
\end{figure}

The injection energy or phase error cause dipole oscillation, that the bunch profile does not change but the phase of the centre of charge moves back and forth with respect to the stable phase. 

Besides, the RF voltage must also match. Fig.~\ref{voltage_error} shows the capture of a bunch (marked in red) with perfect phase and energy matching. 
\begin{figure}[H]
   \centering   
   \includegraphics*[width=80mm]{voltage_error.png}
   \caption{Evolution in phase space. Injection of a bunch (dark red) in the exact centre of the bucket but with phase space trajectories mismatched to the two dimensional phase-momentum bunch profile. The result is a quadrupole oscillation at twice the synchrotron frequency and, after filamentation, significant emittance increase.~\cite{baudrenghien_low-level_2010}}
   \label{voltage_error}
\end{figure}

The centre of the bunch falls in the middle of the bucket. The bunch has a non-zero length and therefore occupies an area defined by the
phase space trajectories in the injector. But it is not matched to the phase space trajectories in the receiving machine (the voltage is too high). The particles of the bunch will follow these trajectories, resulting in the evolution shown on the figure: after one-quarter synchrotron period, the bunch length has been reduced (projection on the phase axis) and the momentum spread has been increased. We call
this a quadrupole oscillation. It is a modulation of the bunch length (and momentum spread) at twice the synchrotron frequency. After filamentation the bunch emittance will be much increased and this must be avoided.~\cite{baudrenghien_low-level_2010}


\section{Objectives, Contribution and Structure of the Dissertation}
This dissertation contributes to the development of the B2B transfer system for FAIR from the timing system view of point. It concentrates on the introduction of the concept of the system and its application for FAIR accelerators. In addition, it explains the realization and systematic investigation of the B2B transfer system from the timing system perspective in details.

The dissertation is structured as follows and as depicted in Fig.~\ref{dissertation_structure}.
\begin{figure}[H]
   \centering   
   \includegraphics*[width=130mm]{dissertation_structure.jpg}
   \caption{Structure of the dissertation. Contributions are marked blue and red is team work; existing system or theory are not colored.}
   \label{dissertation_structure}
\end{figure}

In Chap.2 the basic principles for the B2B transfer are reviewed. First of all, the energy match between the source and target synchrotrons is introduced. Secondly, two RF synchronization methods are discussed from the perspective of beam dynamics in order for the phase match. Once more, the bucket label and the extraction/injection kicker synchronization are discussed. At the end of this chapter, the beam indication for the beam instrumentation is mentioned.

Chap.3 is concerned with the existing FAIR technical basis for the development of the B2B transfer system and the uniqueness of the system. The system is realized based on the FAIR control system and Low-Level RF system, so these two systems are introduced. In addition, the uniqueness of the B2B transfer system for FAIR is discussed before the chapter ends. 

In Chap.4, a brief overview on the basic procedure of the B2B transfer system is presented. The detailed realization of every basic principles mentioned in Chap.2 for B2B transfer system for FAIR is presented. In addition, the B2B transfer system is explained from the data/signal flow point of view.

The application of the B2B transfer system for FAIR accelerators are outlined in Chap.5. The applications are classified into two categories according to the feature of the circumference ratio. For each category, the corresponding FAIR applications are presented. 

Chap.6 presents the realizaiton of the systematic investigation for the system, mainly from the timing point of view. The calculation of the synchronization window is explained and the transfer of the B2B timing messages via the WR network is tested. In addition, for the B2B transfer from SIS18 to SIS100, two synchronization methods are analyzed from the beam dynamics point of view. The SIS18 extraction and SIS100 injection kicker are systematically investigated. Finally, the test setup is introduced and the test result is presented and analyzed.

%%%%%%%%%%%%%%%%%%%%%%%%%%%%%%%%%%%%%%%%%
\bibliography{main}
\bibliographystyle{plain}
