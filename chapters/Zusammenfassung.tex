FAIR hat zum Ziel, hochenergetische Ionenstrahlung angefangen von Protonen bis Uran, Antiprotonen und seltene Isotope, mit hohen Intensit\"aten zur erzeugen. Die existierende Anlage besteht aus dem SIS18 und dem ESR. Der neue FAIR Beschleuniger-Komplex wir aus den Synchrotrons SIS100 und SIS300, dem Kollektor-Ring CR und dem Speicherring HESR bestehen. Eine stabiler Transfer von Bunchen in Buckets zwischen allen GSI und FAIR Ringbeschleuniger, ist aus verschiedenen Gr\"unden erforderlich. Bei einer nicht ordnungsgem\"a\ss{}en Strahl\"ubergabe besteht die Gefahr, dass es zu einer Degeneration der Strahlqualit\"at, bis hin zum Strahlverlust kommt. Ein stabiler Bunch-to-Bucket-Transfer zwischen zwei Ringen ist daher sehr wichtig f\"ur FAIR. Obwohl bereits zwischen SIS18 und ESR ein B2B-Transfer realisiert wurde, so ist diese L\"osung aufgrund verschiedener Einschr\"ankungen nicht nutzbar f\"ur FAIR. Hierzu z\"ahlt beispielsweise, dass es auf Basis des alten GSI-Kontrollsystem realisiert wurde, welches in Zukunft vollst\"andig durch das neue FAIR Kontrollsystem ersetzt werden wird. Des Weiteren unterst\"utzt es nicht die Phase-Shift-Method. Die Entwicklung eines FAIR B2B-Transfer Systems, basierend auf der f\"ur FAIR geplanten technischen Infrastruktur, dazu z\"ahlen das FAIR-Kontrollsystem und das FAIR-Low-Level-Radio-Frequenz-System ist daher unbedingt erforderlich.

In dieser Doktorarbeit wird die konzeptionelle Umsetzung des FAIR B2B-Transfer-System, unter ber\"ucksichtigt aller daf\"ur wichtigen Aspekte vorgestellt. Das FAIR B2B-Transfer-System unterst\"utzt die Phase-Shift-Methode, wie auch die Frequency-Beating-Methode und ist anpassungsf\"ahig genug, um einen Transfer zwischen zwei Ringen mit beliebigen Umfangszahlenverh\"altnis zu erm\"oglichen. Es ist m\"oglich, verschiedenen B2B-Transfers zur gleichen Zeit auszuf\"uhren. Beispielsweise kann der B2B-Transfer vom SIS18 zum SIS100 zur gleichen Zeit stattfinden, wie der B2B-Transfer vom ESR zum CRYRING. Auch k\"onnen verschiedenen Ionensorten von einem zum anderen Maschinenzyklus transferiert werden. Das FAIR B2B-Transfer System ist in der Lage, einen Transfer zwischen zwei Ringen auch \"uber den FRS, das Antiprotonen-Target oder den Super FRS durchzuf\"uhren. Es k\"onnen verschiedenen komplexe Bucket-F\"ullmuster ber\"ucksichtigt werden. Au\ss{}erdem hat das FAIR B2B-Transfer System hat eine Schnittstelle zum FAIR-Maschinenschutzsystem (Machine Protection System, MPS), um die Synchrotrons vor gr\"o\ss{}erem Strahlverlust und dadurch bedingt vor Schaden zu bewahren. 
Das B2B-Transfer System erreicht bei Injektion einen Bunch-to-Bucket Mittenversatz von besser $\pm1^\circ$ innerhalb der erforderlichen Transferzeit von \SI{10}{\ms}, weil das Zahlenverh\"altnis der Umf\"ange der beiden Ringe ganzzahlig oder nahezu ganzzahlig ist. Entwicklungsbedarf besteht noch beim Ionentransfer von Sekund\"arstrahlen, wie sie vom Antiprotonen-Target, dem FRS oder dem Super FRS erzeugt werden. Hier besteht das Problem, dass das Verh\"altnis der Energien zwischen Prim\"ar-und
Sekund\"arstrahl sich stark unterscheiden. F\"ur den Transfer von seltenen Isotopen vom SIS100 zum CR \"uber den Super FRS mit \SI{1.5}{GeV/u} Prim\"arstrahlenergie und \SI{740}{MeV/u} Sekund\"arstrahlenergie, betr\"agt der Bunch-to-Bucket Injektion-Mittenversatz zwar zufällig nur $\pm2.1^\circ$, f\"ur den Antiprotonen B2B-Transfer vom SIS100 zum CR \"uber das Antiprotonen-Target und den seltene Isotopen Strahltransfer vom SIS18 zum ESR \"uber den FRS, ist der Bunch-to-Bucket Mittenversatz aber schon gr\"o\ss{}er als $\pm40^\circ$ und damit weit au\ss{}erhalb der Spezifikation. F\"ur die zwei letztgenannten FAIR-Anwendungsf\"alle muss \"uberpr\"uft werden, ob der Transfer unter Zuhilfenahme von speziellen Strahlaufspeicherung-Methoden wie Beispielsweise barrier bucket oder unstable point accumulation gelingt.

Ferner k\"onnen laut der strahldynamischen Analyse des $U^\mathit{28+}$ B2B-Transfer vom SIS18 in das SIS100, sowohl bei sinusf\"ormiger HF-Frequenzmodulation wie auch bei parabolischer HF-Frequenzmodulation bei gleicher Modulationsdauer die Anforderungen an die Strahlstabilit\"at und einen adiabatischen Strahltransfer eingehalten werden. Dennoch liefert die sinusf\"ormige HF-Frequenzmodulation bei gleicher Modulationsdauer bessere Ergebnisse. Die sinusf\"ormige HF-Frequenzmodulation im SIS18 f\"ur
\SI{200}{MeV/u} $U^\mathit{28+}$ ben\"otigt \SI{7}{\ms} und die sinusf\"ormige HF-Frequenzmodulation im SIS18 f\"ur \SI{4}{GeV} $H^+$ ben\"otigt circa \SI{20}{\ms} f\"ur eine Phasenverschiebung jeweils um $\pi$. Die Firmware f\"ur das FAIR-B2B-Timing-System auf der soft-CPU LatticeMico32 der SCU erf\"ullt die Timing-Anforderungen. Nach \"Uberpr\"ufung aller FAIR-Anwendungsf\"alle stellt der h=1 B2B-Transfer vom SIS18 zum ESR mit ca. \SI{500}{\ns} die h\"artesten Genauigkeitsanforderungen an den Beginn des Synchronisationsfesters. Dar\"uber hinaus wird ein spezielles VLAN innerhalb des WR-Netzwerks f\"ur den B2B-Transfer verwendet. Die hierf\"ur zul\"assige Anzahl von WR-Switches, h\"angen nicht nur von der Obergrenze der Latenzzeit ab, sondern auch von der tolerierbaren Frame-Error-Rate (FER). Wenn keine Vorw\"artsfehlerkorrektur f\"ur das B2B-Netzwerk verwendet wird, ist die Anzahl der zul\"assigen WR-Switches hauts\"achlich durch die FER bestimmt. Unter der Annahme, dass der Verlust von einem Frame pro Monat noch akzeptable ist, sind maximal 18 WR-Switches zul\"assig zwischen Data Master (DM) und den zugeh\"origen SCUs und maximal 4 WR-Switches zwischen den f\"ur das B2B zugeordneten SCUs. Wird eine Vorw\"artsfehlerkorrektur f\"ur das B2B-Netzwerk verwendet, so ist die Anzahl der zul\"assigen WR-Switches durch die Latenzzeit bestimmt. In diesem Fall, sind 67 WR-Switches zwischen den f\"ur das B2B zugeh\"origen SCUs  und DM erlaubt und 13 WR-Switches zwischen den f\"ur das B2B zugeh\"origen SCUs. Ferner kann der der SIS18 Extraktions-Kickermagnet im 2. \"Uberrahmen mit einer feste Verz\"ogerungszeit gegen\"uber dem Extraktions-Kickermagnet im 1. \"Uberrahmen  f\"ur alle Ionensorten getriggert werden, wenn die Bunchl\"ucke $25\%$ der HF-Periode betr\"agt. Der SIS100 Injektions-Kickermagnet kann instantan getriggert werden, wenn die Bunchl\"ucke $35\%$ der HF-Periode betr\"agt. 

Die vorliegende Doktorarbeit stellt die wesentlichen Untersuchungsergebnisse f\"ur den FAIR B2B-Transfer-System, aus Sicht der Strahldynamik, Timing und Kicker-Ausl\"osung vor. Dennoch bleiben weitere Untersuchungsaufgaben, die au\ss{}erhalb des Rahmens dieser Doktorarbeit liegen. Diese sind:
\begin{itemize}

\item Die Synchronisation im Mikrosekundenbereich des Antiprotonenstrahls mit dem magnetischen Horn nach dem Antiprotonen-Target.
\item Die Synchronisation zwischen dem Bunchkompressor und der Strahlextraktion.
\item Es m\"ussen noch L\"osungen f\"ur die Anwendungsf\"alle gefunden werden, bei denen Antiprotonen vom SIS100 \"uber das Antiprotonentarget in den CR und seltene Isotope vom SIS18 \"uber den FRS in den ESR transferiert werden m\"ussen. Zu untersuchen ist, ob Barrier-Bucket oder unstable point beam accumulation ein L\"osungsansatz darstellt. 
\end{itemize}

Das FAIR B2B-Transfer-System, das in dieser Doktorarbeit vorgestellt wird, ist anwendbar f\"ur alle FAIR-Anwendungsf\"alle. Dennoch gibt es noch Verbesserungspotential. F\"ur die Phase Shift Methode muss die HF-Frequenzmodulation sehr langsam erfolgen, damit der Strahl der Frequenz\"anderung folgen kann. Um Bunche sobald als m\"oglich in Buckets transferieren zu k\"onnen, kann mit der Phasenverschiebung bereits auf der Beschleunigungsrampe begonnen werden. Zu einem definierten Zeitpunkt w\"ahrend des Beschleunigungsprozesses wird die Phasendifferenz zwischen den beiden HF-Systemen des Quell- und Zielsynchrotrons unter Zuhilfenahme eines synchronisation reference signal ermittelt. Die Phasendifferenz auf dem RF-Flattop wird \"uber eine look-up-Table aus den
Phasendifferenzen, die zu definierten Zeitpunkten auf der Rampe gewonnen wurden ermittelt. Daraus l\"asst sich die ben\"otigte HF-Frequenzmodulation berechnen. Diese wird der urspr\"unglichen Frequenzrampe \"uberlagert. Die Integration der HF-Frequenzmodulation f\"uhrt dann bei RF-Flattop zu der ben\"otigten Phasendifferenz. Mit der so entstandenen Frequenzrampe, wird die gew\"unschte Phasendifferenz dann automatisch erzielt, wenn die HF-Frequenz von Quell- und Zielsynchrotron das Flattop erreichen.