FAIR hat zum Ziel, hochenergetische Ionenstrahlung für die Elemente Wasserstoff bis Uran, Antiprotonen und exotische Nuklide , mit h\"ochsten  Intensit\"aten zur erzeugen. Die existierende Beschleunigeranlage der GSI wie auch die zukünftige FAIR"=Anlage nutzt unterschiedliche Ringbeschleuniger wie beispielsweise Schwerionensynchrotrons (das SIS18 und das SIS100) und auch Speicherringe (den ESR, den CRYRING, den CR und den HESR) zur Pr\"aperation der Sekund\"arstrahlen und auch f\"ur die Experimente.  Eine stabiler Transfer von \textit{Bunchen} in \textit{Buckets} zwischen allen GSI"= und FAIR"=Ringbeschleuniger, ist aus verschiedenen Gr\"unden erforderlich. Bei einem nicht ordnungsgem\"a\ss{}en Strahltransfer besteht die Gefahr, dass es zu einer Degeneration der Strahlqualit\"at (z.B.  einer Emittanzerh\"ohung), bis hin zum Strahlverlust kommt. Ein stabiler \textit{Bunch"=to"=Bucket"=Transfer} zwischen zwei Ringen ist daher sehr wichtig f\"ur FAIR und ist das Thema, welches im Rahmen dieser Doktorarbeit untersucht wird. Obwohl bereits zwischen dem SIS18 und dem ESR ein B2B"=Transfer realisiert wurde, so ist diese L\"osung aufgrund verschiedener Einschr\"ankungen nicht nutzbar f\"ur FAIR. Es legt das alte GSI"=Kontrollsystem, ein eventbasiertes System zu Grunde, welches in Zukunft vollst\"andig durch das neue FAIR"=Kontrollsystem ersetzt werden wird. Das FAIR-Kontrollsystem basiert auf dem White"=Rabbit"=Netzwerk, dass eine Synchronisation im Sub-Nanosekundenbereich erlaubt. Des Weiteren unterst\"utzt das alte System nicht die \textit{phase shift method} und erm\"oglicht auch keine komplexen Buckte-Füllmuster. Zu einem Strahltransfer \"uber Targets ist es ebenfalls nicht in der Lage. Die Entwicklung eines \textit{FAIR B2B transfer system}s, basierend auf der f\"ur FAIR geplanten technischen Infrastruktur, dazu z\"ahlen das FAIR"=Kontrollsystem und das FAIR"=LLRF"=System ist daher unbedingt erforderlich.

Diese Doktorarbeit stellte erstmals die konzeptionelle Realisierung des \textit{FAIR B2B transfer system} vor. 
%Dieser nutz einen zweistufigen Synchronisationsprozess, die Grobs- und Feinsynchronisation. 
In den meisten F\"allen wird der B2B"=Transfer mit einem B2B"=Injektions"=Mittenversatz von unter $\pm 1^\circ$ innerhalb der oberen Zeitgrenze von 10ms erreicht. Das \textit{FAIR B2B transfer system} unterst\"utzt die \textit{phase shift method}, wie auch die \textit{frequency beating method} und ist anpassungsf\"ahig genug, um einen Transfer zwischen zwei Ringen mit beliebigem Verh\"altnis ihrere Umf\"ange zu erm\"oglichen. Es ist m\"oglich, verschiedene B2B"=Transfers zur gleichen Zeit auszuf\"uhren. Beispielsweise kann der B2B"=Transfer vom SIS18 zum SIS100 zur gleichen Zeit stattfinden, wie der B2B"=Transfer vom ESR zum CRYRING. Auch k\"onnen verschiedene Ionensorten von einem Maschinenzyklus zum Anderen transferiert werden. Das \textit{FAIR B2B transfer system} ist in der Lage, einen Transfer zwischen zwei Ringen auch \"uber das Antiprotonen"=Target, den Fragmentseparator oder den Suprafragmentseparator durchzuf\"uhren. Es k\"onnen verschiedene komplexe Bucket"=F\"ullmuster ber\"ucksichtigt werden. Au\ss{}erdem hat das \textit{FAIR B2B transfer system} eine Schnittstelle zum FAIR"=Maschinenschutzsystem, welches das SIS100 und die nachgeschalteten Beschleuniger und Experimente vor Schaden durch Prim\"arstrahlen bei Fehlerfunktionen bewahrt. 

Als n\"achstes wurde eine Liste von Kriterien vorgestellt, die f\"ur die HF"=Frequenzmodulation bei der \textit{phase shift method} die Erhaltung der Strahlqualit\"at bewertet. Zus\"atzlich wurde f\"ur den SIS18 Strahl das Strahlverhalten auf drei verschiedene HF"=Frequenzmodulationsmuster hin analysiert. Entsprechend der strahldynamischen Analysen, wird der Maximalwert f\"ur die HF"=Frequenzmodulation durch die Randbedingungen, die durch den \textit{momentum shift} gegeben sind eingeschr\"ankt. Die erste Ableitung der HF-Frequenzmodulation muss stetig und klein genug sein, um eine ausreichende Gr\"oße der umlaufenden \textit{buckets}  zu garantieren. Ein kleiner Wert der zweiten Ableitung garantiert,  dass sich die synchrone Phase langsam genug ändert, damit der Strahl folgen kann. Das spiegelt sich auch im adiabatischen Parameter wieder. Um eine \textit{bucket}"=Fl\"ache von größer 80$\%$  und einen adiabatischen Parameter von kleiner $10^{-4}$ f\"ur den SIS18 \SI{200}{MeV/u} $U^\mathit{28+}$ Strahl garantieren zu k\"onnen, muss $|\Delta f_{\mathit{rf}}|$ kleiner als \SI{8.137}{kHz} sein und $|\frac{d\Delta f_{\mathit{rf}}}{dt}|$ muss stetig und kleiner als \SI{95}{Hz/ms} sein. $|\frac{d^2\Delta f_{\mathit{rf}}}{dt^2}|$ muss kleiner als \SI{70}{Hz/ms^2} sein. Für den SIS18 \SI{4}{Gev} $H^{+}$ Strahl, muss $|\Delta f_{\mathit{rf}}|$ kleiner als \SI{283}{Hz} sein und $|\frac{d\Delta f_{\mathit{rf}}}{dt}|$ muss stetig und kleiner als \SI{1.9}{Hz/ms} sein. $|\frac{d^2\Delta f_{\mathit{rf}}}{dt^2}|$ muss kleiner als \SI{0.2}{Hz/ms^2} sein. Nach diesen Anforderungen wurden ein sinusförmiges und parabelförmiges HF-Frequenzmodulationsprofil mit einer bestimmten Zeitdauer f\"ur den SIS18 $U^{28+}$ Strahl \"uberpr\"uft. Beide Modulationsprofile erfüllen die Anforderungen und halten den Strahl stabil. Dennoch ist der adiabatische Parameter bei der sinusf\"ormigen Modulation kleiner als bei der parabelförmigen Modulation. Folglich sollte die sinusförmige Modulation bei der \textit{phase shift method} pr\"aferiert werden. Die sinusf\"ormige HF"=Frequenzmodulation im SIS18 f\"ur
\SI{200}{MeV/u} bei $U^\mathit{28+}$ ben\"otigt \SI{7}{\ms} und die sinusf\"ormige HF"=Frequenzmodulation im SIS18 f\"ur \SI{4}{GeV} bei $H^+$ ben\"otigt circa \SI{50}{\ms} f\"ur eine Phasenverschiebung jeweils um $\pi$.   

In Ergänzung zu den strahldynamischen Analysen, wurden zwei Messaufbauten errichtet. Der erste Messaufbau diente dazu, das WR"=Netzwerk f\"ur den B2B"=Transfer zu cha­rak­te­ri­sie­ren. Nach diesem Messergebnis, ist die zul\"assige Anzahl von WR"=Switch"=Layern f\"ur den B2B"=Transfer nicht nur von der Obergrenze der Latenzzeit abh\"angig (z.B. 400 ms), sondern auch von der tolerierbaren Frame"=Error"=Rate (FER) des B2B"=Transfer"=Systems. Wenn keine Vorw\"artsfehlerkorrektur f\"ur das B2B"=Netzwerk verwendet wird, ist die Anzahl der zul\"assigen WR"=Switches hauts\"achlich durch die FER bestimmt. Unter der Annahme, dass der Verlust von einem Frame, innerhalb von zwei Monaten noch akzeptable ist, sind maximal 38 WR"=Switche zul\"assig zwischen Data Master (DM) und der zugeh\"origen SCUs und maximal 8 WR"=Switche direkt zwischen den SCUs, die dem B2B"=Transfer"=System zugeordneten sind. Wird eine Vorw\"artsfehlerkorrektur f\"ur das B2B"=Netzwerk verwendet, so ist die Anzahl der zul\"assigen WR"=Switches durch die noch tolerierbare Latenzzeit bestimmt. In diesem Fall, sind dann 67 WR"=Switches zwischen den f\"ur das B2B"=Transfer"=System zugeh\"origen SCUs  und DM erlaubt und 13 WR"=Switches direkt zwischen den SCUs, die dem B2B"=Transfer"=System zugeordneten sind. Der zweite Messaufbau diente dazu, die Firmware die auf einer \textit{Soft"=CPU (LatticeMicro32)} in der SCU ausgef\"uhrt wird, f\"ur das \textit{B2B transfer system} zu evaluieren. Gemessen wurden die Laufzeiten für die einzelnen Tasks in der Firmware. Es wurde nachgewiesen, dass die Firmware auf dem LatticeMicro32 in der SCU die Anforderungen an die  Timing"=Bedingungen erfüllt, wenn der zugeh\"orige \textit{System-on-Chip bus} nicht zur gleichen Zeit mit anderen Anwendung, die parallel zur B2B-Firmware ausgef\"uhrt werden belegt ist.

Des Weiteren wurden die Auswirkung der Fehlerfortpflanzung durch die Messunsicherheit (z.B. die Phasenmessgenauigkeit von $\pm 0.1^\circ$, der BuTiS C2"=Clock"=Stabilit\"at von die \SI{100}{ps} und die Messgenauigkeit für den Zeitstempel von 1 ns) f\"ur die Zeit bis zum \textit{phase alignment} in allen FAIR-Anwendungsf\"allen im Rahmen dieser Doktorarbeit überpr\"uft. Der B2B Mittenversatz bei Injektion verschlechtert sich durch die Unsicherheiten bei der Bestimmung des Zeitpunkts des \textit{phase alignment} auf unterschiedliche Art und Weise. In einigen Anwendungsfällen verschlechtert sich der B2B Mittenversatz bei Injektion sehr deutlich. Beispielsweise verschlechtert sich der Mittenversatz um $37\%$ f\"ur den Transfer von $U^\mathit{28+}$  vom SIS18 in den SIS100, was in allen FAIR-Anwendungsf\"allen den Worst"=Case darstellt. Trotz dieser Verschlechterung wird die Anforderung von kleiner $\pm1^\circ$ eingehalten. Daher ist die Messunsicherheit noch akzeptabel für das \textit{FAIR B2B transfer system}. Zus\"atzlich wurden im Rahmen dieser Doktorarbeit auch die Genauigkeitsanforderungen an den Start des Synchronisationsfensters f\"ur alle FAIR"=Anwendungsf\"alle \"uberpr\"uft. Der h=1 B2B"=Transfer vom SIS18 zum ESR stellt mit ca. \SI{500}{\ns} die strengsten Genauigkeitsanforderungen an den Beginn des Synchronisationsfesters. 

Au\ss{}erdem wurden verschiedene Trigger"=Szenarien für die SIS18 Extraktions"= und die SIS100 Injektions"=Kicker"=Magneten untersucht. Die neun Extraktions"=Kicker"=Magneten des SIS18, sind auf zwei Tanks aufgeteilt. Die Kicker"=Magneten jedes Tanks k\"onnen gleichzeitig gez\"undet werden, wenn die Bunchl"ucke mindestens $25\%$ der Kavit\"aten-HF-Periode beträgt. Die vier Kicker-Magneten in dem zweiten Tank, k\"onnen nach einer festen Verz\"ogerungszeit nach dem ausl\"osen der f\"unf Kicker-Magneten im ersten Tank f\"ur alle Ionensorten ausgel\"ost werden, wenn die Bunchl\"ucke mindestens $25\%$ der Kavit\"aten-HF-Periode betr\"agt. Die sechs SIS100 Injektions"=Kicker"=Magneten sind gleichm\"a\ss{}ig in einem Tank verteilt. Sie k\"onnen unverz\"uglich f\"ur alle Ionensorten ausgel\"ost werden, wenn die Bunchl\"ucke mindestens $35\%$ der Kavitäten-HF-Periode betr\"agt.  

Zum Abschlu\ss{} wurde das \textit{FAIR B2B transfer system} unter Anwendung der  \textit{frequency beating method} f\"ur alle FAIR"=Anwendungsf\"alle veranschaulicht. Es wurde gezeigt, dass f\"ur alle Prim\"arstrahl"=Transfers in den FAIR-Anwendungsf\"allen, bei Injektion einen B2B Mittenversatz von besser $\pm1^\circ$ innerhalb der erforderlichen Transferzeit von \SI{10}{\ms} erreicht wird, weil das Zahlenverh\"altnis der Umf\"ange der beiden Ringe ganzzahlig oder nahezu ganzzahlig ist. Entwicklungsbedarf besteht noch beim Ionentransfer von Sekund\"arstrahlen, wie sie vom Antiprotonen"=Target, dem Fragmentseparator oder dem Suprafragmentseparator erzeugt werden. Hier besteht das Problem, dass das Verh\"altnis der Energien zwischen Prim\"ar"=und
Sekund\"arstrahl sich stark unterscheiden. F\"ur den Transfer von exotischen Nukliden vom SIS100 zum CR \"uber den Suprafragmentseparator mit \SI{1.5}{GeV/u} Prim\"arstrahlenergie und \SI{740}{MeV/u} Sekund\"arstrahlenergie, betr\"agt der B2B Injektions"=Mittenversatz zwar zufällig nur $\pm2.1^\circ$, f\"ur den Antiprotonen B2B"=Transfer vom SIS100 zum CR \"uber das Antiprotonen"=Target und den Strahltransfer von exotischen Nukliden vom SIS18 zum ESR \"uber den Fragmentseparator, ist der B2B Mittenversatz aber schon gr\"o\ss{}er als $\pm40^\circ$ und damit weit au\ss{}erhalb der Spezifikation.

Die vorliegende Doktorarbeit stellt die wesentlichen Untersuchungsergebnisse f\"ur das \textit{FAIR B2B transfer system}, aus Sicht der Strahldynamik, Timing"=Anforderung und Kicker"=Ausl\"osung vor. Dennoch bleiben weitere Untersuchungsaufgaben, die für den finalen Anlagenbetrieb notwendig sind offen. Dazu z\"ahlen:
\begin{itemize}

\item Die Synchronisation im Mikrosekundenbereich des Antiprotonenstrahls mit dem magnetischen Horn nach dem Antiprotonen"=Target.
\item Die Synchronisation zwischen dem SIS100 Bunchkompressor und der Strahlextraktion.
\item In einige FAIR-Anwendungsf\"allen ist f\"ur den Transfer des Sekund\"arstrahls der B2B Injektions"=Mittenversatz gr\"o\ss{}er als $\pm40^\circ$.  F\"ur diese Anwendungsf\"alle muss \"uberpr\"uft werden, ob der B2B"=Transfer unter Zuhilfenahme von speziellen Strahlakkumulations"=Methoden wie Beispielsweise die \textit{barrier bucket method} oder die \textit{unstable point accumulation method} gelingt.
\end{itemize}

Das \textit{FAIR B2B transfer system} das in dieser Doktorarbeit vorgestellt wird, ist anwendbar f\"ur alle FAIR"=Anwendungsf\"alle. Dennoch gibt es Verbesserungspotential. F\"ur die \textit{phase shift method} muss die HF"=Frequenzmodulation sehr langsam erfolgen, damit der Strahl der Frequenz\"anderung folgen kann (z.B. \SI{7}{\ms}/\SI{50}{\ms} bei sinusf\"ormiger Modulation für den SIS18 $U^\mathit{28+}$/$H^\mathit{+}$ Strahl). Um \textit{Bunche} sobald wie m\"oglich in \textit{Buckets} transferieren zu k\"onnen, kann mit der Phasenverschiebung bereits auf der Beschleunigungsrampe begonnen werden. Zu einem definierten Zeitpunkt w\"ahrend des Beschleunigungsprozesses wird die Phasendifferenz zwischen den beiden HF"=Systemen der Quell"= und Zielmaschine unter Zuhilfenahme eines \textit{synchronisation reference signal} ermittelt. Die Phasendifferenz auf dem \textit{rf flattop} wird \"uber eine Look"=Up"=Table aus den Phasendifferenzen, die zu definierten Zeitpunkten auf der Rampe gewonnen wurden ermittelt. Daraus l\"asst sich die ben\"otigte HF"=Frequenzmodulation berechnen. Diese wird der urspr\"unglichen Frequenzrampe augenblicklich \"uberlagert. Mit der so entstandenen Frequenzrampe, wird die gew\"unschte Phasendifferenz dann automatisch erzielt, wenn die HF"=Frequenz der Quellmaschine das \textit{rf flattop} erreicht.