Transferring bunches of particles from a synchrotron into specified buckets of another synchrotron has several underlying basic principles. The energy of the beam is same before and after the B2B transfer, so the energy of the source synchrotron must first of all match that of the target synchrotron. Principally speaking, every synchrotron has its independent RF system. Then the phase advance between the bunch and the bucket must be precisely controlled before the bunch is ejected. The process of achieving the detailed phase adjustment between two RF systems is termed ''RF synchronization''. For the correct bucket injection, the filled buckets and the bucket to be filled must be marked. The bunch fast extraction must happen exactly one ''time of flight'' before the required bucket of the target synchrotron passes the injection region. The injection kicker must kick when the bucket passes the injection region.  In this chapter, all of the B2B basic principles will be explained.  
%%%%%%%%%%%%%%%%%%%%%%%%%%%%%%%%%%%%%%%%%%%%%%%%%%%%%%%%%%%%%%%%%%%%%%%%%%%%%%%%
\section{Energy match}

The bunch coordinates in the longitudinal phase plane of the source synchrotron, just before transfer, must be accurately controlled, according to the bucket to be filled ~\cite{garoby_timing_1984}. The target synchrotron has to center the bucket on the desired orbit~\footnote {Design orbit or injection orbit}, according to the energy of the bunch. This requirement guarantees the energy match between the bunch and bucket. The energy of a beam is determined by the 'magnetic rigidity', which is defined as the following:
\begin{equation}
	\label{eq:energy}
	B(t)\rho_0 =\frac{p(t)}{e}
\end{equation}
where p(t) is the magnitude of the particle momentum, e is the charge of the particle, B(t) is magnetic field, and $\rho_0$ is the bending radius of a particle immersed in a magnetic field B(t). The ratio of p(t) to e describes the 'stiffness’ of a beam, it can be considered as a measure of how much angular deflection results when a particle travels through a given magnetic field~\cite{barletta_overview_????}.

The bunch is transferred from the source to the target synchrotron with the same energy. So the beam has the same momentum and velocity for both synchrotrons. According to eq. ~\ref{eq:energy}, the magnetic rigidity of two synchrotrons must be matched:

\begin{equation}
	\label{eq:rigidity}
	B^{src}(t)\rho_0^{src} =\frac{p}{e}=B^{trg}(t)\rho_0^{trg}
\end{equation}

Where the superscript of the symbol denotes the synchrotron, ``src`` represents the source synchrotron and ``trg`` the target synchrotron.

Besides, the rf frequency of two synchrotrons must meet the following relation ~\cite{garoby_timing_1984}.
\begin{equation}
	\label{eq:velocity}
	C^{src}\frac{f_{rf}^{src}(t)}{h^{src}} = \beta c=C^{trg}\frac{f_{rf}^{trg}(t)}{h^{trg}}
\end{equation}

where C is the circumference of the synchrotron, h the harmonic number of the rf signal and $f_{rf}$ denotes the cavity rf frequency, $\beta$  the fraction of the particle velocity to the lightspeed.

%%%%%%%%%%%%%%%%%%%%%%%%%%%%%%%%%%%%%%%%%%%%%%%%%%%%%%%%%%%%%%%%%%%%%%%%%%%%%%%%
\section{Loop freeze}

During the B2B transfer process, feedback loops for the deviations correction of the particles from reference states (e.g. position and velocity) must switch off or freezen. E.g. Beam phase feedback loop~\cite{grieser_beam_2015} and bunch-by-bunch longitudinal rf feedback loop~\cite{gross_bunch-by-bunch_2015}.

%%%%%%%%%%%%%%%%%%%%%%%%%%%%%%%%%%%%%%%%%%%%%%%%%%%%%%%%%%%%%%%%%%%%%%%%%%%%%%%%
\section{Phase difference between two RF systems}
\label{sec:phase_diff}
For the RF synchronization between two synchrotrons, the prerequisite is to know the phase difference between two independent RF systems. 
  
%%%%%%%%%%%%%%%%%%%%%%%%%%%%%%%%%%%%%%%%%%%%%%%%%%%%%%%%%%%%%%%%%%%%%%%%%%%%%%%%
\section{RF synchronization}
\label{two_sync_methods}

There are usually two methods available for the synchronization process. The synchronization is achieved by an azimuthal positioning of the bunch in the source synchrotron or the bucket in the target synchrotron. This is so-called ''phase shift method''. When two rf frequencies are slightly different, they are beating, perceived as periodic variations in phase difference, whose rate is the difference between the two frequencies. The synchronization is automatically achieved. This is so-called ''frequency beating method''. Both methods provide a time frame for the B2B transfer, within which a bunch could be transferred into a bucket with the bunch-to-bucket center mismatch smaller than the upper bound. The time frame is called ``synchronization window``. 

For both methods, the accompanying beam dynamics must be taken into consideration. The momentum of particle is given by 
\begin{equation}
\label{eq:momentum}
p(t)=e\rho_0 [\frac {R(t)}{R_0}]^{1/\alpha_p }B(t) 
\end{equation}

where $R_0$ is its nominal value, R(t) the orbit radius and $\alpha_p$, the momentum compaction factor. From eq. ~\ref{eq:momentum}, the first-order total differential of p(t) is given as

\begin{equation}
\label{eq:1st_momentum}
dp(t)=\frac{e\rho_0}{\alpha_p (R_0)^{1/\alpha_p}}B(t)R(t)^{1/\alpha_p-1}dR(t)+ e\rho_0 [\frac {R(t)}{R_0}]^{1/\alpha_p }B(t)dB(t) 
\end{equation}

Dividing both sides of eq. ~\ref{eq:1st_momentum} by p(t), we obtain
\begin{equation}
\label{eq:pRB}
\frac{dp(t)}{p(t)}={\gamma_t^2}\frac{dR(t)}{R(t)}+\frac{dB(t)}{B(t)} 
\end{equation}

Now, for circular accelerators, the following general relation holds
\begin{equation}
\label{eq:frequency}
f(t)=\frac{\upsilon(t)}{2\pi R(t)} 
\end{equation}
where f(t) is the revolution frequency and $\upsilon(t)$ the velocity. The total differential of f(t) is given by

\begin{equation}
\label{eq:1st_frequency}
df(t)=\frac{1}{2\pi}[\frac{d\upsilon(t)}{R(t)}- \frac{\upsilon(t)}{R^2(t)}dR(t)]
\end{equation}

Dividing both sides of eq. ~\ref{eq:1st_frequency} by f(t) yields
\begin{equation}
\label{eq:fvr}
\frac{df(t)}{f(t)}=\frac{d\upsilon(t)}{\upsilon(t)}- \frac{dR(t)}{R(t)}
\end{equation}

The fractional change in $\upsilon(t)$ is related to the fractional change in p(t):
\begin{equation}
\label{eq:pv}
\frac{dp(t)}{p(t)}=\gamma^2(t)\frac{d\upsilon(t)}{\upsilon(t)}
\end{equation}
where $\gamma(t)$  is the relativistic factor, which measures the total particle energy, E(t), in units of the particle rest energy, $E_0$. Solving $d\upsilon(t)/\upsilon(t)$ from eq. ~\ref{eq:pv} and substituting it into eq. ~\ref{eq:fvr} yields

\begin{equation}
\label{eq:fPR}
\frac{df(t)}{f(t)} ={\gamma^2(t)}\frac{dp(t)}{p(t)}-\frac{dR(t)}{R(t)} 
\end{equation}

Replacing dp(t)/p(t) in eq.~\ref{eq:fPR} with eq.~\ref{eq:pRB}, we have
\begin{equation}
\label{eq:fBR}
\frac{df(t)}{f(t)} ={\gamma^2(t)}\frac{dB(t)}{B(t)}+[\frac{\gamma_t^2}{\gamma^2(t)}-1]\frac{dR(t)}{R(t)} 
\end{equation}

where $\gamma_t$ is the transition gamma, which is related to $\alpha_p$ as $\gamma_t=1/\sqrt{\alpha_p}$. In the same way, solving dR(t)/R(t) from eq. ~\ref{eq:pRB} and substituting it into eq. ~\ref{eq:fPR}, we obtain
\begin{equation}
\label{eq:fPB}
\frac{df(t)}{f(t)} =(\frac{1}{\gamma^2(t)}-\frac{1}{\gamma_t^2}) \frac{dp(t)}{p(t)}+\frac{1}{\gamma_t^2}\frac{dB(t)}{B(t)} 
\end{equation}

Of the four variables, f(t), B(t), p(t) and R(t), only two are independent. This leads to four very useful differential relations, eq. ~\ref{eq:pRB}, eq. ~\ref{eq:fPR}, eq. ~\ref{eq:fBR} and eq. ~\ref{eq:fPB} ~\cite{ezura_beam-dynamics_2008, bovet_selection_1970}. 

%%%%%%%%%%%%%%%%%%%%%%%%%%%%%%%%%%%%%%%%%%%%%%%%%%%%%%%%%%%%%%%%%%%%%%%%%%%%%%%%%%%%%%%%%%%%%%%%%%%
\subsection{Phase shift method}

Based on the phase difference on Sec. \ref{sec:phase_diff}, the rf system of the source or target or both synchrotrons are modulated away from their nominal value for a period of time and then modulated back so that the phase shift created by the frequency modulation could compensate for the expected phase difference. After the phase shift, the bunches of the source synchrotron are synchronized with random buckets of the target synchrotron. The phase shift process must be performed adiabatically for the longitudinal emittance to be preserved.

\begin{figure}[!htb]
   \centering   
   \includegraphics*[width=160mm]{phase_shift.png}
   \caption{The illustration of the phase shift method.}
   \label{phase_shift}
\end{figure}

Fig. 1 illustrates the phase shift method. The top and bottom RF signals are respectively from the source and target synchrotrons. For the phase shift method two RF signals are of the same frequency. The blue dots show the position of the bunches of the source synchrotron, the red dots correspond to the bucket positions of the target synchrotron. The time-of-flight between the bunch and bucket is compensated here. The red dashed line shows the end of the phase shift process and the beginning of the synchronization window, drawn in yellow. After the phase shift, bunches match with the random buckets.  

A particular case of the B2B synchronization occurs, when the target synchrotron is empty, i.e. it did not capture any bunch yet, the phase shift can be done for the target synchrotron without adiabatical consideration (e.g. Phase jump is possible).

%%%%%%%%%%%%%%%%%%%%%%%%%%%%%%%%%%%%%%%%%%%%%%%%%%%%%%%%%%%%%%%%%%%%%%%%%%%%%%%%%%%%%%%%%%%%%%%%%%%
Now we analyze the rf frequency modulation of the phase shift from the the beam dynamics viewpoint.
\begin{itemize}
	\item Radial excursion and momentum shift due to rf frequency modulation

For the phase shift method, the magnetic field is not affected by the frequency modulation, so $\Delta{B}$ = 0. By substituting $\Delta{B}$ = 0 into eq. ~\ref{eq:fBR} and eq. ~\ref{eq:fPB}, we could get respectively the accompanying radial excursion and momentum shift by the frequency modulation.

\begin{equation}
\label{eq:phaseR}
\frac{\Delta{f}}{f} =({\frac{\gamma_t^2}{\gamma^2}-1}) \frac{\Delta{R}}{R}
\end{equation}
and
\begin{equation}
\frac{\Delta{f}}{f} = (\frac{1}{\gamma^2}-\frac{1}{\gamma_t^2})\frac{\Delta{p}}{p}
\label{eq:phaseP}
\end{equation}

%%%%%%%%%%%%%%%%%%%%%%%%%%%%%%%%%%%%%%%%%%%%%%%%%%%%%%%%%%%%%%%%%%%%%%%%%%%%%%%%%%%%%%%%%%%%%%%%%%%
	\item Transverse dynamics analysis

The momentum spread ${\Delta{p}}/{p} \neq 0$ during the phase shift process causes chromaticity drift $\Delta{Q}$. $Q^`$is the chromaticity of the machine ~\cite{holzer_introduction_2013}.

\begin{equation}
\Delta{Q} = Q^`\frac{\Delta{p}}{p}
\label{eq:chromaticity}
\end{equation} 
%%%%%%%%%%%%%%%%%%%%%%%%%%%%%%%%%%%%%%%%%%%%%%%%%%%%%%%%%%%%%%%%%%%%%%%%%%%%%%%%%%%%%%%%%%%%%%
	\item Shift of synchronous phase

The synchronous phase deviates from $0^\circ$ during the frequency modulation. From the expression of the particle momentum, p(t), given in eq. ~\ref{eq:momentum}, the time derivative of p(t) can be written as
\begin{equation}
\frac {dp(t)}{dt} = \frac {e\rho_0B(t)}{\alpha_pR_0^{1/\alpha_p}}R(t)^{1/\alpha_p-1}\frac{dR(t)}{dt}+e\rho_0 (\frac {R(t)}{R_0})^{1/\alpha_p }\frac{dB(t)}{dt}
\label{eq:momentum/t}
\end{equation} 
Now, the relationship between the rate of change in momentum of a particle, dp(t)/dt,
and the force applied on it, F(t), is governed by Newton’s second law:
\begin{equation}
\frac {dp(t)}{dt} = F(t)
\label{eq:Newton}
\end{equation} 
F(t) is given by the product of the accelerating electric field, E(t), and the
charge of particle, e. Substituting dp(t)/dt given in eq. ~\ref{eq:momentum/t} and F(t) = eE(t) into eq.~\ref{eq:Newton}, we have
\begin{equation}
 \frac {e\rho_0B(t)}{\alpha_pR_0^{1/\alpha_p}}R(t)^{1/\alpha_p-1}\frac{dR(t)}{dt}+e\rho_0 (\frac {R(t)}{R_0})^{1/\alpha_p }\frac{dB(t)}{dt}=eE(t)
\label{eq:f=eq}
\end{equation} 

From this equation, we obtain the expression of energy gain in one turn,
\begin{equation}
2\pi R_0 [\frac {e\rho_0B(t)}{\alpha_pR_0^{1/\alpha_p}}R(t)^{1/\alpha_p-1}\frac{dR(t)}{dt}+e\rho_0 (\frac {R(t)}{R_0})^{1/\alpha_p }\frac{dB(t)}{dt}]=eV(t)sin[\phi_{s0}(t)+\Delta \phi_s(t)]
\label{eq:energy_cycle}
\end{equation} 
where V(t) is the RF accelerating voltage per turn; $\phi_{s0}$, the synchronous phase in the
operation with no frequency modulation; and $\Delta\phi_{s}(t)$, the change in the synchronous phase originating from the rf frequency modulation.

The magnetic field is not affected by the frequency change, we can assume dB(t)/dt = 0. Before the synchronization, it is a stationary bucket with the synchronous phase $0^\circ$. Then, eq.~\ref{eq:energy_cycle} reduce to
\begin{equation}
2\pi R_0 [\frac {e\rho_0B(t)}{\alpha_pR_0^{1/\alpha_p}}R(t)^{1/\alpha_p-1}\frac{dR(t)}{dt}]=eV(t)sin[\Delta \phi_s(t)]
\label{eq:energy_cycle_noB}
\end{equation} 

Solving  $\Delta \phi_{s}(t)$  from eq.~\ref{eq:energy_cycle_noB}, we have
\begin{equation}
\Delta \phi_{s}(t)=sin^{-1}[{\frac{2\pi \rho_0 B}{\alpha_pV}(\frac{R(t)}{R_0})^{1/\alpha_p-1}\frac{dR(t)}{dt}}]
\label{eq:delta_phase}
\end{equation} 
From eq.~\ref{eq:delta_phase}, we know that $\Delta \phi_{s}(t)$ is only determined by dR(t)/dt during the frequency modulation.
%%%%%%%%%%%%%%%%%%%%%%%%%%%%%%%%%%%%%%%%%%%%%%%%%%%%%%%%%%%%%%%%%%%%%%%%%%%%%%%%%%%%%%%%%%%%%%%%%%%
\item Bucket area factor

At the flattop, the bucket is a stationary bucket with $\phi_{s0}(t)=0$. During the frequency modulation process, the bucket becomes a running bucket with $\Delta\phi_s(t)\ne0$. The ratio of bucket areas of a running bucket to a stationary bucket is bucket area factor $\alpha(\Delta \phi_s)$. 
The bucket area factor could be estimated by ~\cite{lee_accelerator_2011}.
\begin{equation}
\alpha_b(\Delta\phi_s)\approx(1-sin(\Delta \phi_s))(1+sin(\Delta \phi_s))
\label{eq:buckt_area_factor}
\end{equation} 

%%%%%%%%%%%%%%%%%%%%%%%%%%%%%%%%%%%%%%%%%%%%%%%%%%%%%%%%%%%%%%%%%%%%%%%%%%%%%%%%%%%%%%%%%%%%%%%%%%%
\item Adiabaticity analysis

$\omega_s(t)$ is the small-amplitude synchrotron frequency given by
\begin{equation}
\omega_s(t) =[{-\frac{\eta(t)h\omega_{rev}^2(t)eV(t)cos{\phi_s(t)}}{2\pi\beta^2(t)E(t)}}]^{1/2}
\label{eq:synchfreq}
\end{equation} 

A process is called “adiabatic” when the RF parameters are changed slowly enough for the longitudinal emittance to be preserved. The condition that the parameters are slowly varying can be expressed by
\begin{equation}
\varepsilon=\frac{1}{\omega_s^2(t)}|\frac{d\omega_s(t)}{dt}| \ll 1
\label{eq:adiabaticity}
\end{equation} 

Compared with $\phi_s(t)$, all of the other variables change very slowly. $\phi_s(t)=\phi_{s0}(t)+\Delta\phi_s(t)$. From eq.~(\ref{eq:adiabaticity}) and eq.~(\ref{eq:synchfreq}), we can write the adiabaticity parameter $\varepsilon$, as follows~\cite{ezura_beam-dynamics_2008}:
\begin{equation}
\varepsilon \approx \frac{1}{2\omega_{s0}(t)}|tan\phi_{s}(t)\frac{d\phi_s(t)}{dt}|
\label{eq:derivation}
\end{equation} 

%%%%%%%%%%%%%%%%%%%%%%%%%%%%%%%%%%%%%%%%%%%%%%%%%%%%%%%%%%%%%%%%%%%%%%%%%%%%%%%%%%%%%%%%%%%%%%%%%%%
\item Constraints on the RF frequency modulation

From eq.~\ref{eq:derivation}, we can clearly see that $\phi_s(t)$ and $d\phi_s(t)/dt$ play deterministic roles for the adiabaticity when the frequency is modulated. Now let us deduce how the rf frequency modulation affects $\phi_s(t)$ and $d\phi_s(t)/dt$. From eq.~(\ref{eq:phaseR}), we could get the following equation.
\begin{equation}
\frac{dR(t)}{dt}(\frac{\gamma_t^2}{\gamma^2}-1)f_0=\frac{df(t)}{dt} R_0
\label{eq:RtoF}
\end{equation}


Substituting eq.~\ref{eq:RtoF} into eq.~\ref{eq:energy_cycle_noB}, we get
\begin{equation}
Vsin\phi_s=\frac{2\pi R_0 \rho B}{f_0(\frac{1}{\gamma}^2-\frac{1}{\gamma_t}^2)}[\frac{R(t)}{R_0}]^{(\frac{1}{\alpha_p}-1)}\frac{df(t)}{dt} 
\label{eq:bucketsizeF}
\end{equation}

Because $(R(t)-R_0)/R_0$ is about $10^{-4}$, $[1+\frac{\Delta R}{R_0}]^{(\frac{1}{\alpha_p}-1)}\approx 1$. We can get the relation between df(t)/dt and $\phi_s$ from eq.~\ref{eq:bucketsizeF}.
\begin{equation}
Vsin\phi_s=\frac{2\pi R_0 \rho B}{f_0(\frac{1}{\gamma}^2-\frac{1}{\gamma_t}^2)}\frac{df(t)}{dt} 
\label{eq:dotf}
\end{equation}

From eq.~\ref{eq:buckt_area_factor}, we know that the bucket area factor is determined by the synchronous phase change $\Delta\phi_s$. Based on eq.~\ref{eq:dotf}, we know the synchronous phase $\Delta\phi_s$ is determined by df(t)/dt, so df(t)/dt is important for the bucket size.

In order to get the relation between $d\phi_s(t)/dt$ and the frequency modulation, we get the time derivative of eq.~\ref{eq:dotf}

\begin{equation}
Vcos\phi_s\frac{d\phi_s}{dt}=\frac{2\pi R_0 \rho B}{f_0(\frac{1}{\gamma}^2-\frac{1}{\gamma_t}^2)}\frac{df(t)/dt}{dt} 
\label{eq:2dotf}
\end{equation}
\label{3_criteria}
Based on the adiabaticity eq.~(\ref{eq:derivation}), $d\phi_s(t)/ dt$ must be existing and small enough. So $\frac{df(t)/dt}{dt}$ must be existing and small enough. It means that df(t)/dt and $\phi_s(t)$ must be continuous. In a word, there are three constraints for the rf frequency modulation.
\begin{itemize}
\item[-] The df(t)/dt of the rf frequency modulation must be small enough to guarantee the bucket size.
\item[-] The df(t)/dt of the rf frequency modulation must be continuous to guarantee the continuous synchronous phase.
\item[-] The df(t)/dt/dt of the rf frequency modulation must be small enough to guarantee the change of the synchronous phase slow enough for the beam to follow.
\end{itemize}

\end{itemize}
%%%%%%%%%%%%%%%%%%%%%%%%%%%------------------------------%%%%%%%%%%%%%%%%%%%%%%%%%%%%%%%%%%%%%

\subsection{Frequency beating method}

The frequency beating method uses the effect of two RF signals of slightly different frequencies, perceived as periodic variations in phase difference whose rate is the difference between the two frequencies. The RF frequency of the source or the target or both synchrotrons is detuned long before the ejection, then the difference between the phase of the bunch and bucket is measured. Based on the measured phase, the synchronization is realized when the phase difference of the two RF frequencies corresponds to the ideal phase difference ($\Delta \theta = 0^\circ$). The $\Delta \theta$ is the mismatch between the bunch center and the corresponding bucket center. Because of the slightly different RF frequencies, a mismatch between the bunch and bucket centers exists. In principle, the B2B transfer requirement for FAIR allows a bunch to bucket center mismatch of $1^\circ$, which brings a symmetric time frame with respect to the time of the ideal phase difference, resulting in the maximum synchronization window for the frequency beating method, drawn in yellow, see Fig. ~\ref{frequency_beat}. The red dashed line shows the time for the expected phase difference.

\begin{figure}[!htb]
   \centering   
   \includegraphics*[width=160mm]{frequency_beating.png}
   \caption{The illustration of the frequency beating method.}
   \label{frequency_beat}
\end{figure}

The RF frequency is detuned at the end of the ramp. During the rf frequency detune process, the magnetic field and radius excursion react and the momentum is not affected for the energy match.

\begin {itemize}
\item Longitudinal dynamics analysis

Because the momentum is not affected by the frequency change, namely $\Delta$p = 0, the general relation between the radial excursion and RF frequency change eq.~\ref{eq:fPR} reduces to eq.~\ref{eq:eq4} and the general relation between the magnetic field change and RF frequency change eq.~\ref{eq:pRB} reduces to eq.~\ref{eq:eq5}.

\begin{equation}
\frac{\Delta{f}}{f} = - \frac{\Delta{R}}{R}
\label{eq:eq4}
\end{equation}

\begin{equation}
\frac{\Delta{f}}{f} =  \frac{1}{{\gamma_t}^2}\times{\frac{\Delta{B}}{B}}
\label{eq:eq5}
\end{equation}
\end {itemize}

%%%%%%%%%%%%%%%%%%%%%%%%%%%%%%%%%%%%%%%%%%%%%%%%%%%%%%%%%%%%%%%%%%%%%%%%%%%%%%%%


%\subsubsection{Example of frequency beating method for SIS18 and SIS100 1 Seite}
%Because the circumference ratio of the large synchrotron to the small synchrotron is a perfect integer, the rf frequency at the flattop of SIS18 is same as that of SIS100. So the first step for the bunch to bucket transfer is the RF frequency de-tune. In order to realize the frequency beating between two synchrotrons, the RF frequency of the source synchrotron or the target synchrotron or both synchrotrons can be de-tuned. It means that the particles on the de-tuned synchrotron run at an average radius different by $\bigtriangleup$R from the designed orbit R. For the synchronization of the SIS18 and the SIS100, we will de-tune the RF frequency on the SIS18. The SIS18 operates with a cycle length of 520ms, harmonic number of 2 ( h = 2 ), and RF frequency of approximately 0.43 MHz at injection and approximately 1.57 MHz at ejection for the $U^{28+}$~\cite{SIS18}. During nominal operation, the SIS18 forms two bunches from the beam injected at 11.4 MeV/$\mu$ and accelerates them up to 200 MeV/$\mu$. From the SIS18, 4 batches, each of 2 bunches, are transferred at  maximum 10ms intervals to the SIS100. The harmonic number of the SIS100 is 10 and the SIS100 RF frequency is fixed at approximately 1.57 MHz during the
%injection period to simplify the RF control system and to avoid perturbing batches already transferred.
%
%  This RF frequency de-tune is done accompanying with the RF ramp. Accepting to decentre the orbit by 8mm for the SIS18~\cite{SIS18_man}: 
%
%\begin{equation}
%\frac{\bigtriangleup{R}}{R}\approx{2.4}{\times}10^{-4}\label{eq1}
%\end{equation}
%
%  We know the basic differential relations among the fractional change in the RF frequency f, the fractional change in the momentum p, the fractional change in the bending magnetic field B and the fractional change in the radius R as follows ~\cite{J-PARC}.
%
%
%\begin{equation}
%\label{eq:eq2}
%\frac{\Delta{f}}{f} ={\frac{1}{\gamma^2}}{\frac{\Delta{p}}{p}} - \frac{\Delta{R}}{R}
%\end{equation}
%
%\begin{equation}
%\frac{\Delta{f}}{f} = (\frac{1}{\gamma^2}-\frac{1}{\gamma_t^2})\frac{\Delta{p}}{p}+{\frac{1}{\gamma_t^2}}{\frac{\Delta{B}}{B}}
%\label{eq:eq3}
%\end{equation}
%
%
%where $\gamma$ is the relativistic factor, which measures the total particle energy, E, in
%units of the particle rest energy, $E_0$; $\gamma_t$ is the transition gamma; $\bigtriangleup{f}$ and  $\bigtriangleup{B}$ are the frequency and  bending magnetic field deviation for the frequency de-tune;  $\bigtriangleup{p}$ is the momentum deviation.
%
%In our case of the frequency beating method, we guarantee the extraction and injection energy always match, which means that the momentum is not affected by the frequency change, namely $\Delta$p = 0; then the general relation between the radial excursion and RF frequency change eq.~(\ref{eq:eq2}) reduces to eq.~(\ref{eq:eq4}) and the general relation between the magnetic field change and RF frequency change eq.~(\ref{eq:eq3}) reduces to eq.~(\ref{eq:eq5}).
%
%\begin{equation}
%\frac{\Delta{f}}{f} = - \frac{\Delta{R}}{R}
%\label{eq:eq4}
%\end{equation}
%
%\begin{equation}
%\frac{\Delta{f}}{f} =  \frac{1}{{\gamma_t}^2}\times{\frac{\Delta{B}}{B}}
%\label{eq:eq5}
%\end{equation}

%\subsubsubsection{Frequency beating method for SIS18 and ESR 2-3 Seiten}
%Because the circumference ratio of the ESR injection orbit to the SIS18 designed orbit is not a perfect integer, two synchrotrons begin beating automatically. He 


\section{Bucket label}
After the synchronization, the bunch is synchronized to an arbitrary RF bucket. For the proper injection, we must know which buckets are already filled and which buckets should be filled by next injection cycle. The fast extraction can only proceed when the required bucket comes. 
%The extraction must be correctly synchronized with respect to a reference signal at the following frequency, which is called bucket marker.
%\begin{equation}
%	\label{eq:bucket_label}
%	\frac{f_{rf}^{src}}{p} = \frac{f_{rf}^{trg}}{q}
%\end{equation}

\section{Synchronization of the extraction and injection kicker}
For the proper B2B transfer, the extraction and injection kickers must be synchronized with the beam. The kicker time consists of the rise time, flat-top and fall time ~\cite{udo_injection_2014}.
 
\begin{itemize}
	\item Extraction kicker
		
Here we discuss that all bunches are extracted by one time extraction kick. The flattop is at least with the length of the bunches to be injected. The fall time is not constrained. If there is no empty RF bucket of the ring, the rise time of the extraction kicker must be shorter than the bunch gap. If there is at least one empty RF bucket, the rise of the magnetic field could be achieved within the gap of the empty RF buckets. 

	\item Injectin kicker

For multi-\gls{glos:batch} injection, the rise time of the injection kicker must be shorter than the \gls{glos:bunch_gap}. The flat-top is determined by the length of the bunches to be injected. If all buckets must be filled, the fall time must be shorter than the bunch gap. If at least one bucket is kept empty, the fall of the magnetic field could be achieved within the gap of the empty RF buckets. If the ring needs only one time injection, the rise time is not constrained. The flat-top determined by the length of the bunches to be injected. The fall time must be shorter than the bunch gap or the gap of the empty RF buckets. 

\end{itemize}

\section{Beam indication for the beam instrumentation}
In order to observe the beams and measure related parameters for accelerators and transfer lines ~\cite{forck_lecture_2011}, the beam instrumentation (\gls{BI}) equipments must be synchronized and triggered within the beam schedule. For the B2B transfer, the data acquisition for the beam instrumentation equipments should be triggered before the bunch is extracted. They should not be triggered too early because of the limitation of sampling time. So a pre-trigger is necessary, which indicates that the bunch will be extracted/injected soon. 


%%%%%%%%%%%%%%%%%%%%%%%%%%%%%%%%%%%%%%%%%%
%\bibliography{main}
%\bibliographystyle{plain}


