FAIR hat zum Ziel, hochenergetische Ionenstrahlen aller Elemente von Wasserstoff bis Uran, sowie von Antiprotonen und exotischen Nukliden mit h\"ochsten  Intensit\"aten zur erzeugen. Die existierende Beschleunigeranlage der GSI, wie auch die zukünftige FAIR"=Anlage nutzen unterschiedliche Ringbeschleuniger wie beispielsweise Schwerionensynchrotrons (das SIS18 und das SIS100) und Speicherringe (den ESR, den CRYRING, den CR und den HESR) zur Pr\"aparation der Sekund\"arstrahlen und auch f\"ur Experimente.  Eine stabiler Transfer von \textit{Bunches} in \textit{Buckets} zwischen allen GSI"= und FAIR"=Ringbeschleuniger ist aus verschiedenen Gr\"unden erforderlich. Bei einem nicht ordnungsgem\"a\ss{}en Strahltransfer besteht die Gefahr, dass es zu einer Degeneration der Strahlqualit\"at (z.B.  einer Emittanzerh\"ohung) bis hin zum Strahlverlust kommt. Ein stabiler \textit{Bunch"=to"=Bucket (B2B) Transfer} zwischen zwei Ringen ist daher sehr wichtig f\"ur FAIR und ist das Thema, welches im Rahmen dieser Doktorarbeit untersucht wurde. Die Entwicklung eines \textit{FAIR B2B Transfer System}s, basierend auf der f\"ur FAIR geplanten technischen Infrastruktur, dazu z\"ahlen das FAIR Timing"= und Kontrollsystem und das FAIR"=LLRF"=System, ist daher unbedingt erforderlich. Diese Doktorarbeit stellt erstmals die konzeptionelle Realisierung des \textit{FAIR B2B Transfer System} vor. 

%In den meisten F\"allen wird der B2B"=Transfer mit einem B2B"=Injektions"=Mittenversatz von unter $\pm 1^\circ$ innerhalb der oberen Zeitgrenze von \SI{10}{\ms} erreicht. Das \textit{FAIR B2B Transfer System} unterst\"utzt die \textit{Phase Shift Method}, wie auch die \textit{Frequency Beating Method} und ist anpassungsf\"ahig genug, um einen Transfer zwischen zwei Ringen mit beliebigem Verh\"altnis ihrer Umf\"ange zu erm\"oglichen. Es ist m\"oglich, verschiedene B2B"=Transfers zur gleichen Zeit auszuf\"uhren. Beispielsweise kann der B2B"=Transfer vom SIS18 zum SIS100 zur gleichen Zeit stattfinden, wie der B2B"=Transfer vom ESR zum CRYRING. Auch k\"onnen verschiedene Ionensorten von einem Maschinenzyklus zum anderen transferiert werden. Das \textit{FAIR B2B Transfer System} ist in der Lage, einen Transfer zwischen zwei Ringen auch \"uber das Antiprotonen"=Target, den Fragmentseparator oder den Superfragmentseparator durchzuf\"uhren. Es k\"onnen verschiedene komplexe Bucket"=F\"ullmuster ber\"ucksichtigt werden. Au\ss{}erdem hat das \textit{FAIR B2B Transfer System} eine Schnittstelle zum FAIR"=Maschinenschutzsystem, welches das SIS100 und die nachgeschalteten Beschleuniger und Experimente vor Schaden durch Prim\"arstrahlen bei Fehlerfunktionen bewahrt. 


Es wurde eine Liste von Kriterien vorgestellt, die f\"ur die HF"=Frequenzmodulation bei der \textit{Phase Shift Method} die Erhaltung der Strahlqualit\"at erm\"oglicht. Dazu wurde f\"ur den SIS18"=Strahl das Strahlverhalten auf drei verschiedene HF"=Frequenzmodulationsmuster hin analysiert. Entsprechend den strahldynamischen Analysen wird der Maximalwert f\"ur die HF"=Frequenzmodulation durch die Randbedingungen, die durch den \textit{momentum shift} gegeben sind, eingeschr\"ankt. Die erste Zeitableitung der HF-Frequenzmodulation muss stetig und klein genug sein, um eine ausreichende Gr\"oße der umlaufenden \textit{Buckets}  zu garantieren. Ein kleiner Wert der zweiten Zeitableitung garantiert,  dass sich die synchrone Phase langsam genug ändert, damit der Strahl folgen kann. Das spiegelt sich auch im adiabatischen Parameter wieder. Um eine Bucket"=Fl\"ache von größer 80$\%$  und einen adiabatischen Parameter von kleiner $10^{-4}$ f\"ur den SIS18 \SI{200}{MeV/u} $U^\mathit{28+}$ Strahl garantieren zu k\"onnen, muss $|\Delta f_{\mathit{rf}}|$ kleiner als \SI{8.137}{kHz} sein und $|\frac{d\Delta f_{\mathit{rf}}}{dt}|$ muss stetig und kleiner als \SI{95}{Hz/ms} sein. $|\frac{d^2\Delta f_{\mathit{rf}}}{dt^2}|$ muss kleiner als \SI{70}{Hz/ms^2} sein. Für den SIS18 \SI{4}{GeV/u} $H^{+}$ Strahl muss $|\Delta f_{\mathit{rf}}|$ kleiner als \SI{283}{Hz} sein und $|\frac{d\Delta f_{\mathit{rf}}}{dt}|$ muss stetig und kleiner als \SI{1.9}{Hz/ms} sein. $|\frac{d^2\Delta f_{\mathit{rf}}}{dt^2}|$ muss kleiner als \SI{0.2}{Hz/ms^2} sein. Nach diesen Anforderungen wurden ein sinusförmiges und parabelförmiges HF-Frequenzmodulationsprofil mit einer bestimmten Zeitdauer f\"ur den SIS18 $U^{28+}$ Strahl \"uberpr\"uft. Beide Modulationsprofile erfüllen die Anforderungen und halten den Strahl stabil. Dennoch ist der adiabatische Parameter bei der sinusf\"ormigen Modulation kleiner als bei der parabelförmigen Modulation. Folglich sollte die sinusförmige Modulation bei der \textit{Phase Shift Method} pr\"aferiert werden. Die sinusf\"ormige HF"=Frequenzmodulation im SIS18 f\"ur \SI{200}{MeV/u} bei $U^\mathit{28+}$ ben\"otigt \SI{7}{\ms} und die sinusf\"ormige HF"=Frequenzmodulation im SIS18 f\"ur \SI{4}{GeV/u} bei $H^+$ ben\"otigt circa \SI{50}{\ms} f\"ur eine Phasenverschiebung jeweils um $\pi$.   

In Ergänzung zu den strahldynamischen Analysen wurden zwei Messaufbauten errichtet. Der erste Messaufbau diente dazu, das WR"=Netzwerk f\"ur den B2B"=Transfer zu cha­rak­te­ri­sie­ren. Nach diesem Messergebnis ist die zul\"assige Anzahl von WR"=Switch"=Layern f\"ur den B2B"=Transfer nicht nur von der Obergrenze der Latenzzeit abh\"angig (z.B. 400 ms), sondern auch von der tolerierbaren Frame"=Error"=Rate (FER) des \textit{B2B Transfer System}s. Wenn keine Vorw\"artsfehlerkorrektur f\"ur das B2B"=Netzwerk verwendet wird, ist die Anzahl der zul\"assigen WR"=Switches hauts\"achlich durch die FER bestimmt. Unter der Annahme, dass der Verlust von einem Frame innerhalb von zwei Monaten noch akzeptabel ist, sind maximal 38 WR"=Switche zul\"assig zwischen Data Master (DM) und den zugeh\"origen SCUs und maximal 8 WR"=Switche direkt zwischen den SCUs, die dem B2B"=Transfer"=System zugeordnet sind. Wird eine Vorw\"artsfehlerkorrektur f\"ur das B2B"=Netzwerk verwendet, so ist die Anzahl der zul\"assigen WR"=Switches durch die noch tolerierbare Latenzzeit bestimmt. In diesem Fall sind dann 67 WR"=Switches zwischen den f\"ur das \textit{B2B Transfer System} zugeh\"origen SCUs  und DM und 13 WR"=Switches direkt zwischen den SCUs erlaubt, die dem B2B"=Transfer"=System zugeordnet sind. Der zweite Messaufbau diente dazu, die Firmware, die auf einer \textit{soft CPU (LatticeMico32)} in der SCU ausgef\"uhrt wird, f\"ur das \textit{B2B Transfer System} zu evaluieren. Gemessen wurden die Laufzeiten für die einzelnen Tasks in der Firmware. Es wurde nachgewiesen, dass die Firmware auf dem LatticeMico32 in der SCU die Anforderungen an die  Timing"=Bedingungen erfüllt, wenn der zugeh\"orige \textit{System-on-Chip bus} nicht zur gleichen Zeit mit anderen Anwendung, die parallel zur B2B-Firmware ausgef\"uhrt werden, belegt ist.

Des Weiteren wurde die Auswirkung der Fehlerfortpflanzung durch die Messunsicherheit f\"ur die Zeit bis zum \textit{phase alignment} in allen FAIR-Anwendungsf\"allen im Rahmen dieser Doktorarbeit überpr\"uft. Der B2B Mittenversatz bei Injektion verschlechtert sich durch die Unsicherheiten bei der Bestimmung des Zeitpunkts des \textit{phase alignment} auf unterschiedliche Art und Weise. In einigen Anwendungsfällen verschlechtert sich der B2B Mittenversatz bei Injektion sehr deutlich. Beispielsweise verschlechtert sich der Mittenversatz um $37\%$ f\"ur den Transfer von $U^\mathit{28+}$  vom SIS18 in den SIS100, was in allen FAIR-Anwendungsf\"allen den Worst"=Case darstellt. Trotz dieser Verschlechterung wird die Anforderung von kleiner $\pm1^\circ$ eingehalten. Daher ist die Messunsicherheit noch akzeptabel für das \textit{FAIR B2B Transfer System}. Zus\"atzlich wurden im Rahmen dieser Doktorarbeit auch die Genauigkeitsanforderungen an den Start des Synchronisationsfensters f\"ur alle FAIR"=Anwendungsf\"alle \"uberpr\"uft. Der h=1 B2B"=Transfer vom SIS18 zum ESR stellt mit ca. \SI{500}{\ns} die strengsten Genauigkeitsanforderungen an den Beginn des Synchronisationsfesters. 

Au\ss{}erdem wurden die Randbedingungen f\"ur die verschiedene Trigger"=Szenarien f\"ur die SIS18 Extraktions"= und die SIS100 Injektions"=Kicker"=Magnete untersucht. Die neun Extraktions"=Kicker"=Magnete des SIS18, sind auf zwei Tanks aufgeteilt. Die SIS18"=Kicker"=Magnete jedes Tanks k\"onnen gleichzeitig ausgelöst werden, wenn die Bunchl"ucke mindestens $25\%$ der Kavit\"aten-HF-Periode beträgt. Die vier SIS18"=Kicker-Magnete in dem zweiten Tank k\"onnen \"uber eine feste Verz\"ogerungszeit nach dem Ausl\"osen der f\"unf SIS18"=Kicker-Magnete im ersten Tank f\"ur alle Ionensorten ausgel\"ost werden, wenn die Bunchl\"ucke mindestens $25\%$ der Kavit\"aten-HF-Periode betr\"agt. Die sechs SIS100 Injektions"=Kicker"=Magnete sind gleichm\"a\ss{}ig in einem Tank verteilt. Sie k\"onnen f\"ur alle Ionensorten ausgel\"ost werden, wenn die Bunchl\"ucke mindestens $35\%$ der Kavitäten-HF-Periode betr\"agt.  

Zum Abschluss wurde das \textit{FAIR B2B Transfer System} unter Anwendung der \textit{Frequency Beating Method} f\"ur alle FAIR"=Anwendungsf\"alle veranschaulicht. Es wurde gezeigt, dass f\"ur alle Prim\"arstrahl"=Transfers in den FAIR"=Anwendungsf\"allen bei Injektion ein B2B Mittenversatz von besser als $\pm1^\circ$ innerhalb der erforderlichen Transferzeit von \SI{10}{\ms} erreicht wird, weil das Zahlenverh\"altnis der Umf\"ange der beiden Ringe ganzzahlig oder nahezu ganzzahlig ist.  Entwicklungsbedarf besteht noch beim Ionentransfer von Sekund\"arstrahlen, wie sie vom Antiprotonen"=Target, dem Fragmentseparator oder dem Superfragmentseparator erzeugt werden. Hier besteht das Problem, dass das Verh\"altnis der Energien zwischen Prim\"ar"=und Sekund\"arstrahl sich stark unterscheiden. F\"ur den Transfer von exotischen Nukliden vom SIS100 zum CR \"uber den Superfragmentseparator, betr\"agt der B2B Injektions"=Mittenversatz zwar zufällig nur $\pm2.1^\circ$, f\"ur den Antiprotonen B2B"=Transfer vom SIS100 zum CR \"uber das Antiprotonen"=Target und den Strahltransfer von exotischen Nukliden vom SIS18 zum ESR \"uber den Fragmentseparator ist der B2B Mittenversatz aber schon gr\"o\ss{}er als $\pm40^\circ$ und damit weit au\ss{}erhalb der Spezifikation.

Diese Doktorarbeit spielt eine wichtige Rolle bei der Realisierung des \textit{FAIR B2B Transfer System} und der weiteren praktischen Anwendung des Systems f\"ur alle FAIR"=Transferszenarien.
